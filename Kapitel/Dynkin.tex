\section{Dynkinsysteme}
\imaginarysubsection{Dynkinsysteme}
\thispagestyle{pagenumberonly}
\begin{definition}[Dynkinsystem]
    Ein Mengensystem $\mD \subseteq\mP\of{X}$ heißt Dynkinsystem, falls
    \begin{enumerate}[label=($\text{D}_{\arabic*}$)]
        \item $X\in\mD$
        \item $D\in\mD \impl \comp{D}\in\mD$
        \item Für eine paarweise disjunkte Mengenfolge $(D_n)_n \subseteq \mD \impl \dsty\bigsqcup_{n\in\N} D_n \in\mD$
    \end{enumerate}
\end{definition}

\begin{beispiel}
    \theoremescape
    \begin{enumerate}
        \item Jede $\sigma$-Algebra ist ein Dynkinsystem.
        \item Sei $X$ eine $2n$-elementige Menge. Dann ist $\mD \coloneqq \set{A \subseteq X: A\text{ hat eine gerade Anzahl an Elementen}}$ ein Dynkinsystem, aber keine $\sigma$-Algebra.
    \end{enumerate}
\end{beispiel}

\begin{lemma}
    Sei $I$ eine beliebige Indexmenge und $(\mD_j)_{j\in I}$ eine Familie von Dynkinsystemen in $X$, dann ist $\displaystyle\bigcap_{j\in I} \mD_j$ wieder ein Dynkinsystem.
    \begin{proof}
    (Übung)
    \end{proof}
\end{lemma}

\begin{satz} % Satz 4
    Sei $\mG \subseteq\mP\of{X}$. Dann existiert das kleinste Dynkinsystem $\delta\of{\mG}$, welches $\mG$ enthält. Wir nennen $\delta\of{\mG}$ das von $\mG$ erzeugte Dynkinsystem.
    \begin{proof}
        $\mP\of{X}$ ist ein Dynkinsystem. Wir definieren also
        \begin{align*}
            I &= \set{\mD\subseteq\mP\of{X}: \mD\text{ ist ein Dynkinsystem und }\mG\subseteq\mD} \neq \emptyset
        \end{align*}
        Anschließend setzen wir analog zum Schnitt über $\sigma$-Algebren
        \begin{align*}
            \delta\of{\mG} &\coloneqq \bigcap_{\mD\in I} \mD\qedhere
        \end{align*}
    \end{proof}
\end{satz}

\begin{definition}
    Sei $\mD\subseteq\mP\of{X}$. Wir nennen $\mD$ $\cap$-stabil, falls $A, B \in\mD \impl \pair{A \cap B} \in\mD$. Analog dazu nennen wir $\mD$ $\cup$-stabil, falls $A, B \in\mD \impl \pair{A \cup B} \in\mD$.
\end{definition}

Frage: Wann ist ein Dynkinsystem eine $\sigma$-Algebra?

\begin{lemma}
    \label{lemma:dynkin-sigma-equiv}
    Sei $\mD$ ein Dynkinsystem. Dann gilt $\mD$ ist genau dann eine $\sigma$-Algebra, wenn $A, B \in \mD \impl \pair{A \cap B} \in\mD$.

    \begin{proof}
        \theoremescape
        \anf{$\impl$} Sei $\mD$ eine $\sigma$-Algebra. Dann ist $\mD$ ein Dynkinsystem. Seien $A, B\in\mD$. Dann folgt $\comp{A}, \comp{B} \in \mD \impl A \cap B = \comp{\pair{\comp{A} \cup \comp{B}}} \in \mD$.\\[0.5\baselineskip]
        \anf{$\Leftarrow$} Zu zeigen ist Eigenschaft ($\Sigma_3$). Sei $(D_n)_n \subseteq\mD$ eine Mengenfolge. Wir definieren $D_0' \coloneqq \emptyset$ und $D_n' \coloneqq D_1 \cup D_2 \cup \dots \cup D_n$. Dann ist $(D'_n)_n$ eine aufsteigende Folge und es gilt
        \begin{align*}
            \bigcup_{n\in\N} D_n = \bigcup_{n\in\N} D_n' &= \bigsqcup_{n\in\N} \pair{D_n' \setminus D_{n-1}'}
        \end{align*}
        Außerdem ist
        \begin{align*}
            \bigsqcup_{n\in\N} \pair{D_n' \setminus D_{n-1}'} \in &\mD\text{ falls } \pair{D_n' \setminus D_{n-1}'} \in \mD~\forall n\in\N
            \intertext{Und es gilt $D_n' \setminus D_{n-1}' = \pair{D_n' \cap \comp{\pair{D_{n-1}'}}}\in \mD$, falls $D_n' \in \mD~\forall n\in\N_0$. Wir haben also unsere Behauptung gezeigt, wenn wir gezeigt haben, dass $\mD$ $\cup$-stabil ist. Es gilt aber}
            A \cup B &= \comp{\pair{\comp{A} \cap \comp{B}}} \in \mD
        \end{align*}
        Damit ist ($\Sigma_3$) gezeigt.
    \end{proof}
\end{lemma}

\begin{satz} % Satz 6
    \label{satz:dynkin-cap-stabil}
    Sei $X$ eine beliebige Menge und $\mG\subseteq\mP\of{X}$. Dann folgt aus $\mG$ ist $\cap$-stabil, dass $\delta\of{\mG}$ $\cap$-stabil ist.
    \begin{proof}
        Wir nehmen ein beliebiges $D \in \delta\of{\mG}$ und definieren
        \begin{align*}
            \mD_D \coloneqq \set{Q\in\mP\of{X}: Q \cap D \in \delta\of{\mG}}
        \end{align*}
        Behauptung: $\mD_D$ ist ein Dynkinsystem
        \begin{enumerate}[label=($\text{D}_\arabic*$)]
            \item Da $X \cap D = D \in\delta\of{\mG}$ folgt $X\in \mD_D$.
            \item Sei $Q\in\mD_D$. Dann ist auch $\comp{Q}\in\mD_D$, denn $\comp{Q} \cap D = \pair{\comp{Q} \cup \comp{D}}\cap D = \comp{\pair{Q \cap D}} \cap D = D \setminus\pair{Q\cap D} \in\delta\of{\mG}$.
            \item (Siehe handschriftliches Skript)
        \end{enumerate}
        Nun können wir folgendermaßen argumentieren: Da $\mG$ $\cap$-stabil ist, gilt
        \begin{align*}
            \forall G, D\in \mG\colon &G \cap D\in \mG \subseteq \delta\of{\mG}\\
            \equivalent \forall D\in\mG\colon &\mG \subseteq\mD_D\\
            \impl \forall D\in\mG\colon &\delta\of{\mG}\subseteq \delta\of{\mD_D} \annot{=}{(Beh.)} \mD_D\\
            \equivalent \forall D\in\mG\fa G\in\delta\of{\mG}\colon &G \cap D \in\delta\of{\mG}
            \intertext{Aus Symmetriegründen gilt dann}
            \forall G\in\delta\of{\mG}\fa D\in \mG\colon &D \cap G = G\cap D \in \delta\of{\mG}\\
            \equivalent \forall G\in \delta\of{\mG}\colon &\mG \subseteq \mD_{G}\\
            \impl \delta\of{\mG} \subseteq \delta\of{\mD_G} &= \mD_G\quad\forall G\in\delta\of{\mG}\\
            \equivalent \forall D, G\in\delta\of{\mG}\colon &D \cap G\in \delta\of{\mG}
        \end{align*}
        Das heißt $\delta\of{\mG}$ ist $\sigma$-stabil.\qedhere
    \end{proof}
\end{satz}

\begin{korollar}
    \label{korollar:dynkin-sigma}
    Sei $X$ eine beliebige Menge und $\mG \subseteq\mathcal{P}\of{X}$. Wenn $\mG$ $\cap$-stabil ist, dann ist $\delta\of{\mG}$ eine $\sigma$-Algebra und es gilt $\sigma\of{\mG} = \delta\of{\mG}$.

    \begin{proof}
        Nach Satz~\ref{satz:dynkin-cap-stabil} ist $\delta\of{\mG}$ $\cap$-stabil und damit nach Lemma~\ref{lemma:dynkin-sigma-equiv} eine $\sigma$-Algebra. Damit gilt dann $\sigma\of{\mG} \subseteq \delta\of{\mG}$, da $\sigma\of{\mG}$ die kleinste $\sigma$-Algebra ist, die $\mG$ enthält. Außerdem ist $\delta\of{\mG}\subseteq \delta\of{\sigma\of{\mG}} = \sigma\of{\mG}$.
    \end{proof}
\end{korollar}

\newpage