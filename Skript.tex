\documentclass[11pt, twoside, a4paper]{article}

% Setup
\usepackage[margin=2.4cm, top=3.5cm]{geometry}
\usepackage[utf8]{inputenc}
\usepackage[ngerman]{babel}

% Package imports
\usepackage{amsfonts}
\usepackage{amsmath}
\usepackage{amssymb}
\usepackage{amsthm}
\usepackage{mathtools}
\usepackage{setspace}
\usepackage{float}
\usepackage{enumitem}
\usepackage{hyperref}
\usepackage[pagestyles]{titlesec}
\usepackage{fancyhdr}
\usepackage{colonequals}
\usepackage{caption}
\usepackage{tikz}
\usepackage{marginnote}
\usepackage{etoolbox}
\usepackage{mdframed}
\usepackage{aligned-overset}
\usepackage{esint}
\usepackage{scalerel}

% Font-Encoding
\usepackage[T1]{fontenc}
\usepackage{lmodern}

% TikZ packages
\usetikzlibrary{patterns}

% Theorems
\newtheoremstyle{plain}{}{}{}{}{\bfseries}{.}{ }{}
\theoremstyle{plain}
\newtheorem{blockelement}{Blockelement}[subsection]
\newtheorem{bemerkung}[blockelement]{Bemerkung}
\newtheorem{definition}[blockelement]{Definition}
\newtheorem{lemma}[blockelement]{Lemma}
\newtheorem{satz}[blockelement]{Satz}
\newtheorem{notation}[blockelement]{Notation}
\newtheorem{korollar}[blockelement]{Korollar}
\newtheorem{uebung}[blockelement]{Übung}
\newtheorem{beispiel}[blockelement]{Beispiel}
\newtheorem{folgerung}[blockelement]{Folgerung}
\newtheorem{axiom}[blockelement]{Axiom}
\newtheorem{beobachtung}[blockelement]{Beobachtung}
\newtheorem{konzept}[blockelement]{Konzept}
\newtheorem{konstruktion}[blockelement]{Konstruktion}
\newtheorem{visualisierung}[blockelement]{Visualisierung}
\newtheorem{anwendung}[blockelement]{Anwendung}
\newtheorem{skizze}[blockelement]{Skizze}
\newtheorem{genv}[blockelement]{}

% Numbering (equations and conditions)
\numberwithin{equation}{subsection}
\newcommand{\numbereq}[1]{\addtocounter{equation}{1}\tag{\theequation}\label{#1}}
\newcounter{condition}
\renewcommand{\thecondition}{V\arabic{condition}}
\newcommand{\condition}[1]{\hypertarget{#1}{\refstepcounter{condition}(\label{#1}\thecondition})}

% Marginnotes left
\makeatletter
\patchcmd{\@mn@@@marginnote}{\begingroup}{\begingroup\@twosidefalse}{}{\fail}
\reversemarginpar
\makeatother

\DeclareMathAlphabet{\altmathbb}{U}{BOONDOX-ds}{m}{n}

% Long equations
\allowdisplaybreaks

% \left \right
\newcommand{\set}[1]{\left\{#1\right\}}
\newcommand{\pair}[1]{\left(#1\right)}
\newcommand{\of}[1]{\mathopen{}\mathclose{}\bgroup\left(#1\aftergroup\egroup\right)}
\newcommand{\abs}[1]{\left\lvert#1\right\rvert}
\newcommand{\norm}[1]{\left\lVert#1\right\rVert}
\newcommand{\linterv}[1]{\left[#1\right)}
\newcommand{\rinterv}[1]{\left(#1\right]}
\newcommand{\interv}[1]{\left[#1\right]}
\newcommand{\scalprod}[1]{\left<#1\right>}

% Shorten commands
\newcommand{\equivalent}[0]{\Leftrightarrow{}}
\newcommand{\impl}[0]{\Rightarrow{}}
\newcommand{\definedasequiv}[0]{\ratio\Leftrightarrow{}}
\renewcommand{\emptyset}{\varnothing}
\newcommand{\dif}{\mathop{}\!\mathrm{d}}
\newcommand{\dsty}{\displaystyle}
\newcommand{\charfunc}{\altmathbb{1}}

\newcommand{\toinf}{\to\infty}
\newcommand{\fa}{\;\forall}
\newcommand{\ex}{\;\exists}
\newcommand{\conj}[1]{\overline{#1}}
\newcommand{\comp}[1]{{#1}^{\mathrm{C}}}

\newcommand{\annot}[3][]{\overset{\text{#3}}#1{#2}}
\newcommand{\anf}[1]{\glqq{}#1\grqq}
\newcommand{\OBDA}{o.B.d.A. }
\newcommand{\theoremescape}{\leavevmode}
\newcommand{\aligntoright}[2]{\hfill#1\hspace{#2\textwidth}~}
\newcommand{\horizontalline}[0]{\par\noindent\rule{0.05\textwidth}{0.1pt}\\}
\newcommand{\rgbcolor}[3]{rgb,255:red,#1;green,#2;blue,#3}
\newcommand{\fixedspace}[2]{\makebox[#1][l]{#2}}

\let\Re\relax
\let\Im\relax

% MathOperators
\DeclareMathOperator{\grad}{Grad}
\DeclareMathOperator{\bild}{Bild}
\DeclareMathOperator{\Re}{Re}
\DeclareMathOperator{\Im}{Im}
\DeclareMathOperator{\arcsinh}{arcsinh}
\DeclareMathOperator{\arccosh}{arccosh}
\DeclareMathOperator{\diam}{diam}
\DeclareMathOperator{\fehler}{Fehler}
\DeclareMathOperator{\D}{D\!}
\DeclareMathOperator{\Id}{Id}
\DeclareMathOperator{\op}{op}
\DeclareMathOperator{\rank}{rk}
\DeclareMathOperator{\spann}{Spann}
\DeclareMathOperator{\flaeche}{Fläche}
\DeclareMathOperator{\bew}{Bew}

% Mengenbezeichner
\newcommand{\R}{\mathbb{R}}
\newcommand{\N}{\mathbb{N}}
\newcommand{\C}{\mathbb{C}}
\newcommand{\Z}{\mathbb{Z}}
\newcommand{\Q}{\mathbb{Q}}
\newcommand{\K}{\mathbb{K}}

\newcommand{\mA}{\mathcal{A}}
\newcommand{\mB}{\mathcal{B}}
\newcommand{\mC}{\mathcal{C}}
\newcommand{\mD}{\mathcal{D}}
\newcommand{\mE}{\mathcal{E}}
\newcommand{\mF}{\mathcal{F}}
\newcommand{\mG}{\mathcal{G}}
\newcommand{\mH}{\mathcal{H}}
\newcommand{\mJ}{\mathcal{J}}
\newcommand{\mK}{\mathcal{K}}
\newcommand{\mL}{\mathcal{L}}
\newcommand{\mM}{\mathcal{M}}
\newcommand{\mO}{\mathcal{O}}
\newcommand{\mP}{\mathcal{P}}
\newcommand{\mQ}{\mathcal{Q}}
\newcommand{\mR}{\mathcal{R}}
\newcommand{\mS}{\mathcal{S}}
\newcommand{\mPC}{\mathcal{PC}}

% Spezielle Symbole
\NewDocumentCommand{\Tau}{e{^_}}{
    \scalerel*{\tau}{X}
    \IfValueT{#1}{^{#1}}
    \IfValueT{#2}{_{\!\!#2}}
}

% Spezielle Commands
\newcommand\imaginarysubsection[1]{
    \refstepcounter{subsection}
    \subsectionmark{#1}
}

% Unfassbar hässlich, aber effektiv für temporäre schnelle Lösungen
\def\:={\coloneqq}
\def\->{\to}
\def\=>{\impl}
\def\<={\leq}
\def\>={\geq}

% Envs
\newenvironment{induktionsanfang}{
    \rule{0pt}{3ex}\noindent
    \begin{minipage}[t]{0.11\textwidth}
    {I-Anfang}
    \end{minipage}
    \hfill
    \begin{minipage}[t]{0.89\textwidth}
    }
    {
    \end{minipage}
}
\newenvironment{induktionsvoraussetzung}{
    \rule{0pt}{3ex}\noindent
    \begin{minipage}[t]{0.11\textwidth}
    {I-Vor.}
    \end{minipage}
    \hfill
    \begin{minipage}[t]{0.89\textwidth}
    }
    {
    \end{minipage}
}
\newenvironment{induktionsschritt}{
    \rule{0pt}{3ex}\noindent
    \begin{minipage}[t]{0.11\textwidth}
    {I-Schritt}
    \end{minipage}
    \hfill
    \begin{minipage}[t]{0.89\textwidth}
    }
    {
    \end{minipage}
}

% Section style
\titleformat*{\section}{\LARGE\bfseries}
\titleformat*{\subsection}{\large\bfseries}

% Page styles
\newpagestyle{pagenumberonly}{
    \sethead{}{}{}
    \setfoot[][][\thepage]{\thepage}{}{}
}
\newpagestyle{headfootdefault}{
    \sethead[][][\thesubsection~\textit{\subsectiontitle}]{\thesection~\textit{\sectiontitle}}{}{}
    \setfoot[][][\thepage]{\thepage}{}{}
}
\pagestyle{headfootdefault}

\begin{document}
    \title{\vspace{3cm} Skript zur Vorlesung\\Analysis III\\bei Prof. Dr. Dirk Hundertmark}
    \author{Karlsruher Institut für Technologie}
    \date{Wintersemester 2024/25}
    \maketitle
    \begin{center}
        Dieses Skript ist inoffiziell. Es besteht kein\\Anspruch auf Vollständigkeit oder Korrektheit.
    \end{center}
    \thispagestyle{empty}
    \newpage

    \tableofcontents
    ~\\
    Alle mit [*] markierten Kapitel sind noch nicht Korrektur gelesen und bedürfen eventuell noch Änderungen.

    \newpage

    \include{Kapitel/Einleitung}
    \include{Kapitel/Sigma_Algebren}
    \section{Dynkinsysteme}
\imaginarysubsection{Dynkinsysteme}
\thispagestyle{pagenumberonly}
\begin{definition}[Dynkinsystem]
    Ein Mengensystem $\mD \subseteq\mP\of{X}$ heißt Dynkinsystem, falls
    \begin{enumerate}[label=($\text{D}_{\arabic*}$)]
        \item $X\in\mD$
        \item $D\in\mD \impl \comp{D}\in\mD$
        \item Für eine paarweise disjunkte Mengenfolge $(D_n)_n \subseteq \mD \impl \dsty\bigsqcup_{n\in\N} D_n \in\mD$
    \end{enumerate}
\end{definition}

\begin{beispiel}
    \theoremescape
    \begin{enumerate}
        \item Jede $\sigma$-Algebra ist ein Dynkinsystem.
        \item Sei $X$ eine $2n$-elementige Menge. Dann ist $\mD \coloneqq \set{A \subseteq X: A\text{ hat eine gerade Anzahl an Elementen}}$ ein Dynkinsystem, aber keine $\sigma$-Algebra.
    \end{enumerate}
\end{beispiel}

\begin{lemma}
    Sei $I$ eine beliebige Indexmenge und $(\mD_j)_{j\in I}$ eine Familie von Dynkinsystemen in $X$, dann ist $\displaystyle\bigcap_{j\in I} \mD_j$ wieder ein Dynkinsystem.
    \begin{proof}
    (Übung)
    \end{proof}
\end{lemma}

\begin{satz} % Satz 4
    Sei $\mG \subseteq\mP\of{X}$. Dann existiert das kleinste Dynkinsystem $\delta\of{\mG}$, welches $\mG$ enthält. Wir nennen $\delta\of{\mG}$ das von $\mG$ erzeugte Dynkinsystem.
    \begin{proof}
        $\mP\of{X}$ ist ein Dynkinsystem. Wir definieren also
        \begin{align*}
            I &= \set{\mD\subseteq\mP\of{X}: \mD\text{ ist ein Dynkinsystem und }\mG\subseteq\mD} \neq \emptyset
        \end{align*}
        Anschließend setzen wir analog zum Schnitt über $\sigma$-Algebren
        \begin{align*}
            \delta\of{\mG} &\coloneqq \bigcap_{\mD\in I} \mD\qedhere
        \end{align*}
    \end{proof}
\end{satz}

\begin{definition}
    Sei $\mD\subseteq\mP\of{X}$. Wir nennen $\mD$ $\cap$-stabil, falls $A, B \in\mD \impl \pair{A \cap B} \in\mD$. Analog dazu nennen wir $\mD$ $\cup$-stabil, falls $A, B \in\mD \impl \pair{A \cup B} \in\mD$.
\end{definition}

Frage: Wann ist ein Dynkinsystem eine $\sigma$-Algebra?

\begin{lemma}
    \label{lemma:dynkin-sigma-equiv}
    Sei $\mD$ ein Dynkinsystem. Dann gilt $\mD$ ist genau dann eine $\sigma$-Algebra, wenn $A, B \in \mD \impl \pair{A \cap B} \in\mD$.

    \begin{proof}
        \theoremescape
        \anf{$\impl$} Sei $\mD$ eine $\sigma$-Algebra. Dann ist $\mD$ ein Dynkinsystem. Seien $A, B\in\mD$. Dann folgt $\comp{A}, \comp{B} \in \mD \impl A \cap B = \comp{\pair{\comp{A} \cup \comp{B}}} \in \mD$.\\[0.5\baselineskip]
        \anf{$\Leftarrow$} Zu zeigen ist Eigenschaft ($\Sigma_3$). Sei $(D_n)_n \subseteq\mD$ eine Mengenfolge. Wir definieren $D_0' \coloneqq \emptyset$ und $D_n' \coloneqq D_1 \cup D_2 \cup \dots \cup D_n$. Dann ist $(D'_n)_n$ eine aufsteigende Folge und es gilt
        \begin{align*}
            \bigcup_{n\in\N} D_n = \bigcup_{n\in\N} D_n' &= \bigsqcup_{n\in\N} \pair{D_n' \setminus D_{n-1}'}
        \end{align*}
        Außerdem ist
        \begin{align*}
            \bigsqcup_{n\in\N} \pair{D_n' \setminus D_{n-1}'} \in &\mD\text{ falls } \pair{D_n' \setminus D_{n-1}'} \in \mD~\forall n\in\N
            \intertext{Und es gilt $D_n' \setminus D_{n-1}' = \pair{D_n' \cap \comp{\pair{D_{n-1}'}}}\in \mD$, falls $D_n' \in \mD~\forall n\in\N_0$. Wir haben also unsere Behauptung gezeigt, wenn wir gezeigt haben, dass $\mD$ $\cup$-stabil ist. Es gilt aber}
            A \cup B &= \comp{\pair{\comp{A} \cap \comp{B}}} \in \mD
        \end{align*}
        Damit ist ($\Sigma_3$) gezeigt.
    \end{proof}
\end{lemma}

\begin{satz} % Satz 6
    \label{satz:dynkin-cap-stabil}
    Sei $X$ eine beliebige Menge und $\mG\subseteq\mP\of{X}$. Dann folgt aus $\mG$ ist $\cap$-stabil, dass $\delta\of{\mG}$ $\cap$-stabil ist.
    \begin{proof}
        Wir nehmen ein beliebiges $D \in \delta\of{\mG}$ und definieren
        \begin{align*}
            \mD_D \coloneqq \set{Q\in\mP\of{X}: Q \cap D \in \delta\of{\mG}}
        \end{align*}
        Behauptung: $\mD_D$ ist ein Dynkinsystem
        \begin{enumerate}[label=($\text{D}_\arabic*$)]
            \item Da $X \cap D = D \in\delta\of{\mG}$ folgt $X\in \mD_D$.
            \item Sei $Q\in\mD_D$. Dann ist auch $\comp{Q}\in\mD_D$, denn $\comp{Q} \cap D = \pair{\comp{Q} \cup \comp{D}}\cap D = \comp{\pair{Q \cap D}} \cap D = D \setminus\pair{Q\cap D} \in\delta\of{\mG}$.
            \item (Siehe handschriftliches Skript)
        \end{enumerate}
        Nun können wir folgendermaßen argumentieren: Da $\mG$ $\cap$-stabil ist, gilt
        \begin{align*}
            \forall G, D\in \mG\colon &G \cap D\in \mG \subseteq \delta\of{\mG}\\
            \equivalent \forall D\in\mG\colon &\mG \subseteq\mD_D\\
            \impl \forall D\in\mG\colon &\delta\of{\mG}\subseteq \delta\of{\mD_D} \annot{=}{(Beh.)} \mD_D\\
            \equivalent \forall D\in\mG\fa G\in\delta\of{\mG}\colon &G \cap D \in\delta\of{\mG}
            \intertext{Aus Symmetriegründen gilt dann}
            \forall G\in\delta\of{\mG}\fa D\in \mG\colon &D \cap G = G\cap D \in \delta\of{\mG}\\
            \equivalent \forall G\in \delta\of{\mG}\colon &\mG \subseteq \mD_{G}\\
            \impl \delta\of{\mG} \subseteq \delta\of{\mD_G} &= \mD_G\quad\forall G\in\delta\of{\mG}\\
            \equivalent \forall D, G\in\delta\of{\mG}\colon &D \cap G\in \delta\of{\mG}
        \end{align*}
        Das heißt $\delta\of{\mG}$ ist $\sigma$-stabil.\qedhere
    \end{proof}
\end{satz}

\begin{korollar}
    \label{korollar:dynkin-sigma}
    Sei $X$ eine beliebige Menge und $\mG \subseteq\mathcal{P}\of{X}$. Wenn $\mG$ $\cap$-stabil ist, dann ist $\delta\of{\mG}$ eine $\sigma$-Algebra und es gilt $\sigma\of{\mG} = \delta\of{\mG}$.

    \begin{proof}
        Nach Satz~\ref{satz:dynkin-cap-stabil} ist $\delta\of{\mG}$ $\cap$-stabil und damit nach Lemma~\ref{lemma:dynkin-sigma-equiv} eine $\sigma$-Algebra. Damit gilt dann $\sigma\of{\mG} \subseteq \delta\of{\mG}$, da $\sigma\of{\mG}$ die kleinste $\sigma$-Algebra ist, die $\mG$ enthält. Außerdem ist $\delta\of{\mG}\subseteq \delta\of{\sigma\of{\mG}} = \sigma\of{\mG}$.
    \end{proof}
\end{korollar}

\newpage


    \section{[*] Eindeutigkeit von Maßen und erste Eigenschaften des Lebesgue-Maßes}
    \imaginarysubsection{Eindeutigkeit von Maßen}

    \begin{satz}[Eindeutigkeitssatz]
        \marginnote{[04. Nov]}
        \label{satz:eindeutigkeitssatz}
        Sei $(X, \mA)$ ein beliebiger Messraum und $\mA = \sigma\of{\mE}$ für $\mE \subseteq\mP\of{X}$. Ferner seien $\mu, \nu$ Maße auf $\mA$ mit
        \begin{enumerate}[label=(\alph*)]
            \item $\mE$ ist $\cap$-stabil
            \item Es gibt Mengen $G_n \in \mE$ mit $G_n \nearrow X$ ($X = \bigcup_{n\in\N} G_n$) mit $\mu\of{G_n}, \nu\of{G_n} < \infty~\forall n\in\N$
        \end{enumerate}
        Dann gilt: Aus $\mu\of{A} = \nu\of{A}~\forall A\in\mE$ folgt $\mu = \nu$ auf $\mA$. Das heißt unter den obigen Voraussetzungen wird ein Maß eindeutig durch seine Werte auf dem Erzeuger definiert.

        \begin{proof}
            Da $\mE$ $\cap$-stabil ist, folgt nach Korollar~\ref{korollar:dynkin-sigma}, dass $\delta\of{\mE} = \sigma\of{\mE} = \mA$. Wir halten $n\in\N$ fest und betrachten
            \begin{align*}
                \mD_n &\coloneqq \set{A\in \mA: \mu\of{G_n \cap A} = \nu\of{G_n \cap A}}
            \end{align*}
            $\mD_n$ ist ein Dynkinsystem:
            \begin{enumerate}[label=($\text{D}_{\arabic*}$)]
                \item Folgt direkt.
                \item Sei $A\in\mD_n$. Dann ist
                \begin{align*}
                    \mu\of{G_n \cap \comp{A}} &= \mu\of{G_n \setminus A} = \mu\of{G_n \setminus\pair{A \cap G_n}}\\
                    &= \mu\of{G_n} - \mu\of{A \cap G_n}\\
                    &= \nu\of{G_n} - \nu\of{A \cap G_n}\\
                    &= \nu\of{G_n \cap \comp{A}}\\
                    \impl \comp{A} &\in \mD_n
                \end{align*}
                \item Sei $(A_m)_m \subseteq \mD_n$ eine Folge paarweise disjunkter Mengen. Dann gilt
                \begin{align*}
                    \mu\of{\pair{\bigsqcup_{m\in\N} A_m} \cap G_n} &= \mu\of{\bigsqcup_{m\in\N} \pair{A_m \cap G_n}} = \sum_{m\in\N}^{} \mu\of{A_m \cap G_n}\\
                    &= \sum_{m\in\N}^{} \nu\of{A_m \cap G_n} = \nu\of{\pair{\bigsqcup_{m\in\N} A_m} \cap G_n}\\
                    \impl \bigsqcup_{m\in\N} A_m &\in \D_n
                \end{align*}
            \end{enumerate}
            Nach Konstruktion von $\mD_n$ gilt $\mD_n \subseteq\mA$. Andererseits ist $\mE\subseteq\mD_n$. Sei $A \in\mE$ und $A \cap G_n \in \mE$, da $\mE$ $\cap$-stabil. Nach Voraussetzung gilt $\nu\of{A \cap G_n} = \mu\of{A \cap G_n}$, also folgt $A \in\mD_n$.\\
            Da $\mE \subseteq\mD_n \impl \sigma\of{\mE} = \delta\of{\mE} \subseteq \delta\of{\mE} = \mD_n$. Damit gilt $\sigma\of{\mE} \subseteq  \mD_n$. Das heißt $\forall A\in\sigma\of{\mE}$ folgt $\mu\of{A \cap G_n} = \nu\of{A \cap G_n}$.\\
            Für $A\in \sigma\of{\mE}$ definieren wir eine aufsteigende Folge $A_n \coloneqq A \cap G_n \nearrow A$. Da $\mu, \nu$ Maße sind, sind sie von unten stetig. Das heißt
            \begin{align*}
                \mu\of{A} &= \lim_{n\toinf} \mu\of{A_n} = \lim_{n\toinf} \mu\of{A \cap G_n}\\
                &= \lim_{n\toinf} \nu\of{A \cap G_n} = \nu\of{A}\qedhere
            \end{align*}
        \end{proof}
    \end{satz}

    \begin{bemerkung}[Ausschöpfende Folgen]
        Wir nennen $(G_n)_n$ im Sinne von Satz~\ref{satz:eindeutigkeitssatz} eine ausschöpfende Folge. Wir nennen ein Maß $\mu$ auf $\mA$ $\sigma$-endlich, wenn es eine Folge $(G_n)_n \subseteq \mA$ gibt mit $G_n \nearrow X$ und $\mu\of{G_n} < \infty~\forall n\in\N$.
    \end{bemerkung}

    \begin{satz}[Eigenschaften der Borelmengen]
        In Definition~\ref{definition:sigma-borel} hatten wir $\mB\of{\R^d} \coloneqq \sigma\of{\mO_d}$, wobei $\mO$ das System offener Teilmengen im $\R^d$ war. Wir definieren nun
        \begin{enumerate}[label=-]
            \item $\mA_d$: System der abgeschlossenen Teilmengen im $\R^d$
            \item $\mK_d$: System der kompakten Teilmengen im $\R^d$
        \end{enumerate}
        Dann gilt $\sigma\of{\mK_d} = \sigma\of{\mA_d} = \sigma\of{\mO_d} = \mB\of{\R^d}$.
        \begin{proof}
            \textsc{Schritt 1}: $\sigma\of{\mA_d} = \sigma\of{\mO_d}$ ist klar, da $\sigma$-Algebren stabil unter Komplementbildung sind.\\
            \textsc{Schritt 2}: Es gilt $\mK_d \subseteq \mA_d \impl \sigma\of{\mK_d} \subseteq \sigma\of{\mA_d}$.\\
            \textsc{Schritt 3}: Für $n\in\N$ definieren wir $K_n \coloneqq \set{\abs{x} < n}$. Sei $A \in \mA_d$, dann ist $A\cap K_n$ kompakt und
            \begin{align*}
                \bigcup_{n\in\N} K_n &= \R^d\\
                A &= \bigcup_{n\in\N} \pair{A \cap K_n} \in \sigma\of{\mK_d}\\
                \impl \mA &\subseteq \sigma\of{\mK_d}\\
                \impl \sigma\of{\mA_d} &\subseteq \sigma\of{\sigma\of{\mK_d}} = \sigma\of{\mK_d}\\
                \impl \sigma\of{\mK_d} &= \sigma\of{\mA_d} = \sigma\of{\mO_d}\qedhere
            \end{align*}
        \end{proof}
    \end{satz}

    Im Folgenden nehmen wir an, dass das Lebesgue-Maß $\lambda^d$ auf $\mB\of{\R^d}$ existiert. Wir werden das später noch beweisen, aber entwickeln das Maß nun nach unserem geometrischen Verständnis unter der Annahme, dass es existiert (das tut es) und untersuchen erste Eigenschaften:

    \begin{beobachtung}
        Wir betrachten den Fall $d=1$ und ein halboffenes Intervall $I\coloneqq \linterv{a,b}$. Dann muss gelten $\lambda^1\of{I} = b-a$. Wir betrachten allgemeine $d$ mit $a,b\in\R^d$ wobei $a\leq b$ (das heißt $a_j \leq b_j$). Dann sei
        \begin{align*}
            \linterv{a,b} &\coloneqq \set{x\in\R^d: a_j \leq x_j \leq b_j~\forall j\in\set{1,\ldots, d}}\\
            \intertext{und wir definieren nach unserem geometrischen Verständnis}
            \lambda^d\of{\linterv{a,b}} &\coloneqq \prod_{j=1}^{d} \pair{b_j - a_j}
        \end{align*}
    \end{beobachtung}

    \begin{definition}
        Es sei $J^d \coloneqq \set{\linterv{a,b}: a,b\in\R^d,~a\leq b}$ das Mengensystem der halboffenen Intervalle im $\R^d$.
    \end{definition}

    \begin{bemerkung}[Translationsinvarianz des Lebesgue-Maß]
        Es sei $c\in\R^d$ und wir definieren eine Translation $T_c\of{x} \coloneqq x+c$ mit inverser Funktion $T_c^{-1}$. Dann gilt für ein halboffenes Intervall $I \coloneqq \linterv{a,b}$
        \begin{align*}
            \lambda^d\of{T_c^{-1}\of{I}} &= \lambda^d\of{\linterv{a-c, b-c}}\\
            &= \prod_{j=1}^{d} \pair{b_j - c_j - \pair{a_j - c_j}}\\
            &= \prod_{j=1}^{d} \pair{b_j - a_j} = \lambda^d\of{I}
        \end{align*}
        Das heißt auf $J^d$ ist $\lambda^d$ invariant unter Translation.
    \end{bemerkung}

    \begin{lemma}
        Sei $B \in\mB^d$ eine Borelmenge und $c\in\R^d$. Dann ist $B + c \coloneqq\set{b+c: b\in B} \in\mB^d$.
        \begin{proof}
            Sei $c\in\R^d$ fest. \textsc{Schritt 1}: Wir wenden das \anf{Wünsch-dir-was}-Vorgehen an und definieren
            \begin{align*}
                \mA \coloneqq \set{A \in \mB^d: A + c \in \mB^d}
            \end{align*}
            Dann ist $\mA$ eine $\sigma$-Algebra (Übung).\\
            \textsc{Schritt 2}: $\mO_d$ ist translationsinvariant. Das heißt $\mO_d \subseteq \mA \impl \mB^d = \sigma\of{\mO_d} \subseteq \sigma\of{\mA} = \mA$. Das heißt $\mB^d \subseteq \mA$. Damit sind die Borelmengen translationsinvariant.
        \end{proof}
    \end{lemma}

    \marginnote{[8. Nov]} (fehlt)
    \newpage

    \begin{satz}
        \marginnote{[18. Nov]}
        Es gilt
        \begin{align*}
            T\of{\lambda^d} &= \lambda^d\quad\forall T\in\bew\of{\R^d}
        \end{align*}
        \begin{proof}
            \textsc{Schritt 1}: Sei $T\in\bew\of{\R^d}: T\of{0} = 0$. Das heißt $T$ ist linear. Dann definieren wir eine Translation
            \begin{align*}
                T_a\of{x} &\coloneqq x + a\\
                \impl \pair{T_a \circ T}\of{x} &= T\of{x} + a\\
                &= T\of{x} + T\of{b}\tag{$b\coloneqq T^{-1}\of{a}$}\\
                &= T\of{x+b} = \pair{T\circ T_b}\of{x}\\
                \impl T_a \circ T &= T \circ T_b
                \intertext{Wir definieren}
                \mu &\coloneqq T\of{\lambda^d} &= \lambda^d \circ T^{-1}\\
                T_a\of{\mu} &= T_a\of{T\of{\lambda^d}} = T_a \circ T\of{\lambda^d}\\
                &= T \circ T_b\of{\lambda^d} = T\of{T_b\of{\lambda_d}}\\
                &= T\of{\lambda^d} = \mu
            \end{align*}
            Damit ist $\mu$ invariant unter Translation. Das heißt nach dem Eindeutigkeitssatz, dass $\mu = \alpha\lambda^d$.\\
            Frage: Warum ist $\alpha = 1$?\\
            Wir betrachten die abgeschlossene Einheitskugel $B\coloneqq\set{x\in\R^d: \abs{x}\leq 1}$. Dann folgt
            \begin{align*}
                T^{-1}\of{B} &= B\\
                \impl \lambda^d\of{B} &= \lambda^{d}\of{T^{-1}\of{B}} = \mu\of{B} = \alpha \lambda^d\of{B}\\
                \impl \alpha &= 1 \text{ falls } 0 < \lambda^d\of{B} <\infty\\
                \impl T\of{\lambda^d} &= \lambda^d\quad\forall T\in\bew\of{\R^d}: T\of{0} = 0
            \end{align*}
            \textsc{Schritt 2}: Sei $T\in\bew\of{\R^d}$ beliebig und wir setzen $c\coloneqq T\of{0}$ und $S \coloneqq T_{-c} \circ T \in\bew\of{\R^d}$. Dann gilt
            \begin{align*}
                S\of{0} &= T_{-c} \of{T\of{0}} = T_{-c}\of{c} = 0\\
                \impl S\of{\lambda^d} &= \lambda^d\text{ nach \textsc{Schritt 1}}
                \intertext{Wir wollen das aber noch für allgemeine Bewegungen zeigen. Es gilt}
                T &= T_c \circ S\\
                \impl T\of{\lambda^d} &= T_c\of{S\of{\lambda_d}} = T_c\of{\lambda^d} = \lambda^d
            \end{align*}
            nach \textsc{Schritt 1}.\qedhere
        \end{proof}
    \end{satz}

    \begin{beispiel}[Lebesgue-Maß von einem Punkt]
        Sei $x\in\R^d$. Was ist dann $\lambda^d\of{\set{x}}$?. Wir definieren für $\varepsilon > 0$
        \begin{align*}
            J &\coloneqq \linterv{x, x+\varepsilon}\\
            \impl x&\in J\\
            \impl \lambda^d\of{\set{x}} &\leq \lambda^d\of{J} = \varepsilon^d \to 0\\
            \impl \lambda^d\of{\set{x}} &= 0\quad\forall x\in\R^d
        \end{align*}
        Damit ist auch das Lebesgue-Maß von einer Menge von abzählbar vielen Punkten $A \coloneqq \bigcup_{n\in\N} \set{x_n}$ null, da
        \begin{align*}
            \lambda^d\of{A} &\leq \sum_{n\in\N}^{} \lambda^d\of{\set{x_n}} = 0\\
            \impl \lambda^d\of{\Q^d} &= 0
        \end{align*}
    \end{beispiel}

    \begin{beispiel}[Lebesgue-Maß einer Hyperebene]
        Es sei $j\in\set{1, \ldots, d}$ und wir definieren eine Hyperebene $H_j\coloneqq\set{x\in\R^d: x_j = 0}$. Was ist dann $\lambda^d\of{H_j}$? Wir definieren
        \begin{align*}
            J_n \coloneqq \set{x\in\R^d: -n \leq x_k \leq n, k\neq j\land -\frac{\varepsilon}{(2n)^{d-1} 2\cdot2^n} \leq x_j \leq \frac{\varepsilon}{2\cdot 2^{n}(2n)^{d-1}}}
        \end{align*}
        Damit ist
        \begin{align*}
            H_j &\subseteq \bigcup_{n\in\N} J_n\\
            \lambda^d\of{H_j} &\leq \lambda^d\of{\bigcup_{n\in\N} J_n} \leq \sum_{n=1}^{\infty} \lambda^d\of{J_n}\\
            &= \sum_{n=1}^{\infty} (2n)^{d-1}\cdot \frac{2\varepsilon}{2(2n)^{d-1}2^n}\\
            &= \varepsilon\sum_{n=1}^{\infty} 2^{-n} = \varepsilon\quad\forall\varepsilon > 0\\
            \impl \lambda^d\of{H_j} &= 0
        \end{align*}
        Dieses Result gilt auch für allgemeine Hyperebenen, da wir eine Bewegung finden, die diese auf eine Hyperebene der Form $H_j$ abbildet.
    \end{beispiel}

    \newpage


    \section{[*] Existenz von Maßen}
    \imaginarysubsection{Existenz von Maßen}
    \thispagestyle{pagenumberonly}

    \begin{definition}[Halbring]
        Eine Familie $\mS \subseteq\mP\of{X}$ heißt Halbring, falls
        \begin{enumerate}[label=($\text{S}_{\arabic*}$)]
            \item $\emptyset\in\mS$
            \item $A, B\in\mS\impl A \cap B \in\mS$
            \item $A, B\in\mS$ existieren endlich viele disjunkte Mengen $S_1, \ldots, S_M \in \mS$ mit $A \setminus B = \bigsqcup_{j=1}^{M} S_j$
        \end{enumerate}
    \end{definition}

    \begin{satz}[Nach Carathéodory] % Satz 1
        \label{satz:caratheodory}
        Sei $\mS\subseteq\mP\of{X}$ ein Halbring und $\mu: \mS \to \interv{0,\infty}$ ein Prämaß. Dann existiert (mindestens) eine Fortsetzung von $\mu$ zu einem Maß $\mu$ auf $\sigma\of{\mS}$.\\
        Falls $\mu$ $\sigma$-endlich auf $\mS$ ist, dann ist die Fortsetzung eindeutig.
    \end{satz}

    \begin{definition}[Äußere Maße]
        \marginnote{[25. Nov]}
        Eine Abbildung $\mu^{\ast}: \mP\of{X} \to \interv{0,\infty}$ ist ein äußeres Maß, falls
        \begin{enumerate}[label=(\roman*)]
            \item $\mu^{\ast}\of{\emptyset} = 0$\hfill (Normierung)
            \item $A \subseteq B \impl \mu^{\ast}\of{A} \leq \mu^{\ast}\of{B}$\hfill (Monotonie)
            \item $\mu^{\ast}\of{\bigcup_{n\in\N} A_n} \leq \sum_{n\in\N}^{} \mu^{\ast}\of{A_n}$\hfill ($\sigma$-Subadditivität)
        \end{enumerate}
    \end{definition}

    \begin{konstruktion}
        \label{konstruktion:prem}
        Zu einem Prämaß $\mu$ auf $\mS$ gibt es ein äußeres Maß.\\
        Nehme $A\subseteq X$. Wir bepflastern $A$ mit Mengen aus $\mS$:
        \begin{align*}
            \mC\of{A} &\coloneqq \set{(S_n)_{n\in\N}: S_n \in\mS~\forall n\in N \land A \subseteq \bigcup_{n\in\N} S_n}\tag{Cover}
            \intertext{$\mC\of{A} = \emptyset$ ist dabei möglich. Gegeben ein Prämaß $\mu: \mS\to\interv{0,\infty}$ definieren wir nun}
            \mu^{\ast}\of{A} &\coloneqq \begin{cases}
                                            \inf \set{\sum_{n\in\N}^{} \mu\of{S_n}: (S_n)_n \in\mC\of{A}} &\text{ falls } \mC\of{A} \neq \emptyset\\
                                            +\infty &\text{ falls } \mC\of{A} = \emptyset
            \end{cases}
        \end{align*}
    \end{konstruktion}

    \begin{lemma}
        Das in Konstruktion~\ref{konstruktion:prem} definierte $\mu^{\ast}$ ist ein äußeres Maß.
        \begin{proof}
            \theoremescape
            \begin{enumerate}[label=(\roman*)]
                \item Wir nehmen $S_n = \emptyset$. Dann gilt $\mu^{\ast}\of{\emptyset} = 0$
                \item Sei $A\subseteq B$. Angenommen $\mC\of{B} \neq \emptyset \impl \mC\of{B} \subseteq \mC\of{A}$
                \begin{align*}
                    \mu^{\ast}\of{B} &= \inf\set{\sum_{n\in\N}^{} \mu\of{S_n}: (S_n)_n \in\mC\of{B}}\\
                    &\geq \inf\set{\sum_{n\in\N}^{} \mu\of{S_n}: (S_n)_n \in\mC\of{B}}\\
                    &= \mu^{\ast}\of{A}
                \end{align*}
                \item Es gilt $\mu^{\ast}\of{A_n} = \inf \set{\sum_{n\in\N}^{} \mu\of{S_n}: (S_n)_n \in\mC\of{A}}$. Für jedes $n\in\N$ existiert ein $(S_{n,m})_m \in\mC\of{A_n}$ mit
                \begin{align*}
                    \sum_{m\in\N}^{} \mu\of{S_n,m} &\leq \mu^{\ast}\of{A_n} + \varepsilon \Sigma^n\quad\forall\varepsilon > 0\\
                    \impl \mu^{\ast}\of{A} &\leq \sum_{n,m\in\N}^{} \mu\of{S_{n,m}}\\
                    &= \sum_{n\in\N}^{} \of{\sum_{m\in\N}^{} \mu\of{S_{n,m}}}\\
                    &\leq \sum_{n\in\N}^{} \pair{\mu^{\ast}\of{A_n} + \varepsilon\Sigma^n}\\
                    &= \sum_{m\in\N}^{} \mu^{\ast}\of{A_m} + \varepsilon\quad\forall\varepsilon > 0\qedhere
                \end{align*}
            \end{enumerate}
        \end{proof}
    \end{lemma}

    \begin{konstruktion}[Prämaß auf erzeugtem Ring]
        Sei $\mu$ ein Prämaß auf einem Halbring $\mS$. Wir wollen $\mu$ zu einem Prämaß auf dem erzeugten Mengenring forsetzen. Dazu gehen wir wie folgt vor.\\
        Wir setzen $\mS_{\cup} \coloneqq \set{S_1 \sqcup S_2 \sqcup \dots \sqcup S_n: S_j\in\mS}$ und definieren
        \begin{align*}
            \overline{\mu}\of{S_1 \sqcup \dots \sqcup S_n} &= \sum_{j=1}^{n} \mu\of{S_j}
        \end{align*}
        Wir zeigen die Wohldefiniertheit von $\mu$. Angenommen $S_1 \sqcup \dots \sqcup S_n = T_1 \sqcup \dots \sqcup T_n$ und $S_j, T_j\in \mS$. Dann gilt
        \begin{align*}
            S_j &= S_j \cap \pair{\bigsqcup_{k=1}^n S_k} = S_j \cap \pair{\bigsqcup_{k=1}^n T_k}\\
            &= \bigsqcup_{k=1}^n \pair{S_j \cap T_k}
            \intertext{Analog ist auch}
            T_k &= \bigsqcup_{j=1}^n \pair{T_k \cap S_j}\\
            \impl S_1 \sqcup \dots \sqcup S_M &= \bigsqcup_{j=1}^M \bigsqcup_{k=1}^N \pair{S_j \cap T_k}\\
            \impl \overline{\mu}\of{S_1 \sqcup \dots \sqcup S_n} &= \overline{\mu}\of{\bigsqcup_{j=1}^M \bigsqcup_{k=1}^M\pair{S_j \cap T_k}}\\
            \sum_{n=1}^{M} \sum_{k=1}^{N} \mu\of{S_j \cap T_k} &= \sum_{k=1}^{N} \mu\of{T_k} = \overline{\mu}\of{T_1 \sqcup \dots \sqcup T_N}
        \end{align*}
        Das heißt $\overline{\mu}$ ist wohldefiniert.\\[.5\baselineskip]
        Frage: Wie verhält sich $\mS_{\cup}$ unter allgemeinen (endlichen) Vereinigungen und Schnitten? Wir betrachten $S, T\in \mS_{\cup}$ mit
        \begin{align*}
            S\cap T &= \pair{S_1 \sqcup \dots \sqcup S_M} \cap \pair{T_1 \sqcup \dots \sqcup T_N}\\
            &= \bigsqcup_{j=1}^M \bigsqcup_{k=1}^N \pair{S_j \cap T_k} \in \mS_{\cup}
        \end{align*}
        Das heißt $\mS_{\cup}$ ist stabil unter (endlichen) Schnitten.
        \begin{align*}
            S\setminus T &= \pair{S_1 \sqcup \dots \sqcup S_M} \setminus\pair{T_1 \sqcup \dots \sqcup T_N}\\
            \impl S \cap \comp{T} &= \pair{S_1 \sqcup \dots \sqcup S_M} \cap \bigcap_{n=1}^N \comp{T_k}\\
            &= \bigsqcup_{j=1}^M \pair{S_j \cap \bigcap_{k=1}^N \comp{T_k}}\\
            &= \bigsqcup_{j=1}^M \bigcap_{k=1}^N \pair{S_j \cap \comp{T_k}} \in \mS_{\cup}
        \end{align*}
        Das heißt für $S, T \in \mS_{\cup}$ ist auch $S\setminus T\in\mS_{\cup}$. Außerdem können wir schreiben
        \begin{align*}
            S\cup T &= \pair{S\setminus T} \sqcup \pair{S\cap T} \sqcup \pair{T \setminus S} \in \mS_{\cup}
        \end{align*}
        Das heißt $\mS_{\cup}$ ist ein Mengenring und $\overline{\mu}$ ist eine Fortsetzung von $\mu$ auf $\mS_{\cup}$.
        \begin{align*}
            \impl \overline{\mu}\of{T \cup S} &= \overline{\mu}\of{S \setminus T} + \overline{\mu}\of{S \cap T} + \overline{\mu}\of{T \setminus S}
        \end{align*}
        Das heißt $\overline{\mu}$ ist definiert für endlich viele Vereinigungen von Mengen aus $\mS_{\cup}$.\\[.5\baselineskip]
        Behauptung: $\overline{\mu}$ ist ein Prämaß auf $\mS_{\cup}$. Also ist zu zeigen, dass $\overline{\mu}$ $\sigma$-additiv auf $\mS_{\cup}$ ist. Wir nehmen $(T_k)_k \subseteq \mS_{\cup}$
        \begin{align*}
            T &\coloneqq \bigsqcup_{k\in\N} T_k \in \mS_{\cup}
            \intertext{Zu zeigen:}
            \overline{\mu}\of{T} &= \sum_{k\in\N}^{} \overline{\mu}\of{T_k}
            \intertext{Nach Definition von $\mS_{\cup}$ gibt es $(S_n)_n \subseteq \mS$ und Indizes $0= i\of{0} \leq i\of{1} \leq i\of{2} \leq \dots$ mit}
            T_k &= S_{i\of{k-1} + 1} \sqcup \dots \sqcup S_{i\of{k}}\tag{$k\in\N$}\\
            T &= U_1 \sqcup \dots \sqcup U_L\\
            U_l &\coloneqq \bigsqcup_{i\in J_l} S_i
            \intertext{Indexmengen $J_1, \ldots, J_l \subseteq \N$ paarweise disjunkt und $J_1 \sqcup \ldots \sqcup J_L = \N$}
            \overline{\mu}\of{T} &= \overline{\mu}\of{U_1 \sqcup \ldots \sqcup U_L}\\
            &= \overline{\mu}\of{U_1} +\ldots + \overline{\mu}\of{U_l}\\
            &= \mu\of{U_1} + \ldots + \mu\of{U_l}\\
            &= \sum_{i\in J_1}^{} \mu\of{S_i} + \ldots + \sum_{i\in J_L}^{} \mu\of{S_i}\\
            &= \sum_{j=1}^{\infty} \mu\of{S_j} = \sum_{k=1}^{\infty} \overline{\mu}\of{T_k}
        \end{align*}
    \end{konstruktion}

    \begin{bemerkung}[Prämaß und äußeres Maß]
        \marginnote{[29. Nov]}
        Wenn wir ein Prämaß $\mu$ auf einem Halbring $\mS$ haben, dann gilt
        \begin{align*}
            \mu\of{A} &= \mu^{\ast}\of{A}\quad\forall A\in\mS
        \end{align*}
        wobei $\mu^{\ast}$ das in Konstruktion~\ref{konstruktion:prem} definierte äußere Maß ist. Das heißt $\mu^{\ast}$ ist eine Fortsetzung von $\mu$.
        \begin{proof}
            Sei $A\in\mS$ und $(S_j)_j \in\mC\of{A}$ Überdeckung von $A$. Dann gilt
            \begin{align*}
                \mu\of{A} &= \overline{\mu}\of{A} = \overline{\mu}\of{\pair{\bigcup_{j\in\N} S_j}\cap A}\\
                \intertext{Wegen der $\sigma$-Subadditivität von $\overline{\mu}$ gilt}
                &\leq \sum_{j=1}^{\infty} \overline{\mu}\of{S_j \cap A} \leq \sum_{j=1}^{\infty} \overline{\mu}\of{S_j}
                \intertext{Wir nehmen das Infimum auf beiden Seiten und erhalten}
                \impl\mu\of{A} &\leq \mu^{\ast}\of{A}\quad\forall A\in\mS
                \intertext{Wir wollen noch zeigen, dass $\mu\of{A} \geq \mu^{\ast}\of{A}$. Für ein $A\in\mS$ nehmen wir $(S_j)_j$ mit $S_1 \coloneqq A$, $S_2 = S_3 = \ldots = \emptyset$}
                \impl \mu^{\ast}\of{A} &\leq \mu\of{A}\quad\forall A\in\mS\\
                \impl \mu^{\ast}\of{A} &= \mu\of{A}\quad\forall A\in\mS\qedhere
            \end{align*}
        \end{proof}
    \end{bemerkung}

    \begin{definition}[Zerlegungsbedingung]
        Sei $\mu^{\ast}$ ein äußeres Maß auf $\mP\of{X}$. Dann sagen wir $A\subseteq X$ erfüllt die Zerlegungsbedingung, falls
        \begin{align*}
            \mu^{\ast}\of{B} &= \mu^{\ast}\of{B \cap A} + \mu^{\ast}\of{B \setminus A}\quad\forall B\subseteq X\numbereq{eq:zerleg}
        \end{align*}
        Außerdem definieren wir
        \begin{align*}
            \mA_{\ast} &\coloneqq \set{A \subseteq X: A\text{ erfüllt die Zerlegungsbedingung~\ref{eq:zerleg}}}
        \end{align*}
    \end{definition}

    \begin{bemerkung}
        Die Bedingung~\ref{eq:zerleg} ist äquivalent zu der Bedingung
        \begin{align*}
            \mu^{\ast}\of{B} &\geq \mu^{\ast}\of{B \cap A} + \mu^{\ast}\of{B \setminus A}\quad\forall B\subseteq X
        \end{align*}
        da die Ungleichung in die andere Richtung bereits durch die $\sigma$-Subadditivität von $\mu^{\ast}$ gegeben ist.
    \end{bemerkung}

    \begin{lemma}
        Sei $\mu^{\ast}$ das vom Prämaß $\mu$ auf $\mS$ erzeugte äußere Maß. Dann gilt
        \begin{align*}
            \mS \subseteq \mA_{\ast}
        \end{align*}
        \begin{proof}
            Sei $A \in\mS$. Wir wollen zeigen, dass
            \begin{align*}
                \mu^{\ast}\of{B} &= \mu^{\ast}\of{B \cap A} + \mu^{\ast}\of{B \setminus A}\quad\forall B\subseteq X
            \end{align*}
            O.B.d.A sei $\mC\of{B} \neq \emptyset$. Sei $(B_n)_n \in \mC\of{B}$, $B_n\in\mS$. Dann gilt
            \begin{align*}
                B_n &= \pair{B_n \cap A} \sqcup \pair{B_n \setminus A}\\
                \impl\mu\of{B_n} &= \overline{\mu}\of{B_n} = \overline{\mu}\of{\pair{B_n \cap A}} + \overline{\mu\of{B_n \setminus A}}\\
                \impl \sum_{n=1}^{\infty} \mu\of{B_n} &= \sum_{n=1}^{\infty} \pair{\mu\of{B_n \cap A}} + \sum_{n=1}^{\infty} \pair{\overline{\mu}\of{B_n \setminus A}}
                \intertext{Es gilt $\pair{B_n \cap A}_n \in\mC\of{B \cap A}$ und $\pair{B_n \setminus A}_n \in\mC\of{B\setminus A}$}
                &\geq \mu^{\ast}\of{B\cap A} + \overline{\mu}^{\ast}\of{B\setminus A}\\
                &= \mu^{\ast}\of{B \cap A} + \mu^{\ast}\of{B\setminus A}\\
                \impl \mu\of{B} &\geq \mu^{\ast}\of{B \cap A} + \mu^{\ast}\of{B \setminus A}\qedhere
            \end{align*}
        \end{proof}
    \end{lemma}
    \noindent Wir sind jetzt in der Lage, den Satz von Carathéodory zu beweisen.
    \begin{proof}[Beweis von Satz~\ref{satz:caratheodory}]
        \textsc{Schritt 1}: Wir zeigen, dass $\mA_{\ast}$ eine Algebra ist. Die Stabilität unter Komplementbildung und $\emptyset\in\mA_{\ast}$ zeigt sich leicht. Wir wollen also noch zeigen, dass $A_1 \cup A_2 \in \mA_{\ast}$. Das heißt es soll gelten
        \begin{align*}
            \mu^{\ast}\of{B} &= \mu^{\ast}\of{B\cap \pair{A_1 \cup A_2}} + \mu^{\ast}\of{B \setminus\pair{A_1 \cup A_2}}\\
            \intertext{Wir definieren}
            B_1 &\coloneqq \pair{A_1 \cap B} \setminus A_2\\
            B_2 &\coloneqq \pair{A_2 \cap B} \setminus A_1\\
            B_3 &\coloneqq B \cap A_1 \cap A_2\\
            B_4 = B \setminus \pair{A_1 \cup A_2}
            \intertext{Dann gilt}
            B \cap \pair{A_1 \cup A_2} &= \pair{B \cap A_1}\cup\pair{B\cap A_2}\\
            &= B_1 \cup B_2 \cup B_3\\
            B \setminus\pair{A_1 \cup A_2} &= B_4
            \intertext{Das heißt es ist zu zeigen, dass}
            \mu\of{B} &= \mu\of{B_1 \sqcup B_2 \sqcup B_3} + \mu\of{B_4}
            \intertext{$A_1$ erfüllt die Zerlegungsbedingung. Also}
            \mu^{\ast}\of{B} &= \mu^{\ast}\of{B \cap A_1} + \mu^{\ast}\of{B \setminus A_1}\\
            &= \mu^{\ast}\of{B_1 \cup B_3} + \mu^{\ast}\of{B_2 \cup B_4}
            \intertext{Verwende $A_2$, um $B_1 \cup B_3$ zu zerlegen}
            \mu^{\ast}\of{B_1 \cup B_3} &= \mu^{\ast}\of{\pair{B_1 \cup B_2} \cap A_2} + \mu^{\ast}\of{\pair{B_1 \cup B_2} \setminus A_2}\\
            &= \mu^{\ast}\of{B_1} + \mu^{\ast}\of{B_3}
            \intertext{Genauso zerlegen von $B_2 \cup B_4$ mittels $A_2$}
            \mu^{\ast}\of{B_2 \cup B_4} &= \mu^{\ast}\of{B_2} + \mu^{\ast}\of{B_4}\\
            \impl \mu^{\ast}\of{B} &= \mu^{\ast}\of{B_1} + \mu^{\ast}\of{B_2} + \mu^{\ast}\of{B_3} + \mu^{\ast}\of{B_4}
            \intertext{Machen dasselbe mit $\overline{B} \coloneqq B_1 \cup B_2 \cup B_3$}
            \impl \mu^{\ast}\of{B_1 \cup B_2\cup B_3} &= \mu^{\ast}\of{B_1} + \mu^{\ast}\of{B_2} + \mu^{\ast}\of{B_3}
        \end{align*}
        Es folgt dann $A_1 \cup A_2$ erfüllt die Zerlegungsbedingung und damit $A_1 \cup A_2 \in\mA_{\ast}$.\\
        \marginnote{[02. Dez]}
        \textsc{Schritt 2}: Wir zeigen, dass $\mu^{\ast}$ endlich additiv auf $\mA_{\ast}$ ist. Seien $A_1, A_2\in\mA_{\ast}$ mit $A_1 \cap A_2 = \emptyset$. Wir benutzen $A_1$, um $A_1 \sqcup A_2$ zu zerlegen. Es gilt
        \begin{align*}
            \mu^{\ast}\of{A_1 \sqcup A_2} &= \mu^{\ast}\of{\pair{A_1 \sqcup A_2} \setminus A_1} + \mu^{\ast}\of{\pair{A_1 \sqcup A_2} \setminus A_2}\\
            &= \mu^{\ast}\of{A_1} + \mu^{\ast}\of{A_2}
        \end{align*}
        \textsc{Schritt 3}: Wir zeigen, dass $\mA_{\ast}$ eine $\sigma$-Algebra und $\mu^{\ast}$ ein Maß auf $\mA_{\ast}$ ist. Sei $(A_j)_j \subseteq \mA_{\ast}$ und $A = \bigsqcup_{j\in\N} A_j$. Zu zeigen ist
        \begin{align*}
            \mu^{\ast}\of{B} &= \mu^{\ast}\of{B \cap A} + \mu^{\ast}\of{B\setminus A}\quad\forall B\subseteq X
        \end{align*}
        Dabei reicht $\geq$ zu zeigen, da die andere Ungleichung allgemein erfüllt ist. WIr definieren
        \begin{align*}
            F_n &\coloneqq \bigsqcup_{j=1}^n A_j \in\mA_{\ast}
            \intertext{mit $F_n\nearrow A$ und}
            B\setminus A &\subseteq B \setminus F_n\quad\forall n\in\N
            \intertext{Also wird $B$ von $F_n$ zerlegt}
            \mu^{\ast}\of{B} &= \mu^{\ast}\of{B \cap F_n} + \mu^{\ast}\of{B \setminus F_n}\\
            &= \mu^{\ast}\of{B\cap \pair{\bigsqcup_{j=1}^{n} A_j}} + \mu^{\ast}\of{B \cap \bigcap \comp{A_j}}\\
            \intertext{Wegen der endlichen Additivität von $\mu^{\ast}$ gilt}
            &= \sum_{j=1}^{n} \mu^{\ast}\of{B \cap A_j} + \mu^{\ast}\of{B \cap \bigcap \comp{A_j}}\\[.5\baselineskip]
            B \cap A &= B \cap \pair{\bigsqcup_{j\in\N} A_j} = \bigsqcup_{j\in\N}\pair{B \cap A_j}\\
            \impl \mu^{\ast}\of{B \cap A} &= \mu^{\ast}\of{\bigsqcup_{j\in\N} \pair{B \cap A_j}}\\
            \annot[{&}]{\leq}{$\sigma$-Subadd.} \sum_{j\in\N}^{} \mu^{\ast}\of{B \cap A_j}\\[.5\baselineskip]
            \mu^{\ast}\of{B \cap A} + \mu^{\ast}\of{B\setminus A} &\leq \sum_{j=1}^{n} \mu^{\ast}\of{B \cap A_j} + \mu^{\ast}\of{B\setminus A}\\
            &= \lim_{n\toinf} \sum_{j=1}^{n} \pair{\mu^{\ast}\of{B \cap A_j} + \mu^{\ast}\of{B \setminus A}}
            \intertext{Aber wir wissen $B\setminus A \subseteq B \setminus F_n$. Damit folgt}
            &\leq \lim_{n\toinf} \sum_{j=1}^{n} \pair{\mu^{\ast}\of{B \cap A_j} + \mu^{\ast}\of{B \setminus F_n}}\\
            &= \lim_{n\toinf} \pair{\mu^{\ast} \of{B\cap F_n} + \mu^{\ast}\of{B \setminus F_n}} = \mu\of{B}
        \end{align*}
        Das heißt $A$ erfüllt die Zerlegungsbedingung und damit $A = \bigsqcup_{j\in\N} A_j \in\mA_{\ast}$. Das heißt $A_{\ast}$ ist ein Dynkinsystem. Wir wollen Satz~\ref{lemma:dynkin-sigma-equiv} anwenden und müssen daher noch zeigen, dass $A_{\ast}$ $\cap$-stabil ist. Es gilt
        \begin{align*}
            A_1 \cap A_2 &= \comp{\pair{\comp{A_1} \cup \comp{A_2}}} \in\mA_{\ast}
        \end{align*}
        für $A_1, A_2\in\mA_{\ast}$. Damit ist $A_{\ast}$ nach Lemma~\ref{lemma:dynkin-sigma-equiv} eine $\sigma$-Algebra.\\[\baselineskip]
        \textsc{Schritt 4}: Wir zeigen, dass $\mu^{\ast}$ ein Maß auf $\mA_{\ast}$ ist. Sei
        \begin{align*}
            B\coloneqq A = \bigsqcup_{j\in\N} A_j \in\mA
            \intertext{Wir zerlegen $B$ mittels $A$}
            \impl \mu^{\ast}\of{B} &\geq \sum_{j=1}^{n} \mu^{\ast}\of{B \cap A_j} + \mu^{\ast}\of{B\setminus A}\\
            \impl \mu^{\ast}\of{A} &\geq \sum_{j=1}^{n} \mu^{\ast}\of{A_j}
            \intertext{Durch die $\sigma$-Subadditivität haben wir auch die Ungleichung in die andere Richtung. Also folgt}
            \mu^{\ast}\of{A} &= \mu^{\ast}\of{\bigsqcup_{j\in\N} A_j} = \sum_{j\in\N}^{} \mu^{\ast}\of{A_j}\qedhere
        \end{align*}
    \end{proof}
    \noindent Um den Satz von Carathéodory auf $\lambda^d$ anwenden zu können, brauchen wir noch
    \begin{enumerate}[label=-]
        \item $\mJ^d \coloneqq$ \textit{Mengensystem der (rechts-)halboffenen Intervalle im $\R^d$} ist ein Halbring
        \item $\lambda^d: \mJ^d \to \interv{0,\infty}$ ist ein Prämaß
    \end{enumerate}

    \begin{lemma}
        \label{lemma:halbring-dotprod}
        Seien $X_1, X_2$ Mengen und $\mS_1$ ein Halbring in $X_1$ sowie $\mS_2$ Halbring in $X_2$. Dann gilt $\mS_1 \times \mS_2$ ist ein Halbring in $X_1 \times X_2$.
        \begin{proof}
            \theoremescape
            \begin{enumerate}[label=($\text{S}_{\arabic*}$)]
                \item $\emptyset\in \mS_1 \times \mS_2$ folgt direkt aus $\emptyset \in \mS_1$ und $\emptyset\in\mS_2$
                \item Seien $J_1^{1}, J_1^2\in\mS_1$ und $J_2^{1}, J_2^2\in\mS_2$. Dann gilt
                \begin{align*}
                    \pair{J_1^1 \times J_2^1} \cap \pair{J_1^2 \times J_2^2} &= \pair{J_1^1 \cap J_1^2}\times\pair{J_2^1 \cap J_2^2} \in \mS_1 \times \mS_2
                \end{align*}
                \item (Übung)
            \end{enumerate}
        \end{proof}
    \end{lemma}

    \begin{satz}
        \begin{align*}
            \mJ^{d} &\coloneqq \set{\linterv{a,b}: a,b\in\R^d, a <b}
        \end{align*}
        ist ein Halbring.
        \begin{proof}
            Für eine Dimension (das heißt $d=1$) zeigen sich die Halbring-Eigenschaften von $\mJ^1$ direkt durch Fallunterscheidungen. Wir wollen also für höhere Dimensionen einfach Lemma~\ref{lemma:halbring-dotprod} induktiv anwenden. Wir definieren also
            \begin{align*}
                \mJ^{2}&\coloneqq \mJ^1 \times \mJ^1
                \intertext{Das ist nach dem Lemma ein Halbring}\\
                \impl \mJ^{3} &\coloneqq \mJ^2 \times \mJ^1\text{ ist ein Halbring}\\
                &\vdots\\
                \impl\mJ^{3} &\coloneqq \mJ^{d-1}\times \mJ^1\text{ ist ein Halbring}\qedhere
            \end{align*}
        \end{proof}
    \end{satz}

    \begin{lemma}
        Sei $\mu$ endlich-additiv auf einem Halbring $\mS$. Wir betrachten die folgenden Aussagen
        \begin{enumerate}[label=(\alph*)]
            \item $\mu$ ein Prämaß
            \item $A_n \in \mS \land A_n \nearrow A \in \mS \impl \lim_{n\toinf} \mu\of{A_n} = \mu\of{A}$
            \item $A_n\in\mS \land A_n \searrow A \in \mS \land \mu\of{A_1} < \infty \impl \lim_{n\toinf} \mu\of{A_n} = \mu\of{A}$
            \item $A_n \in\mS\land A_n \searrow \emptyset \land \mu\of{A_1} < \infty \impl \lim_{n\toinf} A_n = 0$
        \end{enumerate}
        Dann gilt (a) $\equivalent$ (b) $\impl$ (c) $\equivalent$ (d). Gilt ferner $\mu\of{A} < \infty~\forall A\in\mS$, dann ist auch (c) $\impl (b)$. In diesem Fall sind also alle Aussagen äquivalent und damit verwendbar, um zu prüfen, ob $\mu$ ein Prämaß ist.
        \begin{proof}
        (Später)
        \end{proof}
    \end{lemma}

    \begin{satz}
        Sei $\lambda^d$ für $A\in\mJ^d$ mit
        \begin{align*}
            A &= \times_{j=1}^d \linterv{a_j, b_j} = \linterv{a,b}
            \intertext{definiert als}
            \lambda^d\of{\linterv{a,b}} &\coloneqq \prod_{j=1}^{d} \pair{b_j - a_j}
        \end{align*}
        Dann ist $\lambda^d$ ein Prämaß auf $\mJ^d$.
        \begin{proof}
            Es gilt $\lambda^d\of{\emptyset} = \lambda^d\of{\linterv{a,a}} = 0$. ??
        \end{proof}
    \end{satz}

    \marginnote{[06. Dez]}

    Wir wollen Maße nun einfacher darstellen, als nur als Fortsetzung eines Prämaßes, was auf einem Halbring definiert ist. Dafür sei nun $\mu$ ein Borelmaß auf $\R^d$ mit $\mu\of{K} < \infty$ für alle kompakten Mengen $K\subseteq \R^d$.

    \begin{konstruktion}[Maße mit Funktionen identifizieren]
        Wir betrachten zunächst nur den Fall $d=1$. Wir wollen $\mu$ auf den halboffenen Intervallen durch eine Funktion $F: \R \to \R$ darstellen. Das heißt es soll gelten
        \begin{align*}
            \mu\of{\linterv{a,b}} &= F\of{b}- F\of{a}\quad\forall a,b\in\R
        \end{align*}
        Welche Eigenschaften hat dann die Funktion $F$?
        \begin{enumerate}
            \item $F$ ist monoton wachsend
            \begin{align*}
                F\of{b} - F\of{a} &= \mu\of{\linterv{a,b}}\geq 0 \quad\forall b\geq a
            \end{align*}
            \item $F$ ist ?seitig stetig. Wir betrachten eine Folge $x_n \to b$ ($a < b$) mit $x_n \leq x_{n+1}$. Dann gilt
            \begin{align*}
                \bigcup_{n\in\N} \linterv{a, x_n} &= \linterv{a,b}
                \intertext{Da $\mu$ ein Maß ist, ist es von unten stetig. Das heißt es gilt}
                F\of{x_n} - F\of{a} &= \mu\of{\linterv{a, x_n}} \nearrow \mu\of{\linterv{a, b}} = F\of{b} - F\of{a}\\
                \impl \lim_{x\nearrow b} F\of{x} &= F\of{b}
            \end{align*}
        \end{enumerate}
    \end{konstruktion}

    \begin{beispiel}[Dirac-Maß]

    \end{beispiel}

    \begin{satz}
        Jede wachsende Funktion $F: \R\to\R$ die von rechts stetig ist und (von links Grenzwerte hat) erzeugt ein Borelmaß $\mu_F$, sodass
        \begin{align*}
            \mu_F\of{\linterv{a,b}} &= F\of{b} - F\of{a}\quad\forall a,b\in\R: a\leq b
        \end{align*}
        Ferner: Sei $G: \R\to\R$ wachsend und von rechts stetig mit $\mu_G = \mu_F$. Dann folgt
        \begin{align*}
            \exists c\in\R\colon G\of{x}&= F\of{x}+ C\quad\forall x\in\R
        \end{align*}

        \begin{proof}
            Der letzte Teil zeigt sich durch nachrechnen direkt. Erster Teil: Siehe Literatur.
        \end{proof}
    \end{satz}

    \newpage


    \section{[*] Messbare Abbildungen und Bildmaße}

    \subsection{Messbare Abbildungen}
    \thispagestyle{pagenumberonly}

    \begin{definition}
        Seien $\pair{X, \mA}$ und $\pair{X', \mA'}$ Messräume und $T: X\to X'$ eine Funktion. Dann heißt $T$ $\mA-\mA'$-messbar, falls
        \begin{align*}
            T^{-1}\of{A'} \in \mA\quad\forall A'\in\mA'
        \end{align*}
    \end{definition}

    \begin{beispiel}
        \theoremescape
        \begin{enumerate}
            \item Konstante Funktionen sind messbar
            \item Wir betrachten $\pair{X, \mO}$, $\pair{X', \mO'}$ mit $\mO$ Mengensystem der offenen Mengen. Dann ist eine stetige Funktion $T$ $\mB\of{\mO} - \mB\of{\mO'}$-messbar, da die Urbilder offener Mengen unter stetigen Funktionen offen sind
        \end{enumerate}
    \end{beispiel}

    \begin{satz}
        Seien $\pair{X, \mA}$, $\pair{X', \mA'}$ Messräume und $\mE' \subseteq \mP\of{X'}$ Erzeuger von $\mA' = \sigma\of{\mE'}$. Dann ist $T$ genau dann $\mA$-$\mA'$-messbar, wenn
        \begin{align*}
            T^{-1}\of{E'} \in \mA\quad\forall E'\in\mE'
        \end{align*}
        \begin{proof}
            $\Sigma' \coloneqq \set{A' \subseteq X': T^{-1}\of{A'} \in \mA}$ ist eine $\sigma$-Algebra in $X'$ (Übung) mit $\mE' \subseteq \Sigma'$
            \begin{align*}
                \impl \mA' = \sigma\of{\mE'} &\subseteq \sigma\of{\Sigma'} = \Sigma'\\
                \impl \mA' &= \Sigma'\qedhere
            \end{align*}
        \end{proof}
    \end{satz}

    \begin{satz}
        \label{satz:verknuepfung-messbar}
        Seien $\pair{X_1, \mA_1}, \pair{X_2, \mA_2}, \pair{X_3, \mA_3}$ Messräume und $T_1: X_1 \to X_2$ $\mA_1$-$\mA_2$-messbar sowie $T_2: X_2 \to X_3$ $\mA_2 $- $\mA_3$-messbar. Dann ist $T_2 \circ T_1: X_1 \to X_3$ $\mA_1 $-$ \mA_3$-messbar.

        \begin{proof}
            Sei $A_3\in\mA_3$. Dann ist
            \begin{align*}
                \pair{T_2 \circ T_1}^{-1}\of{A_3} &= T_1^{-1}\underbrace{\of{T_2^{-1}\of{A_3}}}_{\in\mA_2}\in\mA_3\qedhere
            \end{align*}
        \end{proof}
    \end{satz}

    \begin{bemerkung}
        Sei $I$ eine Index-Menge und $\pair{X_j, \mA_j}$ ein Messraum für $j\in I$. Abbildung $T_j: X \to X_j$. Dann ist $\sigma\of{\bigcup_{j\in I} T_j^{-1}\of{\mA_j}}$ die kleinste $\sigma$-Algebra, auf $X$, für die alle Abbildungen $T_j$ messbar (also $\sigma\of{\bigcup_{j\in I} T_j^{-1}\of{\mA_j}}$-$\mA_j$-messbar) sind.
    \end{bemerkung}

    \begin{satz}
        $S: X_0 \to X$ ist $\mA_0$-$\mA$-messbar. Dann ist $\mA = \sigma\of{\bigcup_{j\in I} T^{-1}\of{\mA_j}}$ genau dann, wenn
        \begin{align*}
            T_j \circ S: X_0 \to X_j\text{ ist } \mA_0 \text{-} \mA_j\text{-messbar}
        \end{align*}
        \begin{proof}
            \anf{$\impl$} Folgt direkt aus Satz~\ref{satz:verknuepfung-messbar}.\\[.5\baselineskip]
            \anf{$\Leftarrow$} Nehmen $E \in \bigcup_{j\in I} T_j^{-1}\of{\mA_j}$. Dann ist $E \in T^{-1}_j\of{\mA_j}$ für ein $j\in I$. Dann ist $E = T_j^{-1}\of{\mA_j}$. Es folgt $S^{-1}\of{E} = S^{-1}\of{T_j^{-1}\of{\mA_j}} = \pair{T_j \circ S}^{-1}\of{\mA_j} \subseteq \mA_0$.\\
            $\mE \coloneqq \bigcup_{j\in I} T_j^{-1}\of{\mA_j}$ ist Erzeuger von $\mA \coloneqq \sigma\of{\mE}$. Damit folgt die Behauptung.
        \end{proof}
    \end{satz}

    \subsection{Bildmaße}

    \begin{notation}
        \marginnote{[09. Dez]}
        Seien $\pair{X, \mA}, \pair{X', \mA'}$ Messräume. Wir schreiben $T: \pair{X, \mA} \to \pair{X', \mA'}$ für eine Funktion $T: X \to X'$, die $\mA$-$\mA'$-messbar ist.
    \end{notation}

    \begin{konstruktion}
        \label{konstruktion:bildmass}
        Sei $\mu$ ein Maß auf $\pair{X, \mA}$ und $T: \pair{X, \mA} \to \pair{X', \mA'}$ eine Funktion. Wir nehmen $A' \in \mA'$ und definieren
        \begin{align*}
            \mu'\of{A'} &\coloneqq \mu\of{T^{-1}\of{A'}}
        \end{align*}
        Dann ist $\mu'$ ein Maß auf $\pair{X', \mA'}$.
    \end{konstruktion}

    \begin{definition}[Bildmaß]
        Im Sinne von Konstruktion~\ref{konstruktion:bildmass} nennen wir $\mu'$ das Bild von $\mu$ unter $T$. Das heißt wir schreiben
        \begin{align*}
            \mu' &\coloneqq T\of{\mu}
            \intertext{Es gilt}
            T\of{\mu}\of{A'} &\coloneqq \mu\of{T^{-1}\of{A'}}
        \end{align*}
    \end{definition}

    \begin{bemerkung}[Transitivität der Bildmaße]
        Ist $T_1: \pair{X_1, \mA_1} \to \pair{X_2, \mA_2},~T_2: \pair{X_2, \mA_2} \to \pair{X_3, \mA_3}$ und $\mu$ ein Maß auf $\pair{X_1, \mA_1}$. Dann gilt
        \begin{align*}
            \pair{T_2 \circ T_1}\of{\mu} &= T_2\of{T_1\of{\mu}}\\
            \pair{T_2 \circ T_1}^{-1} &= T_1^{-1}\circ T_2^{-1}\\
            \impl \pair{T_2 \circ T_1}^{-1}\of{A_3} &=T_1^{-1}\of{T_2^{-1}\of{A_3}}
        \end{align*}
    \end{bemerkung}

    \newpage


    \section{[*] Messbare numerische Funktionen}
    \imaginarysubsection{Messbare numerische Funktionen}
    \thispagestyle{pagenumberonly}

    \begin{definition}[$\overline{\R}$]
        Wir schreiben $\overline{\R} \coloneqq \R \cup \set{-\infty, \infty}$ für die Menge der reellen Zahlen einschließlich $\pm\infty$. Wir definieren die algebraischen Operationen wie folgt:
        \begin{align*}
            x + \infty &\coloneqq \infty\quad\forall x\in\R\\
            x - \infty &\coloneqq -\infty\quad\forall x\in\R\\
            \infty + \infty &\coloneqq \infty\\
            -\infty + \pair{-\infty} = -\infty -\infty &\coloneqq -\infty\\
            -\infty &< x < \infty\quad\forall x\in\R\\
            x\cdot\infty &\coloneqq\infty\quad\forall x > 0\\
            x\cdot\pair{-\infty} &\coloneqq -\infty\quad\forall x > 0\\
            x\cdot\infty &\coloneqq\pair{-\infty}\quad\forall x < 0\\
            x\cdot\pair{-\infty} &\coloneqq \infty\quad\forall x < 0\\
            0\cdot\pair{\pm \infty} &\coloneqq 0\\
            \frac{x}{\pm\infty} &\coloneqq 0\quad\forall x\in\R\\
            \infty\cdot\infty &= \infty\\
            \infty\cdot\pair{-\infty} &= -\infty
        \end{align*}
        Dabei bleiben Ausdrücke wie $\infty - \infty$ nicht definiert. Wir definieren offene Mengen im $\overline{\R}$ wie folgt: $U \subseteq \overline{\R}$ ist offen, wenn
        \begin{align*}
            \forall x\in U\cap\R\ex R>0\colon \pair{x-R, x+R} &\subseteq U
            \intertext{Außerdem muss für den Fall $\infty\in U$ zu sätzlich gelten}
            \exists R > 0\colon \rinterv{R, \infty} &\subseteq U\\
            \intertext{Analog muss für $-\infty\in U$ gelten}
            \exists L > 0\colon \linterv{-\infty, L} &\subseteq U
        \end{align*}
        Mit dieser Definition können wir die Borel-$\sigma$-Algebra auf $\overline{\R}$ definieren mit $\overline{\mB}^{1} = \mB\of{\overline{\R}} = \overline{\mB}$.\\
        Dann gilt für die Spur-$\sigma$-Algebra $\overline{\mB}^{1} \cap \R = \mB^1$.
    \end{definition}

    \begin{bemerkung}[Identitätsfunktion]
        Sei $\pair{X, \mA}$ ein Messraum und $A \in\mA$. Dann definieren wir die Identitätsfunktion (von $A$) mit
        \begin{align*}
            \mathbf{1}_A\of{x} &\coloneqq \begin{cases}
                                              1 &x\in A\\
                                              0 &x\not\in A
            \end{cases}
        \end{align*}
        Diese ist Borel-messbar. Außerdem gilt
        \begin{align*}
            A\subseteq B \impl \mathbf{1}_A &\leq \mathbf{1}_B\\
            \mathbf{1}_{\comp{A}} &= 1 - \mathbf{1}_A
        \end{align*}
    \end{bemerkung}

    \begin{definition}[Numerische Funktion]
        Eine numerische Funktion ist eine Funktion $f: X\to\overline{\R}$. Ist $\mA$ eine $\sigma$-Algebra in $X$, so heißt $f$ $\mA$-messbar, falls es $\mA$-$\overline{\mB}$-messbar ist. Das heißt
        \begin{align*}
            \forall B\in\overline{\mB}\colon f^{-1}\of{B} \in\mA
        \end{align*}
    \end{definition}

    \begin{satz}
        \label{satz:messbarkeit}
        Sei $\mA$ eine $\sigma$-Algebra auf $X$. Dann ist eine numerische Funktion $f: X\to\overline{\R}$ genau dann $\mA$-messbar, wenn
        \begin{align*}
            \set{x\in X: f\of{x} \geq \alpha} \in\mA\quad\forall \alpha\in\R
        \end{align*}
        \begin{proof}
            Wir setzen $\overline{\mE} \coloneqq \set{\interv{\alpha, \infty}: \alpha\in\R}$. Zu zeigen ist $\sigma\of{\overline{\mE}} = \overline{\mB}^1$
            \begin{enumerate}[label=(\arabic*)]
                \item Es gilt $\interv{\alpha, \infty} \in \overline{\mB}^1~\forall\alpha\in\R$. Das heißt
                \begin{align*}
                    \impl\Sigma &\coloneqq \sigma\of{\overline{\mE}} \subseteq \overline{\mB}^1\\
                    \linterv{\alpha, \beta} &= \interv{\alpha, \infty} \setminus \interv{\beta, \infty} \in\overline{\mB}^1\tag{$\alpha \leq \beta$}\\
                    \impl \linterv{\alpha, \beta} &\in \R\cap\overline{\mB}^1\\
                    \impl \mB^1 &\subseteq \R \cap \overline{B}^1 = \sigma\of{\R \cap \overline{\mB}^1}
                \end{align*}
                \item Es gilt
                \begin{align*}
                    \set{+\infty} &= \bigcap_{n\in\N} \interv{n, \infty} \in \sigma\of{\mE}\\
                    \set{-\infty} &= \bigcap_{n\in\N} \comp{\interv{-n, \infty}} \in \sigma\of{\mE}\\
                    \impl \forall \overline{G}\in\Sigma\colon \overline{G} \cap \R &= \overline{G} \cap \pair{\comp{\set{-\infty,\infty}}} = \Sigma\\
                    \impl \R \cap \Sigma &\coloneqq \Sigma\\
                    \impl \mB^1 &\subseteq \Sigma\\
                    \impl \overline{\mB}^1 &= \mB^1 \cup \set{0, \set{-\infty}, \set{\infty}, \linterv{-\infty, \infty}} \subseteq \Sigma
                \end{align*}
                Also ist $\overline{\mE}$ ein Erzeuger von $\overline{\mB}^1$ und es genügt die Maßeigenschaften auf dem Erzeuger zu haben.
            \end{enumerate}
        \end{proof}
    \end{satz}

    \begin{satz}
        Sei $\mA$ eine $\sigma$-Algebra auf $X$. Dann ist eine numerische Funktion $f: X\to\overline{\R}$ genau dann $\mA$-messbar, wenn eine der folgenden äquivalenten Bedingungen erfüllt ist:
        \begin{align*}
            &\set{x\in X: f\of{x} \geq \alpha} \in\mA\quad\forall \alpha\in\R\tag{1}\\
            \equivalent &\set{x\in X: f\of{x} > \alpha} \in\mA\quad\forall \alpha\in\R\tag{2}\\
            \equivalent &\set{x\in X: f\of{x} \leq \alpha} \in\mA\quad\forall \alpha\in\R\tag{3}\\
            \equivalent &\set{x\in X: f\of{x} < \alpha} \in\mA\quad\forall \alpha\in\R\tag{4}
        \end{align*}

        \begin{proof}
            Die Äquivalenz zu (1) folgt direkt aus Satz~\ref{satz:messbarkeit}. Warum sind die anderen Aussagen dazu äquivalent? (Übung).
        \end{proof}
    \end{satz}

    \begin{satz}
        \label{satz:messbarkeit-1}
        Für messbare Funktionen $f, g: X \to \overline{\R}$ und $\pair{X, \mA}$ ein Messraum folgt
        \begin{align*}
            \set{f < g}, \set{f \leq g}, \set{f = g}, \set{f \neq g} \in\mA
        \end{align*}
        \begin{proof}
            $\Q$ ist abzählbar. Dann ist
            \begin{align*}
                \set{f < g} &= \set{x\in X: f\of{x} < g\of{x}}\\
                &= \bigcup_{r\in\Q} \set{f < r} \cap \set{r < g} \in\mA\\
                \set{f \leq g} &= \comp{\set{f > g}}\\
                \set{f = g} &= \set{f\leq g} \cap \set{f\geq g} = \mA\\
                \set{f\neq g} &= \comp{\set{f = g}} \in \mA\qedhere
            \end{align*}
        \end{proof}
    \end{satz}

    \begin{satz}
        \marginnote{[13. Dez]}
        Seien $f, g: X \to \R$ $\mA$-messbar. Dann folgt $f\pm g$ (falls definiert) sowie $f\cdot g$ sind messbar.
        \begin{proof}
            \begin{align*}
                \set{f - g\geq \alpha} &= \set{f\geq \underbrace{g + \alpha}_{\eqqcolon h}}\\
                &= \set{f\geq h}
                \intertext{Das heißt es reicht aus zu zeigen, dass $\alpha + g$ (bzw. $h$) messbare Funktionen für ein festes $\alpha\in\R$ sind. $g$ ist messbar, das heißt}
                \set{g \leq \beta} &\in\mA\\
                h\geq\beta &\equivalent g + \alpha \geq \beta\equivalent g \leq \beta - \alpha\\
                \set{h\geq \beta} &= \set{g \leq \beta - \alpha}
                \intertext{Das ist messbar, da $g$ messbar ist und wir Satz~\ref{satz:messbarkeit-1} anwenden können. Damit ist $h$ und damit $f - g$ messbar.\endgraf\noindent Um zu zeigen, dass auch $f+g$ messbar ist, können wir einfach zeigen, dass $-g$ messbar ist. Es gilt}
                \set{-g \geq \gamma} &= \set{g\geq -\gamma}
                \intertext{Damit ist $-g$, also auch $f - (-g) = f + g$ messbar.\endgraf\noindent Wir zeigen noch die Messbarkeit von $f\cdot g$. Annahme: $f$ und $g$ sind reellwertig}
                \pair{f+g}^2 - \pair{f-g}^2 &= 4fg\\
                \impl fg &= \frac{1}{4}\pair{\pair{f+g}^2 - \pair{f-g}^2}
                \intertext{Das heißt es reicht aus, zu zeigen, dass die Messbarkeit einer Funktion $h$ auch die Messbarkeit von $h^2$ impliziert}
                \set{h^2 \geq \beta} &= X\quad\text{ falls } \beta \leq 0\\
                \set{h^2\geq \beta} &= \set{h\geq \sqrt{\beta}} \cup \set{h\leq - \sqrt{\beta}} \in \mA\quad\text{ falls } \beta > 0
                \intertext{Damit haben wir die Messbarkeit von $h^2$ gezeigt. Wir wollen nun noch die Annahme loswerden, dass $f$ und $g$ reellwertig sind. Wir definieren}
                X_1 &\coloneqq \set{fg = +\infty} \cup \pair{\set{f > 0} \cap \set{g = \infty} \sqcup \pair{\set{f < 0} \cap \set{g = -\infty}}}\\
                &~~\cup \pair{\set{f = \infty} \cap \set{g\geq 0} \cup \set{f = \infty} \cap \set{g < 0}} (?)\\
                X_2 &\coloneqq \set{fg = -\infty}\\
                X_3 &\coloneqq \set{fg = 0} = \set{f = 0} \cup \set{g = 0}\\
                X_4 &\coloneqq \comp{\pair{X_1 \cup X_2 \cup X_3}}\\
                \impl f, g: X_4 &\to \R\text{ sind reellwertig}\\
                fg &= fg\mathbf{1}_{X_4} + \infty\mathbf{1}_{X_1} - \infty\mathbf{1}_{X_2} + 0\mathbf{1}_{X_3}
            \end{align*}
            Damit ist $fg$ nach Voraussetzung messbar, da $X_4\in\mA$.\qedhere
        \end{proof}
    \end{satz}

    \begin{satz}[WICHTIG]
        \label{satz:funktfolge-sup}
        Sei $(f_n)_n$ eine Folge messbarer Funktionen $f_n: X\to \overline{\R}$. Dann sind $\sup_n f_n$, $\inf_n f_n$, $\limsup_{n\toinf} f_n$ sowie $\liminf_{n\toinf} f_n$ messbar.
    \end{satz}

    \begin{bemerkung}
        Sei $s\coloneqq \sup_n f_n = \sup_{n\in \N} f_n$ ist punktweise definiert. Das heißt
        \begin{align*}
            s\of{x} &= \pair{\sup_n f_n}\of{x} \coloneqq \sup_n f_n\of{x} = \sup\set{f_n\of{x}: n\in \N}
        \end{align*}
    \end{bemerkung}

    \begin{proof}[Beweis von Satz~\ref{satz:funktfolge-sup}]
        Sei $s\of{x} \coloneqq \sup_{n} f_n\of{x}$. Dann ist
        \begin{align*}
            \set{s \leq \alpha} &= \set{x: s\of{x} \leq \alpha} \eqqcolon A_1\\
            \bigcup_{n\in\N}\set{f_n \leq \alpha} &= \bigcap_{n\in\N} \set{x\in X: f_n\of{x} < \alpha} = A_2 \in \mA
            \intertext{Wir zeigen $A_1 = A_2$. Sei $x\in A_1$. Dann ist}
            \alpha &\geq s\of{x} = \sup f_n\of{x} \geq f_n\of{x}\\
            \impl f_m\of{x} &\leq \alpha\quad\forall m\in\N\\
            \impl A_1 &\subseteq A_2
            \intertext{Sei $x\in A_2$. Dann folgt}
            f_n\of{x} &\leq\alpha\quad\forall n\in\N\\
            \impl \sup f_n\of{x} &\leq \alpha\impl x \in A_1
            \intertext{Damit ist $A_1 = A_2$. Es gilt}
            \inf_n f_n &= -\sup_{n} \pair{-f_n}
            \intertext{ist messbar}
            \pair{\limsup_{n} f_n}\of{x} &= \inf_n \sup_{\alpha \geq n} f_{\alpha}\of{x}\\
            \pair{\liminf_{n} f_n}\of{x} &= \sup_n \inf_{\alpha \geq n} f_{\alpha}\of{x}\qedhere
        \end{align*}
    \end{proof}

    \begin{korollar}
        Sei $(f_n)_n$ eine Folge messbarer Funktionen auf $X$ und $\pair{X, \mA}$ ein Messraum. Angenommen $\forall x\in X$ existiert der Grenzwert $f\of{x} \coloneqq \lim_{n\toinf} f_n\of{x}$ (wir schreiben auch $f = \lim_{n\toinf} f_n$). Dann ist $f$ messbar.

        \begin{proof}
            \begin{align*}
                f = \lim_{n\toinf} f_n &= \limsup_{n\toinf} f_n = \liminf_{n\toinf} f_n
            \end{align*}
            Damit ist $f$ nach dem vorherigen Satz messbar.
        \end{proof}
    \end{korollar}

    \begin{korollar}
        Seien $f_1, \ldots, f_n: X\to\overline{\R}$ messbare Funktionen. Dann sind
        \begin{align*}
            f_1 \lor f_2 \lor \dots \lor f_n &\coloneqq \max\of{f_1, f_2, \ldots, f_n}\tag{punktweise}
            \intertext{sowie}
            f_1 \land f_2 \land \dots \land f_n &\coloneqq \min\of{f_1, f_2, \ldots, f_n}
        \end{align*}
        messbar.
        \begin{proof}
            Nehme die Folge $(\overline{f}_n)_n$ mit $\overline{f}_m = f_m$ für $1\leq m \leq n$ und $\overline{f}_m = f_n$ für $m > n$. Damit ist
            \begin{align*}
                f_1 \lor \dots f_n &= \sup_n \overline{f}_n
            \end{align*}
            messbar nach Satz~\ref{satz:funktfolge-sup}. Die zweite Behauptung zeigt sich analog.
        \end{proof}
    \end{korollar}

    \begin{notation}
        Sei $f: X \to \overline{\R}$ eine Funktion. Wir definieren
        \begin{align*}
            f_+ &\coloneqq f \lor 0 = \max\of{f, 0} \geq 0\tag{Positivteil}\\
            f_- &\coloneqq \pair{-f} \lor 0 \geq 0\tag{Negativteil}
            \intertext{Damit gilt außerdem}
            f &= f_+ - f_-\\
            \abs{f} &= f_+ + f_-
        \end{align*}
    \end{notation}

    \begin{korollar}
        Sei $f: X \to \overline{\R}$ eine Funktion. Dann ist $f$ genau dann $\mA$-messbar, wenn $f_+$ und $f_-$ messbar sind.
    \end{korollar}

    \begin{korollar}
        Sei $f: X \to \overline{\R}$ eine $\mA$-messbare Funktion. Dann ist $\abs{f}$ messbar.
    \end{korollar}

    \newpage


    \section{[*] Elementarfunktionen und ihr Integral}
    \imaginarysubsection{Elementarfunktionen}
    \thispagestyle{pagenumberonly}

    \begin{definition}
        Es sei $\pair{X, \mA}$ ein Messraum. Wir definieren $E_+ \coloneqq E_+\of{X, \mA}$ als die Menge aller nicht-negativen Elementarfunktionen. Das heißt $u\in E_+$, falls $u: X \to \R_+ = \linterv{0, \infty}$ $\mA$-messbar ist und $\bild\of{u}$ endlich viele Werte annimmt. Das heißt
        \begin{align*}
            u\of{X} &= \set{\alpha_1, \alpha_2, \ldots, \alpha_n}\\
            \intertext{mit}
            0&\leq \alpha_j < \infty\quad\forall j \leq n\\
            \alpha_i &\neq \alpha_n\quad\forall i\neq j
        \end{align*}
        Wir setzen $A_j \coloneqq u^{-1}\of{\set{\alpha_j}} \in \mA$. Damit gilt $u\of{x} = \sum_{j=1}^{n} \alpha_j \charfunc_{A_j}\of{x}$. Wir haben den Raum $X$ nun folgendermaßen disjunkt zerlegt: $X= A_1 \sqcup A_2 \sqcup \dots \sqcup A_n$.\\
        Sind umgekehrt nicht-notwendigerweise disjunkte $\tilde{A}_1, \ldots, \tilde{A}_n \in \mA$. Dann ist $ \sum_{j=1}^{n} \tilde{\alpha_j} \charfunc_{\tilde{A}_j}$ eine Elementarfunktion.
    \end{definition}

    \begin{bemerkung}[Eigenschaften von $E_+$]
        Sei $\alpha \in \R_+$ und $u, v\in E_+$. Dann gilt
        \begin{enumerate}[label=-]
            \item $\alpha u\in E_+$
            \item $u + v \in E_+$
            \item $u - v\in E_+$
            \item $u \land v \in E_+$
            \item $u\lor v\in E_+$
        \end{enumerate}

        \begin{proof}
            Folgt direkt aus der Definition und den Messbarkeitseigenschaften aus dem letzten Kapitel.
        \end{proof}
    \end{bemerkung}

    \begin{notation}[Normaldarstellung]
        Sei $u \in E_+$. Dann schreiben wir $u\of{x} = \sum_{j=1}^{n} \alpha_j \charfunc_{A_j}$ als Normaldarstellung mit $\bigsqcup_{j=1}^n = X$.
    \end{notation}

    \begin{lemma}
        \label{lemma:elementar-eind}
        Sei $u\in E_+ = E_+\of{X, \mA}$ mit Normaldarstellungen
        \begin{align*}
            u = \sum_{j=1}^{n} \alpha_j \charfunc_{A_j} &= \sum_{k=1}^{n} \beta_k \charfunc_{B_k}
            \intertext{Sei $\mu$ ein Maß auf $\pair{X, \mA}$. Dann gilt}
            \sum_{j=1}^{n} \alpha_j \mu\of{A_j} &= \sum_{k=1}^{n} \beta_k \mu\of{B_k}
        \end{align*}
        Das heißt die Uneindeutigkeit der Normaldarstellung verschwindet bei der Verwendung eines Maßes.
        \begin{proof}
            \begin{align*}
                X &= A_1 \sqcup \dots \sqcup A_m\\
                &= B_1 \sqcup \dots \sqcup B_n\\
                \impl A_j = \bigsqcup_{k=1}^n \pair{A_j \cap B_k} &\quad B_k = \bigsqcup_{j=1}^m \pair{A_j \cap B_k}\\
                \impl \charfunc_{A_j} = \sum_{k=1}^{n} \charfunc_{A_j \cap B_k} &\quad\charfunc_{B_k} = \sum_{j=1}^{m} \charfunc_{A_j \cap B_k}\\
                \impl u &= \sum_{j=1}^{m} \alpha_j \charfunc_{A_j} &= \sum_{j=1}^{n} \sum_{k=1}^{m} \alpha_j \charfunc_{A_j \cap B_k} = \sum_{j=1}^{n} \sum_{k=1}^{m} \beta_k \charfunc_{A_j \cap B_k}
            \end{align*}
            Jeder Punkt $x\in X$ liegt in genau einer Menge $A_j \cap B_k$
            \begin{align*}
                u\of{x} &= \alpha_{j_0} \charfunc_{A_{j_0} \cap B_k}\of{x} = \beta_{k_0} \charfunc_{A_{j_0} \cap B_{k_0}}\\
                \sum_{j=1}^{m} \alpha_j \mu\of{A_j} &= \sum_{j=1}^{m} \alpha_j \mu\of{\bigcup_{k=1}^n A_j \cap B_k}\\
                &= \sum_{k=1}^{n} \mu\of{A_j \cap B_k} = \sum_{j=1}^{m} \sum_{k=1}^{n} \alpha_j \mu\of{A_j \cap B_k}\\
                &= \sum_{j=1}^{m} \sum_{k=1}^{n} \beta_k \mu\of{A_j \cap B_k} = \sum_{k=1}^{n} \beta_k \mu\of{B_k}\qedhere
            \end{align*}
        \end{proof}
    \end{lemma}

    \begin{definition}
        \marginnote{[16. Dez]}
        Sei $\pair{X, \mA, \mu}$ ein Maßraum und $u$ eine Elementarfunktion. Dann heißt die von der spezifischen Normaldarstellung $\mu = \sum_{j=1}^{n} \alpha_j \charfunc_{A_j}$ unabhängige Zahl (Lemma~\ref{lemma:elementar-eind})
        \begin{align*}
            \int_{}^{} u \dif \mu \coloneqq \sum_{j=1}^{n} \alpha_j \mu\of{A_j}
        \end{align*}
        das ($\mu$-)Integral von $u$ (über $X$). Wir schreiben auch
        \begin{align*}
            \int_{}^{} u \dif \mu = \int_{X}^{} u \dif \mu &= \int_{X}^{} u\of{x} \dif \mu\of{x} = \int_{X}^{} u\of{x} \mu\of{\dif x}
        \end{align*}
        und definieren in diesem Sinne eine Abbildung
        \begin{align*}
            E_+ &\to \overline{\R}\\
            u &\mapsto \int_{}^{} u \dif \mu
        \end{align*}
    \end{definition}

    \begin{lemma}[Eigenschaften des Integrals]
        Für $A\subseteq X$, $\alpha\in\R$ und $u,v\in E_+$ gilt
        \begin{enumerate}[label=(\roman*)]
            \item $\dsty \int_{}^{} \charfunc_{A} \dif \mu = \mu\of{A}$
            \item $\dsty \int_{}^{} \alpha u \dif \mu = \alpha \int_{}^{} u \dif \mu$
            \item $\dsty \int_{}^{} \pair{u + v} \dif \mu = \int_{}^{} u \dif \mu + \int_{}^{} v \dif \mu$
            \item $u\leq v \impl \dsty \int_{}^{} u \dif \mu \leq \int_{}^{} v \dif \mu$
        \end{enumerate}
        \begin{proof}
        (i)
            und (ii) folgen direkt aus der Definition. Für (iii) sei
            \begin{align*}
                u = \sum_{j=1}^{m} \alpha_j \charfunc_{A_j} &\quad v = \sum_{k=1}^{n} \beta_k \charfunc_{B_k}\\
                \impl A_j = \bigsqcup_{k=1}^n A_j \cap B_k &\quad B_k = \bigsqcup_{j=1}^m A_j \cap B_k\\
                u = \sum_{j=1}^{m} \sum_{k=1}^{n} \alpha_j \charfunc_{A_j \cap B_k} &\quad v = \sum_{j=1}^{m} \sum_{k=1}^{n} \beta_k \charfunc_{A_j \cap B_k}\\
                \impl u +v = \sum_{j=1}^{m} \sum_{k=1}^{n} &\pair{\alpha_j + \beta_k} \charfunc_{A_j \cap B_k}\\
                \int_{}^{} u \dif \mu = \sum_{j=1}^{m} \sum_{k=1}^{n} \alpha_j \mu\of{A_j \cap B_k} &\quad \int_{}^{} v \dif \mu = \sum_{j=1}^{m} \sum_{k=1}^{n} \beta_k \mu\of{A_j \cap B_k}
            \end{align*}
            \begin{align*}
                \int_{}^{} \pair{u+v} \dif \mu &= \sum_{j=1}^{m} \sum_{k=1}^{n} \pair{\alpha_j + \beta_k} \mu\of{A_j \cap B_k}\\
                &= \sum_{j=1}^{m} \sum_{k=1}^{n} \alpha_j \mu\of{A_j \cap B_k} + \sum_{j=1}^{m} \sum_{k=1}^{n} \beta_k \mu\of{A_j \cap B_k} = \int_{}^{} u \dif \mu + \int_{}^{} v \dif \mu
            \end{align*}
            Für (iv) gilt mit der Definition, die wir auch schon bei (iii) verwendet haben, dass $\alpha_j \leq \beta_k$ auf $A_j \cap B_k$
            \begin{align*}
                \impl \int_{}^{} u \dif \mu = \sum_{j}^{k} \alpha_j \mu\of{A_j \cap B_k} \leq \sum_{j}^{} \sum_{k}^{} \beta_k \mu\of{A_j \cap B_k} = \int_{}^{} v \dif \mu\qedhere
            \end{align*}
        \end{proof}
    \end{lemma}

    \newpage


    \section{[*] Das Integral von nicht-negativen meßbaren Funktionen}
    \imaginarysubsection{Integral nicht-negativer Funktionen}
    \thispagestyle{pagenumberonly}

    \begin{satz}
        Für jede wachsende Folge $(u_n)_n \subseteq E_+$ und jedes $u\in E_+$ gilt
        \begin{align*}
            u \leq \sup_{n\in\N} u_n \impl \int_{}^{} u \dif \mu \leq \sup_{n\in \N} \int_{}^{} u_n \dif \mu
        \end{align*}

        \begin{proof}
            Sei $u = \sum_{j=1}^{m} \alpha_j \charfunc_{A_j}$ und $0 < \delta < 1$. Wir definieren $B_n \coloneqq \set{x\in X: u_n\of{x} \geq \delta u\of{x}}$. Wir wissen nach Voraussetzung, dass $u_n \leq u_{n+1}$, das heißt $\sup_{n\in\N} u_n\of{x} \geq u\of{x}~\forall x\in X$. Sei $x\in X$ fest. Ist $u\of{x} > 0$, dann gilt $\exists N\of{x}\in\N: u_n\of{x}\geq \delta u\of{x}$ für $n\geq N\of{x}$.\\
            Aus $u_{n+1} \geq u_n$ folgt $B_n \subseteq B_{n+1}$. Außerdem ist $\forall x\in X: x\in B_n$ für ein ausreichend großes $n$. Also haben wir $B_n \nearrow X$. Auch ist $u_n \geq \delta u\charfunc_{B_n}$
            \begin{align*}
                \impl \int_{}^{} u_n \dif \mu &\geq \int_{}^{} \delta u\charfunc_{B_n} \dif \mu\\
                u &= \sum_{j=1}^{n} \alpha_j \charfunc_{A_j}\\
                \impl u \charfunc_{B_n} &= \sum_{j=1}^{n} \alpha_j \charfunc_{A_j \cap B_n}\\
                \delta \int_{}^{} u\charfunc_{B_n} \dif\mu &= \delta \int_{}^{} \sum_{j=1}^{n} \alpha_j \charfunc_{A_j \cap B_n} \dif \mu\\
                &= \delta \sum_{j=1}^{n} \alpha_j \mu\of{A_j \cap B_n} \to \delta \sum_{j=1}^{n} \alpha_j \mu\of{A_j}\\
                \impl \sup_{n\in\N} \int_{}^{} u_n \dif \mu = \lim_{n\toinf} \int_{}^{} u \dif \mu &\geq \delta\sum_{j=1}^{n} \alpha_j \lim_{n\toinf} \mu\of{A_j \cap B_n} = \delta\sum_{j=1}^{n} \alpha_j \charfunc_{A_j} = \delta\int_{}^{} u \dif \mu
            \end{align*}
            Das gilt für alle $\delta < 1$. Mit dem Limes für $\delta \to 1$ folgt dann die Behauptung.
        \end{proof}
    \end{satz}

    \begin{korollar}
        Seien $(u_n)_n, (v_n)_n \subseteq E_+$ wachsende Folgen von Elementarfunktionen. Dann gilt
        \begin{align*}
            \sup_{n\in\N} u_n = \sup_{n\in\N} v_n \impl \sup_{n\in\N} \int_{}^{} u_n \dif \mu = \sup_{n\in\N} \int_{}^{} v_n \dif \mu
        \end{align*}

        \begin{proof}
            Nach dem vorherigen Satz und $u_n \leq \sup_{n\in\N} v_n$ gilt
            \begin{align*}
                \int_{}^{} u_n \dif \mu &\leq \sup_{n\in\N} \int_{}^{} v_n \dif \mu\\
                \impl \sup_{n\in\N} \int_{}^{} u_n \dif x &\leq \sup_{n\in\N} \int_{}^{} v_n \dif \mu\\
                \intertext{Aus Symmetriegründen gilt dann}
                \impl \sup_{n\in\N} \int_{}^{} u_n \dif x &= \sup_{n\in\N} \int_{}^{} v_n \dif \mu
            \end{align*}
        \end{proof}
    \end{korollar}

    \begin{notation}
        Wir schreiben $E^{\ast} = E^{\ast}_+ = E^{\ast}_+\of{X, \mA}$ für die Menge aller numerischen meßbaren Funktionen $f\geq 0$ auf $X$, für die es eine wachsende Folge $(u_n)_n \subseteq E_+$ gibt, sodass $\sup_{n\in\N} u_n = f$. Für $f\in E_+^n$ ist dann
        \begin{align*}
            \int_{}^{} f \dif \mu \coloneqq \sup\set{\int_{}^{} u_n \dif \mu: u_n \subseteq E_+,~\sup_{n\in\N} u_n = f}
        \end{align*}
        Das nennen wir das ($\mu$-)Integral von $f$.
    \end{notation}

    \begin{bemerkung}[Eigenschaften des Integrals]
        \marginnote{[20. Dez]}
        Seien $f, g\in E^{\ast}$ und $\alpha\in\R_+$. Dann gilt
        \begin{enumerate}[label=(\roman*)]
            \item $\alpha f\in E^{\ast}$
            \item $f\pm g \in E^{\ast}$
            \item $fg \in E^{\ast}$
            \item $f \lor g \coloneqq \max\of{f, g}\in E^{\ast}$
            \item $f \land g \coloneqq \min\of{f, g}\in E^{\ast}$
            \item $\dsty \int_{}^{} \pair{\alpha f} \dif \mu = \alpha\int_{}^{} f \dif \mu$
            \item $\dsty \int_{}^{} \pair{f+g} \dif \mu = \int_{}^{} f \dif \mu + \int_{}^{} g \dif \mu$
            \item $f\leq g \impl \dsty \int_{}^{} f \dif \mu \leq \int_{}^{} g \dif \mu$
        \end{enumerate}

        \begin{proof}
            \theoremescape
            \begin{enumerate}
                \item[(i)-(v)] Beachte $f = \sup_{n\in\N} u_n$ und $g = \sup_{n\in\N} v_n$. Damit lassen sich entsprechende Folgen finden, um auch Kombinationen von $f$ und $g$ im obigen Sinne zu finden.
                \item[(vi)] (Übung)
                \item[(vii)] Es ist $f = \lim_{n\toinf} u_n$, $g= \lim_{n\toinf} v_n$. Das heißt
                \begin{align*}
                    \int_{}^{} f \dif \mu &= \sup_{n\in\N} \int_{}^{} u_n \dif \mu = \lim_{n\toinf} \int_{}^{} u_n \dif \mu\\
                    \int_{}^{} g \dif \mu &= \sup_{n\in\N} \int_{}^{} v_n \dif \mu = \lim_{n\toinf} \int_{}^{} v_n \dif \mu\\
                    \int_{}^{} f+g \dif \mu &= \sup_{n\in\N} \int_{}^{} (u_n + v_n) \dif \mu = \lim_{n\toinf} \int_{}^{} (u_n + v_n) \dif \mu\\
                    &= \lim_{n\toinf} \pair{\int_{}^{} u_n \dif \mu + \int_{}^{} v_n \dif \mu} = \lim_{n\toinf} \int_{}^{} u_n \dif \mu + \lim_{n\toinf} \int_{}^{} v_n \dif \mu\\
                    &= \int_{}^{} f \dif \mu + \int_{}^{} g \dif \mu
                \end{align*}
                \item[(viii)] $f\leq g \impl u_k \leq \sup_{n\in\N} v_n$. Wir definieren $(\tilde{u}_n)_n$ mit $\tilde{u}_n \coloneqq u_k$ für ein festes $k$
                \begin{align*}
                    \impl \sup_{n\in\N} \tilde{u}_k = u_k &\leq \sup_{n\in\N} v_n\\
                    \impl \int_{}^{} u_k \dif \mu = \sup_{n\in\N} \int_{}^{} \tilde{u}_k \dif \mu &\leq \sup_{n\in\N} \int_{}^{} v_n \dif \mu = \int_{}^{} g \dif \mu\\
                    \impl \int_{}^{} u_k \dif \mu &\leq \int_{}^{} g \dif \mu\\
                    \impl \int_{}^{} f \dif \mu &= \sup_{k\in\N} \int_{}^{} u_k \dif \mu = \int_{}^{} g \dif \mu
                \end{align*}
            \end{enumerate}
        \end{proof}
    \end{bemerkung}

    \begin{bemerkung}[Interpretation des Integrals]
        Das Integral bezüglich einem Maß $\mu$ ist eine monotone Linearform von $E^{\ast}$ nach $\interv{0, \infty}$. Aber was ist $E^{\ast}$?
    \end{bemerkung}

    \begin{notation}
        Es sei $\mM_+ \coloneqq \mM_+\of{X, \mA} \coloneqq \set{f: x\to\interv{0, \infty},~f\text{ ist }\mA\text{-messbar}}$.
    \end{notation}

    \begin{satz}
        $E^{\ast}\of{X, \mA} = \mM_+\of{X, \mA}$.

        \begin{proof}
            $E^{\ast} \subseteq \mM_+$ ist klar. Wir zeigen die Inklusion in die andere Richtung. Wir definieren für ein $n\in\N$, $j\in\set{0, \ldots, n2^{n}}$
            \begin{align*}
                A_{j,n} &\coloneqq \begin{cases}
                                       \set{f \geq j2^{-n}} \cap \set{f \leq (j+1)2^{-n}} & j\in\set{0, \ldots, n2^{n}-1}\\
                                       \set{f\geq n} &j = n2^{n}-1
                \end{cases}\\
                &= \begin{cases}
                       \set{j2^{-n} \leq f \leq (j+1)2^{-n}}& j\in\set{0, \ldots, n2^{n}-1}\\
                       \set{f\geq n} &j = n2^{n}-1
                \end{cases}
            \end{align*}
            Für ein festes $n$ ist $A_{j,n} \cap A_{k, n} = \emptyset$ für $j\neq k$. Wir setzen
            \begin{align*}
                u_n &\coloneqq \sum_{j=0}^{n2^n} j2^{-n} \charfunc_{A_{j,n}}
            \end{align*}
            Behauptung 1: $u_n \leq u_{n+1}~\forall n\in\N$. Bew.:
            \begin{align*}
                A_{2k, n+1} &= \set{2k2^{-(n+1)} \leq f \leq (2k+1)2^{-(n+1)}} = \set{k2^{-n} \leq f \leq (2k+1)2^{-(n+1)}}\\
                A_{2k+1, n+1} &= \set{(2k+1)2^{-(n+1)} \leq f \leq (2k+1)2^{-(n+1)}}\\
                \impl A_{k,n} &= A_{2k, n+1} \sqcup A_{2k+1, n+1}\\
                \impl u_{n} &\leq u_{n+1}
            \end{align*}
            Damit ist auch $u_n \leq f~\forall n\in\N$. Umgekehrt auf $A_{j,n}$ ist $f\of{x} \leq (j+1)2^{-n} = u_n\of{x} + 2^{-n}$. Das heißt $\lim_{n\toinf} u_n = f$ für $x\in A_{j,n}$. Insgesamt gilt damit $\lim_{n\toinf}u_n = f$.
        \end{proof}
    \end{satz}

    \begin{satz}[Monotone Konvergenz] % Satz 5
        \label{satz:monoton-konv}
        Sei $(f_n)_n$ eine wachsende Folge in $E^{\ast}\of{X, \mA}$. Dann folgt $\dsty\sup_{n\in\N} f_n \in E^{\ast}$ und $\dsty\sup_{n\in\N} \int_{}^{} f_n \dif \mu = \int_{}^{} \sup_{n\in\N} f \dif \mu$.

        \begin{proof}
            Sei $f\coloneqq \sup_{n\in\N} f_n$. Dann reicht es, zu zeigen, dass $\exists (v_n)_n\subseteq E_+$ wachsend mit $\sup_{n\in\N} v_n = f$ und $v_n \leq f_n$. Denn in diesem Fall ist
            \begin{align*}
                \int_{}^{} v_n \dif \mu &\leq \int_{}^{} f_n \dif \mu \leq \sup_{n\in\N} \int_{}^{} f_n \dif \mu\\
                \impl \sup_{n\in\N} \int_{}^{} v_n \dif \mu &\leq \sup_{n\in\N} \int_{}^{} f_n \dif \mu
                \intertext{Mit Bemerkung~\ref{bemerkung:temp-lemma} folgt dann}
                \sup_{n\in\N}\int_{}^{} f_n \dif \mu &= \int_{}^{} \sup_{n\in\N} f_n \dif \mu
                \intertext{Wir brauchen also nur noch die Existenz von $(v_n)_n$. Zu $(f_n)$ existiert ein $(u_{m,n})_n \subseteq E_+$ mit $u_{m, n} \leq u_{m-1, n}$ sowie $f_n = \sup_{m} u_{m,n}$. Das heißt}
                v_m &\leq v_{m+1}\\
                \impl \sup_{m} v_m &\in E_+
                \intertext{Behauptung: $\sup_{m} v_m = f = \sup_n f_n$. Da $u_{m,n} \leq f_n \leq f_{n+1}$}
                \impl \sup_{m} v_m &\leq \sup_{m} f_m = f\\
                \intertext{Auch ist $v_m = u_{m-1} \lor u_{m-2} \lor \ldots \geq u_{m,n} \to f_n$}
                f_n &= \sup_{m} u_{m,n} = \sup_{m} v_m\\
                \impl f &= \sup_{n} f_n \leq \sup_{m} v_m \leq f\\
                \impl f &= \sup_{m\in\N} v_m\qedhere
            \end{align*}
        \end{proof}
    \end{satz}

    \begin{bemerkung}
        \label{bemerkung:temp-lemma}
        Im Sinne von Satz~\ref{satz:monoton-konv} gilt die Ungleichung $\dsty\sup_{n\in\N} \int_{}^{} f_n \dif \mu \leq \int_{}^{} \sup_{n\in\N} f \dif \mu$ allgemein. Die Voraussetzung, dass die Folge wächst ist nur für die Ungleichung in die andere Richtung relevant.
    \end{bemerkung}

    \begin{korollar}
        Sei $(f_n)_n \subseteq E^{\ast}$. Dann folgt $\dsty \sum_{n=1}^{\infty} f_n \in E^{\ast}$ und $\dsty \int_{}^{} \pair{ \sum_{n=1}^{\infty} f_n} \dif \mu = \sum_{n=1}^{\infty} \int_{}^{} f_n \dif \mu$.
        \begin{proof}
            Wir haben $f_n \geq 0$. Das heißt die Partialsummen von $(f_n)_n$ sind monoton wachsend. Damit ergibt sich die Behauptung direkt aus Satz~\ref{satz:monoton-konv}.
        \end{proof}
    \end{korollar}

    \newpage


    \section{[*] Integrierbarkeit}
    \imaginarysubsection{Integrierbarkeit}
    \thispagestyle{pagenumberonly}

    Wir haben $f\geq 0$ messbar auf $X$. Dann ist $\dsty \int_{X}^{} f \dif \mu$ wohldefiniert. Wir wollen das auch auf Funktionen mit negativen Werten erweitern.

    \begin{definition}
        Eine numerische Funktion $f: X\to\overline{\R}$ heißt ($\mu$-)integrierbar, wenn sie $\mA$-messbar ist sowie $ \int_{}^{} f_+ \dif \mu, \int_{}^{} f_- \dif \mu\in\R$. In diesem Fall definieren wir
        \begin{align*}
            \int_{}^{} f \dif \mu &= \int_{}^{} f_+ \dif \mu - \int_{}^{} f_- \dif \mu
        \end{align*}
    \end{definition}

    \begin{satz}
        Sei $f: X \to\overline{\R}$ messbar und $\mu$ ein Maß auf $\mA$. Dann sind folgende Aussagen äquivalent
        \begin{enumerate}[label=(\roman*)]
            \item $f_+$ und $f_-$ sind integrierbar
            \item Es existieren integrierbare Funktionen $u, v\geq 0$ mit $f = u-v$
            \item Es existiert eine integrierbare Funktion $g\geq 0$ mit $\abs{f} \leq g$
            \item $\abs{f}$ ist integrierbar
        \end{enumerate}
        Im Fall (ii) ist $\dsty \int_{}^{} f \dif \mu = \int_{}^{} u \dif \mu - \int_{}^{} v \dif \mu$.
    \end{satz}

\end{document}
