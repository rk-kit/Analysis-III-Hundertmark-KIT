\documentclass[11pt, twoside, a4paper]{article}

% Setup
\usepackage[margin=2.4cm, top=3.5cm]{geometry}
\usepackage[utf8]{inputenc}
\usepackage[ngerman]{babel}

% Package imports
\usepackage{amsfonts}
\usepackage{amsmath}
\usepackage{amssymb}
\usepackage{amsthm}
\usepackage{mathtools}
\usepackage{setspace}
\usepackage{float}
\usepackage{enumitem}
\usepackage{hyperref}
\usepackage[pagestyles]{titlesec}
\usepackage{fancyhdr}
\usepackage{colonequals}
\usepackage{caption}
\usepackage{tikz}
\usepackage{marginnote}
\usepackage{etoolbox}
\usepackage{mdframed}
\usepackage{aligned-overset}
\usepackage{esint}
\usepackage{scalerel}

% Font-Encoding
\usepackage[T1]{fontenc}
\usepackage{lmodern}

% TikZ packages
\usetikzlibrary{patterns}

% Theorems
\newtheoremstyle{plain}{}{}{}{}{\bfseries}{.}{ }{}
\theoremstyle{plain}
\newtheorem{blockelement}{Blockelement}[subsection]
\newtheorem{bemerkung}[blockelement]{Bemerkung}
\newtheorem{definition}[blockelement]{Definition}
\newtheorem{lemma}[blockelement]{Lemma}
\newtheorem{satz}[blockelement]{Satz}
\newtheorem{notation}[blockelement]{Notation}
\newtheorem{korollar}[blockelement]{Korollar}
\newtheorem{uebung}[blockelement]{Übung}
\newtheorem{beispiel}[blockelement]{Beispiel}
\newtheorem{folgerung}[blockelement]{Folgerung}
\newtheorem{axiom}[blockelement]{Axiom}
\newtheorem{beobachtung}[blockelement]{Beobachtung}
\newtheorem{konzept}[blockelement]{Konzept}
\newtheorem{konstruktion}[blockelement]{Konstruktion}
\newtheorem{visualisierung}[blockelement]{Visualisierung}
\newtheorem{anwendung}[blockelement]{Anwendung}
\newtheorem{skizze}[blockelement]{Skizze}
\newtheorem{konvention}[blockelement]{Konvention}
\newtheorem{genv}[blockelement]{}

% Numbering (equations and conditions)
\numberwithin{equation}{subsection}
\newcommand{\numbereq}[1]{\addtocounter{equation}{1}\tag{\theequation}\label{#1}}
\newcounter{condition}
\renewcommand{\thecondition}{V\arabic{condition}}
\newcommand{\condition}[1]{\hypertarget{#1}{\refstepcounter{condition}(\label{#1}\thecondition})}

% Marginnotes left
\makeatletter
\patchcmd{\@mn@@@marginnote}{\begingroup}{\begingroup\@twosidefalse}{}{\fail}
\reversemarginpar
\makeatother

\DeclareMathAlphabet{\altmathbb}{U}{BOONDOX-ds}{m}{n}

% Long equations
\allowdisplaybreaks

% \left \right
\newcommand{\set}[1]{\left\{#1\right\}}
\newcommand{\pair}[1]{\left(#1\right)}
\newcommand{\of}[1]{\mathopen{}\mathclose{}\bgroup\left(#1\aftergroup\egroup\right)}
\newcommand{\abs}[1]{\left\lvert#1\right\rvert}
\newcommand{\norm}[1]{\left\lVert#1\right\rVert}
\newcommand{\linterv}[1]{\left[#1\right)}
\newcommand{\rinterv}[1]{\left(#1\right]}
\newcommand{\interv}[1]{\left[#1\right]}
\newcommand{\scalprod}[1]{\left<#1\right>}

% Shorten commands
\newcommand{\equivalent}[0]{\Leftrightarrow{}}
\newcommand{\impl}[0]{\Rightarrow{}}
\newcommand{\definedasequiv}[0]{\ratio\Leftrightarrow{}}
\renewcommand{\emptyset}{\varnothing}
\newcommand{\dif}{\mathop{}\!\mathrm{d}}
\newcommand{\dsty}{\displaystyle}
\newcommand{\charfunc}{\altmathbb{1}}

\newcommand{\toinf}{\to\infty}
\newcommand{\fa}{\;\forall}
\newcommand{\ex}{\;\exists}
\newcommand{\conj}[1]{\overline{#1}}
\newcommand{\comp}[1]{{#1}^{\mathrm{C}}}

\newcommand{\annot}[3][]{\overset{\text{#3}}#1{#2}}
\newcommand{\anf}[1]{\glqq{}#1\grqq}
\newcommand{\OBDA}{o.B.d.A. }
\newcommand{\theoremescape}{\leavevmode}
\newcommand{\aligntoright}[2]{\hfill#1\hspace{#2\textwidth}~}
\newcommand{\horizontalline}[0]{\par\noindent\rule{0.05\textwidth}{0.1pt}\\}
\newcommand{\rgbcolor}[3]{rgb,255:red,#1;green,#2;blue,#3}
\newcommand{\fixedspace}[2]{\makebox[#1][l]{#2}}

\let\Re\relax
\let\Im\relax

% MathOperators
\DeclareMathOperator{\grad}{Grad}
\DeclareMathOperator{\bild}{Bild}
\DeclareMathOperator{\Re}{Re}
\DeclareMathOperator{\Im}{Im}
\DeclareMathOperator{\arcsinh}{arcsinh}
\DeclareMathOperator{\arccosh}{arccosh}
\DeclareMathOperator{\diam}{diam}
\DeclareMathOperator{\fehler}{Fehler}
\DeclareMathOperator{\D}{D\!}
\DeclareMathOperator{\Id}{Id}
\DeclareMathOperator{\op}{op}
\DeclareMathOperator{\rank}{rk}
\DeclareMathOperator{\spann}{Spann}
\DeclareMathOperator{\flaeche}{Fläche}
\DeclareMathOperator{\bew}{Bew}

% Mengenbezeichner
\newcommand{\R}{\mathbb{R}}
\newcommand{\N}{\mathbb{N}}
\newcommand{\C}{\mathbb{C}}
\newcommand{\Z}{\mathbb{Z}}
\newcommand{\Q}{\mathbb{Q}}
\newcommand{\K}{\mathbb{K}}

\newcommand{\mA}{\mathcal{A}}
\newcommand{\mB}{\mathcal{B}}
\newcommand{\mC}{\mathcal{C}}
\newcommand{\mD}{\mathcal{D}}
\newcommand{\mE}{\mathcal{E}}
\newcommand{\mF}{\mathcal{F}}
\newcommand{\mG}{\mathcal{G}}
\newcommand{\mH}{\mathcal{H}}
\newcommand{\mJ}{\mathcal{J}}
\newcommand{\mK}{\mathcal{K}}
\newcommand{\mL}{\mathcal{L}}
\newcommand{\mM}{\mathcal{M}}
\newcommand{\mO}{\mathcal{O}}
\newcommand{\mP}{\mathcal{P}}
\newcommand{\mQ}{\mathcal{Q}}
\newcommand{\mR}{\mathcal{R}}
\newcommand{\mS}{\mathcal{S}}
\newcommand{\mPC}{\mathcal{PC}}

% Spezielle Symbole
\NewDocumentCommand{\Tau}{e{^_}}{
    \scalerel*{\tau}{X}
    \IfValueT{#1}{^{#1}}
    \IfValueT{#2}{_{\!\!#2}}
}

% Spezielle Commands
\newcommand\imaginarysubsection[1]{
    \refstepcounter{subsection}
    \subsectionmark{#1}
}

% Unfassbar hässlich, aber effektiv für temporäre schnelle Lösungen
\def\:={\coloneqq}
\def\->{\to}
\def\=>{\impl}
\def\<={\leq}
\def\>={\geq}

% Envs
\newenvironment{induktionsanfang}{
    \rule{0pt}{3ex}\noindent
    \begin{minipage}[t]{0.11\textwidth}
    {I-Anfang}
    \end{minipage}
    \hfill
    \begin{minipage}[t]{0.89\textwidth}
    }
    {
    \end{minipage}
}
\newenvironment{induktionsvoraussetzung}{
    \rule{0pt}{3ex}\noindent
    \begin{minipage}[t]{0.11\textwidth}
    {I-Vor.}
    \end{minipage}
    \hfill
    \begin{minipage}[t]{0.89\textwidth}
    }
    {
    \end{minipage}
}
\newenvironment{induktionsschritt}{
    \rule{0pt}{3ex}\noindent
    \begin{minipage}[t]{0.11\textwidth}
    {I-Schritt}
    \end{minipage}
    \hfill
    \begin{minipage}[t]{0.89\textwidth}
    }
    {
    \end{minipage}
}

% Section style
\titleformat*{\section}{\LARGE\bfseries}
\titleformat*{\subsection}{\large\bfseries}

% Page styles
\newpagestyle{pagenumberonly}{
    \sethead{}{}{}
    \setfoot[][][\thepage]{\thepage}{}{}
}
\newpagestyle{headfootdefault}{
    \sethead[][][\thesubsection~\textit{\subsectiontitle}]{\thesection~\textit{\sectiontitle}}{}{}
    \setfoot[][][\thepage]{\thepage}{}{}
}
\pagestyle{headfootdefault}

\begin{document}
    \title{\vspace{3cm} Skript zur Vorlesung\\Analysis III\\bei Prof. Dr. Dirk Hundertmark}
    \author{Karlsruher Institut für Technologie}
    \date{Wintersemester 2024/25}
    \maketitle
    \begin{center}
        Dieses Skript ist inoffiziell. Es besteht kein\\Anspruch auf Vollständigkeit oder Korrektheit.
    \end{center}
    \thispagestyle{empty}
    \newpage

    \tableofcontents
    ~\\
    Alle mit [*] markierten Kapitel sind noch nicht Korrektur gelesen und bedürfen eventuell noch Änderungen.

    \newpage

    \include{Kapitel/Einleitung}
    \include{Kapitel/Sigma_Algebren}
    \section{Dynkinsysteme}
\imaginarysubsection{Dynkinsysteme}
\thispagestyle{pagenumberonly}
\begin{definition}[Dynkinsystem]
    Ein Mengensystem $\mD \subseteq\mP\of{X}$ heißt Dynkinsystem, falls
    \begin{enumerate}[label=($\text{D}_{\arabic*}$)]
        \item $X\in\mD$
        \item $D\in\mD \impl \comp{D}\in\mD$
        \item Für eine paarweise disjunkte Mengenfolge $(D_n)_n \subseteq \mD \impl \dsty\bigsqcup_{n\in\N} D_n \in\mD$
    \end{enumerate}
\end{definition}

\begin{beispiel}
    \theoremescape
    \begin{enumerate}
        \item Jede $\sigma$-Algebra ist ein Dynkinsystem.
        \item Sei $X$ eine $2n$-elementige Menge. Dann ist $\mD \coloneqq \set{A \subseteq X: A\text{ hat eine gerade Anzahl an Elementen}}$ ein Dynkinsystem, aber keine $\sigma$-Algebra.
    \end{enumerate}
\end{beispiel}

\begin{lemma}
    Sei $I$ eine beliebige Indexmenge und $(\mD_j)_{j\in I}$ eine Familie von Dynkinsystemen in $X$, dann ist $\displaystyle\bigcap_{j\in I} \mD_j$ wieder ein Dynkinsystem.
    \begin{proof}
    (Übung)
    \end{proof}
\end{lemma}

\begin{satz} % Satz 4
    Sei $\mG \subseteq\mP\of{X}$. Dann existiert das kleinste Dynkinsystem $\delta\of{\mG}$, welches $\mG$ enthält. Wir nennen $\delta\of{\mG}$ das von $\mG$ erzeugte Dynkinsystem.
    \begin{proof}
        $\mP\of{X}$ ist ein Dynkinsystem. Wir definieren also
        \begin{align*}
            I &= \set{\mD\subseteq\mP\of{X}: \mD\text{ ist ein Dynkinsystem und }\mG\subseteq\mD} \neq \emptyset
        \end{align*}
        Anschließend setzen wir analog zum Schnitt über $\sigma$-Algebren
        \begin{align*}
            \delta\of{\mG} &\coloneqq \bigcap_{\mD\in I} \mD\qedhere
        \end{align*}
    \end{proof}
\end{satz}

\begin{definition}
    Sei $\mD\subseteq\mP\of{X}$. Wir nennen $\mD$ $\cap$-stabil, falls $A, B \in\mD \impl \pair{A \cap B} \in\mD$. Analog dazu nennen wir $\mD$ $\cup$-stabil, falls $A, B \in\mD \impl \pair{A \cup B} \in\mD$.
\end{definition}

Frage: Wann ist ein Dynkinsystem eine $\sigma$-Algebra?

\begin{lemma}
    \label{lemma:dynkin-sigma-equiv}
    Sei $\mD$ ein Dynkinsystem. Dann gilt $\mD$ ist genau dann eine $\sigma$-Algebra, wenn $A, B \in \mD \impl \pair{A \cap B} \in\mD$.

    \begin{proof}
        \theoremescape
        \anf{$\impl$} Sei $\mD$ eine $\sigma$-Algebra. Dann ist $\mD$ ein Dynkinsystem. Seien $A, B\in\mD$. Dann folgt $\comp{A}, \comp{B} \in \mD \impl A \cap B = \comp{\pair{\comp{A} \cup \comp{B}}} \in \mD$.\\[0.5\baselineskip]
        \anf{$\Leftarrow$} Zu zeigen ist Eigenschaft ($\Sigma_3$). Sei $(D_n)_n \subseteq\mD$ eine Mengenfolge. Wir definieren $D_0' \coloneqq \emptyset$ und $D_n' \coloneqq D_1 \cup D_2 \cup \dots \cup D_n$. Dann ist $(D'_n)_n$ eine aufsteigende Folge und es gilt
        \begin{align*}
            \bigcup_{n\in\N} D_n = \bigcup_{n\in\N} D_n' &= \bigsqcup_{n\in\N} \pair{D_n' \setminus D_{n-1}'}
        \end{align*}
        Außerdem ist
        \begin{align*}
            \bigsqcup_{n\in\N} \pair{D_n' \setminus D_{n-1}'} \in &\mD\text{ falls } \pair{D_n' \setminus D_{n-1}'} \in \mD~\forall n\in\N
            \intertext{Und es gilt $D_n' \setminus D_{n-1}' = \pair{D_n' \cap \comp{\pair{D_{n-1}'}}}\in \mD$, falls $D_n' \in \mD~\forall n\in\N_0$. Wir haben also unsere Behauptung gezeigt, wenn wir gezeigt haben, dass $\mD$ $\cup$-stabil ist. Es gilt aber}
            A \cup B &= \comp{\pair{\comp{A} \cap \comp{B}}} \in \mD
        \end{align*}
        Damit ist ($\Sigma_3$) gezeigt.
    \end{proof}
\end{lemma}

\begin{satz} % Satz 6
    \label{satz:dynkin-cap-stabil}
    Sei $X$ eine beliebige Menge und $\mG\subseteq\mP\of{X}$. Dann folgt aus $\mG$ ist $\cap$-stabil, dass $\delta\of{\mG}$ $\cap$-stabil ist.
    \begin{proof}
        Wir nehmen ein beliebiges $D \in \delta\of{\mG}$ und definieren
        \begin{align*}
            \mD_D \coloneqq \set{Q\in\mP\of{X}: Q \cap D \in \delta\of{\mG}}
        \end{align*}
        Behauptung: $\mD_D$ ist ein Dynkinsystem
        \begin{enumerate}[label=($\text{D}_\arabic*$)]
            \item Da $X \cap D = D \in\delta\of{\mG}$ folgt $X\in \mD_D$.
            \item Sei $Q\in\mD_D$. Dann ist auch $\comp{Q}\in\mD_D$, denn $\comp{Q} \cap D = \pair{\comp{Q} \cup \comp{D}}\cap D = \comp{\pair{Q \cap D}} \cap D = D \setminus\pair{Q\cap D} \in\delta\of{\mG}$.
            \item (Siehe handschriftliches Skript)
        \end{enumerate}
        Nun können wir folgendermaßen argumentieren: Da $\mG$ $\cap$-stabil ist, gilt
        \begin{align*}
            \forall G, D\in \mG\colon &G \cap D\in \mG \subseteq \delta\of{\mG}\\
            \equivalent \forall D\in\mG\colon &\mG \subseteq\mD_D\\
            \impl \forall D\in\mG\colon &\delta\of{\mG}\subseteq \delta\of{\mD_D} \annot{=}{(Beh.)} \mD_D\\
            \equivalent \forall D\in\mG\fa G\in\delta\of{\mG}\colon &G \cap D \in\delta\of{\mG}
            \intertext{Aus Symmetriegründen gilt dann}
            \forall G\in\delta\of{\mG}\fa D\in \mG\colon &D \cap G = G\cap D \in \delta\of{\mG}\\
            \equivalent \forall G\in \delta\of{\mG}\colon &\mG \subseteq \mD_{G}\\
            \impl \delta\of{\mG} \subseteq \delta\of{\mD_G} &= \mD_G\quad\forall G\in\delta\of{\mG}\\
            \equivalent \forall D, G\in\delta\of{\mG}\colon &D \cap G\in \delta\of{\mG}
        \end{align*}
        Das heißt $\delta\of{\mG}$ ist $\sigma$-stabil.\qedhere
    \end{proof}
\end{satz}

\begin{korollar}
    \label{korollar:dynkin-sigma}
    Sei $X$ eine beliebige Menge und $\mG \subseteq\mathcal{P}\of{X}$. Wenn $\mG$ $\cap$-stabil ist, dann ist $\delta\of{\mG}$ eine $\sigma$-Algebra und es gilt $\sigma\of{\mG} = \delta\of{\mG}$.

    \begin{proof}
        Nach Satz~\ref{satz:dynkin-cap-stabil} ist $\delta\of{\mG}$ $\cap$-stabil und damit nach Lemma~\ref{lemma:dynkin-sigma-equiv} eine $\sigma$-Algebra. Damit gilt dann $\sigma\of{\mG} \subseteq \delta\of{\mG}$, da $\sigma\of{\mG}$ die kleinste $\sigma$-Algebra ist, die $\mG$ enthält. Außerdem ist $\delta\of{\mG}\subseteq \delta\of{\sigma\of{\mG}} = \sigma\of{\mG}$.
    \end{proof}
\end{korollar}

\newpage


    \section{[*] Eindeutigkeit von Maßen und erste Eigenschaften des Lebesgue-Maßes}
    \imaginarysubsection{Eindeutigkeit von Maßen}

    \begin{satz}[Eindeutigkeitssatz]
        \marginnote{[04. Nov]}
        \label{satz:eindeutigkeitssatz}
        Sei $(X, \mA)$ ein beliebiger Messraum und $\mA = \sigma\of{\mE}$ für $\mE \subseteq\mP\of{X}$. Ferner seien $\mu, \nu$ Maße auf $\mA$ mit
        \begin{enumerate}[label=(\alph*)]
            \item $\mE$ ist $\cap$-stabil
            \item Es gibt Mengen $G_n \in \mE$ mit $G_n \nearrow X$ ($X = \bigcup_{n\in\N} G_n$) mit $\mu\of{G_n}, \nu\of{G_n} < \infty~\forall n\in\N$
        \end{enumerate}
        Dann gilt: Aus $\mu\of{A} = \nu\of{A}~\forall A\in\mE$ folgt $\mu = \nu$ auf $\mA$. Das heißt unter den obigen Voraussetzungen wird ein Maß eindeutig durch seine Werte auf dem Erzeuger definiert.

        \begin{proof}
            Da $\mE$ $\cap$-stabil ist, folgt nach Korollar~\ref{korollar:dynkin-sigma}, dass $\delta\of{\mE} = \sigma\of{\mE} = \mA$. Wir halten $n\in\N$ fest und betrachten
            \begin{align*}
                \mD_n &\coloneqq \set{A\in \mA: \mu\of{G_n \cap A} = \nu\of{G_n \cap A}}
            \end{align*}
            $\mD_n$ ist ein Dynkinsystem:
            \begin{enumerate}[label=($\text{D}_{\arabic*}$)]
                \item Folgt direkt.
                \item Sei $A\in\mD_n$. Dann ist
                \begin{align*}
                    \mu\of{G_n \cap \comp{A}} &= \mu\of{G_n \setminus A} = \mu\of{G_n \setminus\pair{A \cap G_n}}\\
                    &= \mu\of{G_n} - \mu\of{A \cap G_n}\\
                    &= \nu\of{G_n} - \nu\of{A \cap G_n}\\
                    &= \nu\of{G_n \cap \comp{A}}\\
                    \impl \comp{A} &\in \mD_n
                \end{align*}
                \item Sei $(A_m)_m \subseteq \mD_n$ eine Folge paarweise disjunkter Mengen. Dann gilt
                \begin{align*}
                    \mu\of{\pair{\bigsqcup_{m\in\N} A_m} \cap G_n} &= \mu\of{\bigsqcup_{m\in\N} \pair{A_m \cap G_n}} = \sum_{m\in\N}^{} \mu\of{A_m \cap G_n}\\
                    &= \sum_{m\in\N}^{} \nu\of{A_m \cap G_n} = \nu\of{\pair{\bigsqcup_{m\in\N} A_m} \cap G_n}\\
                    \impl \bigsqcup_{m\in\N} A_m &\in \D_n
                \end{align*}
            \end{enumerate}
            Nach Konstruktion von $\mD_n$ gilt $\mD_n \subseteq\mA$. Andererseits ist $\mE\subseteq\mD_n$. Sei $A \in\mE$ und $A \cap G_n \in \mE$, da $\mE$ $\cap$-stabil. Nach Voraussetzung gilt $\nu\of{A \cap G_n} = \mu\of{A \cap G_n}$, also folgt $A \in\mD_n$.\\
            Da $\mE \subseteq\mD_n \impl \sigma\of{\mE} = \delta\of{\mE} \subseteq \delta\of{\mE} = \mD_n$. Damit gilt $\sigma\of{\mE} \subseteq  \mD_n$. Das heißt $\forall A\in\sigma\of{\mE}$ folgt $\mu\of{A \cap G_n} = \nu\of{A \cap G_n}$.\\
            Für $A\in \sigma\of{\mE}$ definieren wir eine aufsteigende Folge $A_n \coloneqq A \cap G_n \nearrow A$. Da $\mu, \nu$ Maße sind, sind sie von unten stetig. Das heißt
            \begin{align*}
                \mu\of{A} &= \lim_{n\toinf} \mu\of{A_n} = \lim_{n\toinf} \mu\of{A \cap G_n}\\
                &= \lim_{n\toinf} \nu\of{A \cap G_n} = \nu\of{A}\qedhere
            \end{align*}
        \end{proof}
    \end{satz}

    \begin{bemerkung}[Ausschöpfende Folgen]
        Wir nennen $(G_n)_n$ im Sinne von Satz~\ref{satz:eindeutigkeitssatz} eine ausschöpfende Folge. Wir nennen ein Maß $\mu$ auf $\mA$ $\sigma$-endlich, wenn es eine Folge $(G_n)_n \subseteq \mA$ gibt mit $G_n \nearrow X$ und $\mu\of{G_n} < \infty~\forall n\in\N$.
    \end{bemerkung}

    \begin{satz}[Eigenschaften der Borelmengen]
        In Definition~\ref{definition:sigma-borel} hatten wir $\mB\of{\R^d} \coloneqq \sigma\of{\mO_d}$, wobei $\mO$ das System offener Teilmengen im $\R^d$ war. Wir definieren nun
        \begin{enumerate}[label=-]
            \item $\mA_d$: System der abgeschlossenen Teilmengen im $\R^d$
            \item $\mK_d$: System der kompakten Teilmengen im $\R^d$
        \end{enumerate}
        Dann gilt $\sigma\of{\mK_d} = \sigma\of{\mA_d} = \sigma\of{\mO_d} = \mB\of{\R^d}$.
        \begin{proof}
            \textsc{Schritt 1}: $\sigma\of{\mA_d} = \sigma\of{\mO_d}$ ist klar, da $\sigma$-Algebren stabil unter Komplementbildung sind.\\
            \textsc{Schritt 2}: Es gilt $\mK_d \subseteq \mA_d \impl \sigma\of{\mK_d} \subseteq \sigma\of{\mA_d}$.\\
            \textsc{Schritt 3}: Für $n\in\N$ definieren wir $K_n \coloneqq \set{\abs{x} < n}$. Sei $A \in \mA_d$, dann ist $A\cap K_n$ kompakt und
            \begin{align*}
                \bigcup_{n\in\N} K_n &= \R^d\\
                A &= \bigcup_{n\in\N} \pair{A \cap K_n} \in \sigma\of{\mK_d}\\
                \impl \mA &\subseteq \sigma\of{\mK_d}\\
                \impl \sigma\of{\mA_d} &\subseteq \sigma\of{\sigma\of{\mK_d}} = \sigma\of{\mK_d}\\
                \impl \sigma\of{\mK_d} &= \sigma\of{\mA_d} = \sigma\of{\mO_d}\qedhere
            \end{align*}
        \end{proof}
    \end{satz}

    Im Folgenden nehmen wir an, dass das Lebesgue-Maß $\lambda^d$ auf $\mB\of{\R^d}$ existiert. Wir werden das später noch beweisen, aber entwickeln das Maß nun nach unserem geometrischen Verständnis unter der Annahme, dass es existiert (das tut es) und untersuchen erste Eigenschaften:

    \begin{beobachtung}
        Wir betrachten den Fall $d=1$ und ein halboffenes Intervall $I\coloneqq \linterv{a,b}$. Dann muss gelten $\lambda^1\of{I} = b-a$. Wir betrachten allgemeine $d$ mit $a,b\in\R^d$ wobei $a\leq b$ (das heißt $a_j \leq b_j$). Dann sei
        \begin{align*}
            \linterv{a,b} &\coloneqq \set{x\in\R^d: a_j \leq x_j \leq b_j~\forall j\in\set{1,\ldots, d}}\\
            \intertext{und wir definieren nach unserem geometrischen Verständnis}
            \lambda^d\of{\linterv{a,b}} &\coloneqq \prod_{j=1}^{d} \pair{b_j - a_j}
        \end{align*}
    \end{beobachtung}

    \begin{definition}
        Es sei $J^d \coloneqq \set{\linterv{a,b}: a,b\in\R^d,~a\leq b}$ das Mengensystem der halboffenen Intervalle im $\R^d$.
    \end{definition}

    \begin{bemerkung}[Translationsinvarianz des Lebesgue-Maß]
        Es sei $c\in\R^d$ und wir definieren eine Translation $T_c\of{x} \coloneqq x+c$ mit inverser Funktion $T_c^{-1}$. Dann gilt für ein halboffenes Intervall $I \coloneqq \linterv{a,b}$
        \begin{align*}
            \lambda^d\of{T_c^{-1}\of{I}} &= \lambda^d\of{\linterv{a-c, b-c}}\\
            &= \prod_{j=1}^{d} \pair{b_j - c_j - \pair{a_j - c_j}}\\
            &= \prod_{j=1}^{d} \pair{b_j - a_j} = \lambda^d\of{I}
        \end{align*}
        Das heißt auf $J^d$ ist $\lambda^d$ invariant unter Translation.
    \end{bemerkung}

    \begin{lemma}
        Sei $B \in\mB^d$ eine Borelmenge und $c\in\R^d$. Dann ist $B + c \coloneqq\set{b+c: b\in B} \in\mB^d$.
        \begin{proof}
            Sei $c\in\R^d$ fest. \textsc{Schritt 1}: Wir wenden das \anf{Wünsch-dir-was}-Vorgehen an und definieren
            \begin{align*}
                \mA \coloneqq \set{A \in \mB^d: A + c \in \mB^d}
            \end{align*}
            Dann ist $\mA$ eine $\sigma$-Algebra (Übung).\\
            \textsc{Schritt 2}: $\mO_d$ ist translationsinvariant. Das heißt $\mO_d \subseteq \mA \impl \mB^d = \sigma\of{\mO_d} \subseteq \sigma\of{\mA} = \mA$. Das heißt $\mB^d \subseteq \mA$. Damit sind die Borelmengen translationsinvariant.
        \end{proof}
    \end{lemma}

    \marginnote{[8. Nov]} (fehlt)
    \newpage

    \begin{satz}
        \marginnote{[18. Nov]}
        Es gilt
        \begin{align*}
            T\of{\lambda^d} &= \lambda^d\quad\forall T\in\bew\of{\R^d}
        \end{align*}
        \begin{proof}
            \textsc{Schritt 1}: Sei $T\in\bew\of{\R^d}: T\of{0} = 0$. Das heißt $T$ ist linear. Dann definieren wir eine Translation
            \begin{align*}
                T_a\of{x} &\coloneqq x + a\\
                \impl \pair{T_a \circ T}\of{x} &= T\of{x} + a\\
                &= T\of{x} + T\of{b}\tag{$b\coloneqq T^{-1}\of{a}$}\\
                &= T\of{x+b} = \pair{T\circ T_b}\of{x}\\
                \impl T_a \circ T &= T \circ T_b
                \intertext{Wir definieren}
                \mu &\coloneqq T\of{\lambda^d} &= \lambda^d \circ T^{-1}\\
                T_a\of{\mu} &= T_a\of{T\of{\lambda^d}} = T_a \circ T\of{\lambda^d}\\
                &= T \circ T_b\of{\lambda^d} = T\of{T_b\of{\lambda_d}}\\
                &= T\of{\lambda^d} = \mu
            \end{align*}
            Damit ist $\mu$ invariant unter Translation. Das heißt nach dem Eindeutigkeitssatz, dass $\mu = \alpha\lambda^d$.\\
            Frage: Warum ist $\alpha = 1$?\\
            Wir betrachten die abgeschlossene Einheitskugel $B\coloneqq\set{x\in\R^d: \abs{x}\leq 1}$. Dann folgt
            \begin{align*}
                T^{-1}\of{B} &= B\\
                \impl \lambda^d\of{B} &= \lambda^{d}\of{T^{-1}\of{B}} = \mu\of{B} = \alpha \lambda^d\of{B}\\
                \impl \alpha &= 1 \text{ falls } 0 < \lambda^d\of{B} <\infty\\
                \impl T\of{\lambda^d} &= \lambda^d\quad\forall T\in\bew\of{\R^d}: T\of{0} = 0
            \end{align*}
            \textsc{Schritt 2}: Sei $T\in\bew\of{\R^d}$ beliebig und wir setzen $c\coloneqq T\of{0}$ und $S \coloneqq T_{-c} \circ T \in\bew\of{\R^d}$. Dann gilt
            \begin{align*}
                S\of{0} &= T_{-c} \of{T\of{0}} = T_{-c}\of{c} = 0\\
                \impl S\of{\lambda^d} &= \lambda^d\text{ nach \textsc{Schritt 1}}
                \intertext{Wir wollen das aber noch für allgemeine Bewegungen zeigen. Es gilt}
                T &= T_c \circ S\\
                \impl T\of{\lambda^d} &= T_c\of{S\of{\lambda_d}} = T_c\of{\lambda^d} = \lambda^d
            \end{align*}
            nach \textsc{Schritt 1}.\qedhere
        \end{proof}
    \end{satz}

    \begin{beispiel}[Lebesgue-Maß von einem Punkt]
        Sei $x\in\R^d$. Was ist dann $\lambda^d\of{\set{x}}$?. Wir definieren für $\varepsilon > 0$
        \begin{align*}
            J &\coloneqq \linterv{x, x+\varepsilon}\\
            \impl x&\in J\\
            \impl \lambda^d\of{\set{x}} &\leq \lambda^d\of{J} = \varepsilon^d \to 0\\
            \impl \lambda^d\of{\set{x}} &= 0\quad\forall x\in\R^d
        \end{align*}
        Damit ist auch das Lebesgue-Maß von einer Menge von abzählbar vielen Punkten $A \coloneqq \bigcup_{n\in\N} \set{x_n}$ null, da
        \begin{align*}
            \lambda^d\of{A} &\leq \sum_{n\in\N}^{} \lambda^d\of{\set{x_n}} = 0\\
            \impl \lambda^d\of{\Q^d} &= 0
        \end{align*}
    \end{beispiel}

    \begin{beispiel}[Lebesgue-Maß einer Hyperebene]
        Es sei $j\in\set{1, \ldots, d}$ und wir definieren eine Hyperebene $H_j\coloneqq\set{x\in\R^d: x_j = 0}$. Was ist dann $\lambda^d\of{H_j}$? Wir definieren
        \begin{align*}
            J_n \coloneqq \set{x\in\R^d: -n \leq x_k \leq n, k\neq j\land -\frac{\varepsilon}{(2n)^{d-1} 2\cdot2^n} \leq x_j \leq \frac{\varepsilon}{2\cdot 2^{n}(2n)^{d-1}}}
        \end{align*}
        Damit ist
        \begin{align*}
            H_j &\subseteq \bigcup_{n\in\N} J_n\\
            \lambda^d\of{H_j} &\leq \lambda^d\of{\bigcup_{n\in\N} J_n} \leq \sum_{n=1}^{\infty} \lambda^d\of{J_n}\\
            &= \sum_{n=1}^{\infty} (2n)^{d-1}\cdot \frac{2\varepsilon}{2(2n)^{d-1}2^n}\\
            &= \varepsilon\sum_{n=1}^{\infty} 2^{-n} = \varepsilon\quad\forall\varepsilon > 0\\
            \impl \lambda^d\of{H_j} &= 0
        \end{align*}
        Dieses Result gilt auch für allgemeine Hyperebenen, da wir eine Bewegung finden, die diese auf eine Hyperebene der Form $H_j$ abbildet.
    \end{beispiel}

    \newpage


    \section{[*] Existenz von Maßen}
    \imaginarysubsection{Existenz von Maßen}
    \thispagestyle{pagenumberonly}

    \begin{definition}[Halbring]
        Eine Familie $\mS \subseteq\mP\of{X}$ heißt Halbring, falls
        \begin{enumerate}[label=($\text{S}_{\arabic*}$)]
            \item $\emptyset\in\mS$
            \item $A, B\in\mS\impl A \cap B \in\mS$
            \item $A, B\in\mS$ existieren endlich viele disjunkte Mengen $S_1, \ldots, S_M \in \mS$ mit $A \setminus B = \bigsqcup_{j=1}^{M} S_j$
        \end{enumerate}
    \end{definition}

    \begin{satz}[Nach Carathéodory] % Satz 1
        \label{satz:caratheodory}
        Sei $\mS\subseteq\mP\of{X}$ ein Halbring und $\mu: \mS \to \interv{0,\infty}$ ein Prämaß. Dann existiert (mindestens) eine Fortsetzung von $\mu$ zu einem Maß $\mu$ auf $\sigma\of{\mS}$.\\
        Falls $\mu$ $\sigma$-endlich auf $\mS$ ist, dann ist die Fortsetzung eindeutig.
    \end{satz}

    \begin{definition}[Äußere Maße]
        \marginnote{[25. Nov]}
        Eine Abbildung $\mu^{\ast}: \mP\of{X} \to \interv{0,\infty}$ ist ein äußeres Maß, falls
        \begin{enumerate}[label=(\roman*)]
            \item $\mu^{\ast}\of{\emptyset} = 0$\hfill (Normierung)
            \item $A \subseteq B \impl \mu^{\ast}\of{A} \leq \mu^{\ast}\of{B}$\hfill (Monotonie)
            \item $\mu^{\ast}\of{\bigcup_{n\in\N} A_n} \leq \sum_{n\in\N}^{} \mu^{\ast}\of{A_n}$\hfill ($\sigma$-Subadditivität)
        \end{enumerate}
    \end{definition}

    \begin{konstruktion}
        \label{konstruktion:prem}
        Zu einem Prämaß $\mu$ auf $\mS$ gibt es ein äußeres Maß.\\
        Nehme $A\subseteq X$. Wir bepflastern $A$ mit Mengen aus $\mS$:
        \begin{align*}
            \mC\of{A} &\coloneqq \set{(S_n)_{n\in\N}: S_n \in\mS~\forall n\in N \land A \subseteq \bigcup_{n\in\N} S_n}\tag{Cover}
            \intertext{$\mC\of{A} = \emptyset$ ist dabei möglich. Gegeben ein Prämaß $\mu: \mS\to\interv{0,\infty}$ definieren wir nun}
            \mu^{\ast}\of{A} &\coloneqq \begin{cases}
                                            \inf \set{\sum_{n\in\N}^{} \mu\of{S_n}: (S_n)_n \in\mC\of{A}} &\text{ falls } \mC\of{A} \neq \emptyset\\
                                            +\infty &\text{ falls } \mC\of{A} = \emptyset
            \end{cases}
        \end{align*}
    \end{konstruktion}

    \begin{lemma}
        Das in Konstruktion~\ref{konstruktion:prem} definierte $\mu^{\ast}$ ist ein äußeres Maß.
        \begin{proof}
            \theoremescape
            \begin{enumerate}[label=(\roman*)]
                \item Wir nehmen $S_n = \emptyset$. Dann gilt $\mu^{\ast}\of{\emptyset} = 0$
                \item Sei $A\subseteq B$. Angenommen $\mC\of{B} \neq \emptyset \impl \mC\of{B} \subseteq \mC\of{A}$
                \begin{align*}
                    \mu^{\ast}\of{B} &= \inf\set{\sum_{n\in\N}^{} \mu\of{S_n}: (S_n)_n \in\mC\of{B}}\\
                    &\geq \inf\set{\sum_{n\in\N}^{} \mu\of{S_n}: (S_n)_n \in\mC\of{B}}\\
                    &= \mu^{\ast}\of{A}
                \end{align*}
                \item Es gilt $\mu^{\ast}\of{A_n} = \inf \set{\sum_{n\in\N}^{} \mu\of{S_n}: (S_n)_n \in\mC\of{A}}$. Für jedes $n\in\N$ existiert ein $(S_{n,m})_m \in\mC\of{A_n}$ mit
                \begin{align*}
                    \sum_{m\in\N}^{} \mu\of{S_n,m} &\leq \mu^{\ast}\of{A_n} + \varepsilon \Sigma^n\quad\forall\varepsilon > 0\\
                    \impl \mu^{\ast}\of{A} &\leq \sum_{n,m\in\N}^{} \mu\of{S_{n,m}}\\
                    &= \sum_{n\in\N}^{} \of{\sum_{m\in\N}^{} \mu\of{S_{n,m}}}\\
                    &\leq \sum_{n\in\N}^{} \pair{\mu^{\ast}\of{A_n} + \varepsilon\Sigma^n}\\
                    &= \sum_{m\in\N}^{} \mu^{\ast}\of{A_m} + \varepsilon\quad\forall\varepsilon > 0\qedhere
                \end{align*}
            \end{enumerate}
        \end{proof}
    \end{lemma}

    \begin{konstruktion}[Prämaß auf erzeugtem Ring]
        Sei $\mu$ ein Prämaß auf einem Halbring $\mS$. Wir wollen $\mu$ zu einem Prämaß auf dem erzeugten Mengenring forsetzen. Dazu gehen wir wie folgt vor.\\
        Wir setzen $\mS_{\cup} \coloneqq \set{S_1 \sqcup S_2 \sqcup \dots \sqcup S_n: S_j\in\mS}$ und definieren
        \begin{align*}
            \overline{\mu}\of{S_1 \sqcup \dots \sqcup S_n} &= \sum_{j=1}^{n} \mu\of{S_j}
        \end{align*}
        Wir zeigen die Wohldefiniertheit von $\mu$. Angenommen $S_1 \sqcup \dots \sqcup S_n = T_1 \sqcup \dots \sqcup T_n$ und $S_j, T_j\in \mS$. Dann gilt
        \begin{align*}
            S_j &= S_j \cap \pair{\bigsqcup_{k=1}^n S_k} = S_j \cap \pair{\bigsqcup_{k=1}^n T_k}\\
            &= \bigsqcup_{k=1}^n \pair{S_j \cap T_k}
            \intertext{Analog ist auch}
            T_k &= \bigsqcup_{j=1}^n \pair{T_k \cap S_j}\\
            \impl S_1 \sqcup \dots \sqcup S_M &= \bigsqcup_{j=1}^M \bigsqcup_{k=1}^N \pair{S_j \cap T_k}\\
            \impl \overline{\mu}\of{S_1 \sqcup \dots \sqcup S_n} &= \overline{\mu}\of{\bigsqcup_{j=1}^M \bigsqcup_{k=1}^M\pair{S_j \cap T_k}}\\
            \sum_{n=1}^{M} \sum_{k=1}^{N} \mu\of{S_j \cap T_k} &= \sum_{k=1}^{N} \mu\of{T_k} = \overline{\mu}\of{T_1 \sqcup \dots \sqcup T_N}
        \end{align*}
        Das heißt $\overline{\mu}$ ist wohldefiniert.\\[.5\baselineskip]
        Frage: Wie verhält sich $\mS_{\cup}$ unter allgemeinen (endlichen) Vereinigungen und Schnitten? Wir betrachten $S, T\in \mS_{\cup}$ mit
        \begin{align*}
            S\cap T &= \pair{S_1 \sqcup \dots \sqcup S_M} \cap \pair{T_1 \sqcup \dots \sqcup T_N}\\
            &= \bigsqcup_{j=1}^M \bigsqcup_{k=1}^N \pair{S_j \cap T_k} \in \mS_{\cup}
        \end{align*}
        Das heißt $\mS_{\cup}$ ist stabil unter (endlichen) Schnitten.
        \begin{align*}
            S\setminus T &= \pair{S_1 \sqcup \dots \sqcup S_M} \setminus\pair{T_1 \sqcup \dots \sqcup T_N}\\
            \impl S \cap \comp{T} &= \pair{S_1 \sqcup \dots \sqcup S_M} \cap \bigcap_{n=1}^N \comp{T_k}\\
            &= \bigsqcup_{j=1}^M \pair{S_j \cap \bigcap_{k=1}^N \comp{T_k}}\\
            &= \bigsqcup_{j=1}^M \bigcap_{k=1}^N \pair{S_j \cap \comp{T_k}} \in \mS_{\cup}
        \end{align*}
        Das heißt für $S, T \in \mS_{\cup}$ ist auch $S\setminus T\in\mS_{\cup}$. Außerdem können wir schreiben
        \begin{align*}
            S\cup T &= \pair{S\setminus T} \sqcup \pair{S\cap T} \sqcup \pair{T \setminus S} \in \mS_{\cup}
        \end{align*}
        Das heißt $\mS_{\cup}$ ist ein Mengenring und $\overline{\mu}$ ist eine Fortsetzung von $\mu$ auf $\mS_{\cup}$.
        \begin{align*}
            \impl \overline{\mu}\of{T \cup S} &= \overline{\mu}\of{S \setminus T} + \overline{\mu}\of{S \cap T} + \overline{\mu}\of{T \setminus S}
        \end{align*}
        Das heißt $\overline{\mu}$ ist definiert für endlich viele Vereinigungen von Mengen aus $\mS_{\cup}$.\\[.5\baselineskip]
        Behauptung: $\overline{\mu}$ ist ein Prämaß auf $\mS_{\cup}$. Also ist zu zeigen, dass $\overline{\mu}$ $\sigma$-additiv auf $\mS_{\cup}$ ist. Wir nehmen $(T_k)_k \subseteq \mS_{\cup}$
        \begin{align*}
            T &\coloneqq \bigsqcup_{k\in\N} T_k \in \mS_{\cup}
            \intertext{Zu zeigen:}
            \overline{\mu}\of{T} &= \sum_{k\in\N}^{} \overline{\mu}\of{T_k}
            \intertext{Nach Definition von $\mS_{\cup}$ gibt es $(S_n)_n \subseteq \mS$ und Indizes $0= i\of{0} \leq i\of{1} \leq i\of{2} \leq \dots$ mit}
            T_k &= S_{i\of{k-1} + 1} \sqcup \dots \sqcup S_{i\of{k}}\tag{$k\in\N$}\\
            T &= U_1 \sqcup \dots \sqcup U_L\\
            U_l &\coloneqq \bigsqcup_{i\in J_l} S_i
            \intertext{Indexmengen $J_1, \ldots, J_l \subseteq \N$ paarweise disjunkt und $J_1 \sqcup \ldots \sqcup J_L = \N$}
            \overline{\mu}\of{T} &= \overline{\mu}\of{U_1 \sqcup \ldots \sqcup U_L}\\
            &= \overline{\mu}\of{U_1} +\ldots + \overline{\mu}\of{U_l}\\
            &= \mu\of{U_1} + \ldots + \mu\of{U_l}\\
            &= \sum_{i\in J_1}^{} \mu\of{S_i} + \ldots + \sum_{i\in J_L}^{} \mu\of{S_i}\\
            &= \sum_{j=1}^{\infty} \mu\of{S_j} = \sum_{k=1}^{\infty} \overline{\mu}\of{T_k}
        \end{align*}
    \end{konstruktion}

    \begin{bemerkung}[Prämaß und äußeres Maß]
        \marginnote{[29. Nov]}
        Wenn wir ein Prämaß $\mu$ auf einem Halbring $\mS$ haben, dann gilt
        \begin{align*}
            \mu\of{A} &= \mu^{\ast}\of{A}\quad\forall A\in\mS
        \end{align*}
        wobei $\mu^{\ast}$ das in Konstruktion~\ref{konstruktion:prem} definierte äußere Maß ist. Das heißt $\mu^{\ast}$ ist eine Fortsetzung von $\mu$.
        \begin{proof}
            Sei $A\in\mS$ und $(S_j)_j \in\mC\of{A}$ Überdeckung von $A$. Dann gilt
            \begin{align*}
                \mu\of{A} &= \overline{\mu}\of{A} = \overline{\mu}\of{\pair{\bigcup_{j\in\N} S_j}\cap A}\\
                \intertext{Wegen der $\sigma$-Subadditivität von $\overline{\mu}$ gilt}
                &\leq \sum_{j=1}^{\infty} \overline{\mu}\of{S_j \cap A} \leq \sum_{j=1}^{\infty} \overline{\mu}\of{S_j}
                \intertext{Wir nehmen das Infimum auf beiden Seiten und erhalten}
                \impl\mu\of{A} &\leq \mu^{\ast}\of{A}\quad\forall A\in\mS
                \intertext{Wir wollen noch zeigen, dass $\mu\of{A} \geq \mu^{\ast}\of{A}$. Für ein $A\in\mS$ nehmen wir $(S_j)_j$ mit $S_1 \coloneqq A$, $S_2 = S_3 = \ldots = \emptyset$}
                \impl \mu^{\ast}\of{A} &\leq \mu\of{A}\quad\forall A\in\mS\\
                \impl \mu^{\ast}\of{A} &= \mu\of{A}\quad\forall A\in\mS\qedhere
            \end{align*}
        \end{proof}
    \end{bemerkung}

    \begin{definition}[Zerlegungsbedingung]
        Sei $\mu^{\ast}$ ein äußeres Maß auf $\mP\of{X}$. Dann sagen wir $A\subseteq X$ erfüllt die Zerlegungsbedingung, falls
        \begin{align*}
            \mu^{\ast}\of{B} &= \mu^{\ast}\of{B \cap A} + \mu^{\ast}\of{B \setminus A}\quad\forall B\subseteq X\numbereq{eq:zerleg}
        \end{align*}
        Außerdem definieren wir
        \begin{align*}
            \mA_{\ast} &\coloneqq \set{A \subseteq X: A\text{ erfüllt die Zerlegungsbedingung~\ref{eq:zerleg}}}
        \end{align*}
    \end{definition}

    \begin{bemerkung}
        Die Bedingung~\ref{eq:zerleg} ist äquivalent zu der Bedingung
        \begin{align*}
            \mu^{\ast}\of{B} &\geq \mu^{\ast}\of{B \cap A} + \mu^{\ast}\of{B \setminus A}\quad\forall B\subseteq X
        \end{align*}
        da die Ungleichung in die andere Richtung bereits durch die $\sigma$-Subadditivität von $\mu^{\ast}$ gegeben ist.
    \end{bemerkung}

    \begin{lemma}
        Sei $\mu^{\ast}$ das vom Prämaß $\mu$ auf $\mS$ erzeugte äußere Maß. Dann gilt
        \begin{align*}
            \mS \subseteq \mA_{\ast}
        \end{align*}
        \begin{proof}
            Sei $A \in\mS$. Wir wollen zeigen, dass
            \begin{align*}
                \mu^{\ast}\of{B} &= \mu^{\ast}\of{B \cap A} + \mu^{\ast}\of{B \setminus A}\quad\forall B\subseteq X
            \end{align*}
            O.B.d.A sei $\mC\of{B} \neq \emptyset$. Sei $(B_n)_n \in \mC\of{B}$, $B_n\in\mS$. Dann gilt
            \begin{align*}
                B_n &= \pair{B_n \cap A} \sqcup \pair{B_n \setminus A}\\
                \impl\mu\of{B_n} &= \overline{\mu}\of{B_n} = \overline{\mu}\of{\pair{B_n \cap A}} + \overline{\mu\of{B_n \setminus A}}\\
                \impl \sum_{n=1}^{\infty} \mu\of{B_n} &= \sum_{n=1}^{\infty} \pair{\mu\of{B_n \cap A}} + \sum_{n=1}^{\infty} \pair{\overline{\mu}\of{B_n \setminus A}}
                \intertext{Es gilt $\pair{B_n \cap A}_n \in\mC\of{B \cap A}$ und $\pair{B_n \setminus A}_n \in\mC\of{B\setminus A}$}
                &\geq \mu^{\ast}\of{B\cap A} + \overline{\mu}^{\ast}\of{B\setminus A}\\
                &= \mu^{\ast}\of{B \cap A} + \mu^{\ast}\of{B\setminus A}\\
                \impl \mu\of{B} &\geq \mu^{\ast}\of{B \cap A} + \mu^{\ast}\of{B \setminus A}\qedhere
            \end{align*}
        \end{proof}
    \end{lemma}
    \noindent Wir sind jetzt in der Lage, den Satz von Carathéodory zu beweisen.
    \begin{proof}[Beweis von Satz~\ref{satz:caratheodory}]
        \textsc{Schritt 1}: Wir zeigen, dass $\mA_{\ast}$ eine Algebra ist. Die Stabilität unter Komplementbildung und $\emptyset\in\mA_{\ast}$ zeigt sich leicht. Wir wollen also noch zeigen, dass $A_1 \cup A_2 \in \mA_{\ast}$. Das heißt es soll gelten
        \begin{align*}
            \mu^{\ast}\of{B} &= \mu^{\ast}\of{B\cap \pair{A_1 \cup A_2}} + \mu^{\ast}\of{B \setminus\pair{A_1 \cup A_2}}\\
            \intertext{Wir definieren}
            B_1 &\coloneqq \pair{A_1 \cap B} \setminus A_2\\
            B_2 &\coloneqq \pair{A_2 \cap B} \setminus A_1\\
            B_3 &\coloneqq B \cap A_1 \cap A_2\\
            B_4 = B \setminus \pair{A_1 \cup A_2}
            \intertext{Dann gilt}
            B \cap \pair{A_1 \cup A_2} &= \pair{B \cap A_1}\cup\pair{B\cap A_2}\\
            &= B_1 \cup B_2 \cup B_3\\
            B \setminus\pair{A_1 \cup A_2} &= B_4
            \intertext{Das heißt es ist zu zeigen, dass}
            \mu\of{B} &= \mu\of{B_1 \sqcup B_2 \sqcup B_3} + \mu\of{B_4}
            \intertext{$A_1$ erfüllt die Zerlegungsbedingung. Also}
            \mu^{\ast}\of{B} &= \mu^{\ast}\of{B \cap A_1} + \mu^{\ast}\of{B \setminus A_1}\\
            &= \mu^{\ast}\of{B_1 \cup B_3} + \mu^{\ast}\of{B_2 \cup B_4}
            \intertext{Verwende $A_2$, um $B_1 \cup B_3$ zu zerlegen}
            \mu^{\ast}\of{B_1 \cup B_3} &= \mu^{\ast}\of{\pair{B_1 \cup B_2} \cap A_2} + \mu^{\ast}\of{\pair{B_1 \cup B_2} \setminus A_2}\\
            &= \mu^{\ast}\of{B_1} + \mu^{\ast}\of{B_3}
            \intertext{Genauso zerlegen von $B_2 \cup B_4$ mittels $A_2$}
            \mu^{\ast}\of{B_2 \cup B_4} &= \mu^{\ast}\of{B_2} + \mu^{\ast}\of{B_4}\\
            \impl \mu^{\ast}\of{B} &= \mu^{\ast}\of{B_1} + \mu^{\ast}\of{B_2} + \mu^{\ast}\of{B_3} + \mu^{\ast}\of{B_4}
            \intertext{Machen dasselbe mit $\overline{B} \coloneqq B_1 \cup B_2 \cup B_3$}
            \impl \mu^{\ast}\of{B_1 \cup B_2\cup B_3} &= \mu^{\ast}\of{B_1} + \mu^{\ast}\of{B_2} + \mu^{\ast}\of{B_3}
        \end{align*}
        Es folgt dann $A_1 \cup A_2$ erfüllt die Zerlegungsbedingung und damit $A_1 \cup A_2 \in\mA_{\ast}$.\\
        \marginnote{[02. Dez]}
        \textsc{Schritt 2}: Wir zeigen, dass $\mu^{\ast}$ endlich additiv auf $\mA_{\ast}$ ist. Seien $A_1, A_2\in\mA_{\ast}$ mit $A_1 \cap A_2 = \emptyset$. Wir benutzen $A_1$, um $A_1 \sqcup A_2$ zu zerlegen. Es gilt
        \begin{align*}
            \mu^{\ast}\of{A_1 \sqcup A_2} &= \mu^{\ast}\of{\pair{A_1 \sqcup A_2} \setminus A_1} + \mu^{\ast}\of{\pair{A_1 \sqcup A_2} \setminus A_2}\\
            &= \mu^{\ast}\of{A_1} + \mu^{\ast}\of{A_2}
        \end{align*}
        \textsc{Schritt 3}: Wir zeigen, dass $\mA_{\ast}$ eine $\sigma$-Algebra und $\mu^{\ast}$ ein Maß auf $\mA_{\ast}$ ist. Sei $(A_j)_j \subseteq \mA_{\ast}$ und $A = \bigsqcup_{j\in\N} A_j$. Zu zeigen ist
        \begin{align*}
            \mu^{\ast}\of{B} &= \mu^{\ast}\of{B \cap A} + \mu^{\ast}\of{B\setminus A}\quad\forall B\subseteq X
        \end{align*}
        Dabei reicht $\geq$ zu zeigen, da die andere Ungleichung allgemein erfüllt ist. WIr definieren
        \begin{align*}
            F_n &\coloneqq \bigsqcup_{j=1}^n A_j \in\mA_{\ast}
            \intertext{mit $F_n\nearrow A$ und}
            B\setminus A &\subseteq B \setminus F_n\quad\forall n\in\N
            \intertext{Also wird $B$ von $F_n$ zerlegt}
            \mu^{\ast}\of{B} &= \mu^{\ast}\of{B \cap F_n} + \mu^{\ast}\of{B \setminus F_n}\\
            &= \mu^{\ast}\of{B\cap \pair{\bigsqcup_{j=1}^{n} A_j}} + \mu^{\ast}\of{B \cap \bigcap \comp{A_j}}\\
            \intertext{Wegen der endlichen Additivität von $\mu^{\ast}$ gilt}
            &= \sum_{j=1}^{n} \mu^{\ast}\of{B \cap A_j} + \mu^{\ast}\of{B \cap \bigcap \comp{A_j}}\\[.5\baselineskip]
            B \cap A &= B \cap \pair{\bigsqcup_{j\in\N} A_j} = \bigsqcup_{j\in\N}\pair{B \cap A_j}\\
            \impl \mu^{\ast}\of{B \cap A} &= \mu^{\ast}\of{\bigsqcup_{j\in\N} \pair{B \cap A_j}}\\
            \annot[{&}]{\leq}{$\sigma$-Subadd.} \sum_{j\in\N}^{} \mu^{\ast}\of{B \cap A_j}\\[.5\baselineskip]
            \mu^{\ast}\of{B \cap A} + \mu^{\ast}\of{B\setminus A} &\leq \sum_{j=1}^{n} \mu^{\ast}\of{B \cap A_j} + \mu^{\ast}\of{B\setminus A}\\
            &= \lim_{n\toinf} \sum_{j=1}^{n} \pair{\mu^{\ast}\of{B \cap A_j} + \mu^{\ast}\of{B \setminus A}}
            \intertext{Aber wir wissen $B\setminus A \subseteq B \setminus F_n$. Damit folgt}
            &\leq \lim_{n\toinf} \sum_{j=1}^{n} \pair{\mu^{\ast}\of{B \cap A_j} + \mu^{\ast}\of{B \setminus F_n}}\\
            &= \lim_{n\toinf} \pair{\mu^{\ast} \of{B\cap F_n} + \mu^{\ast}\of{B \setminus F_n}} = \mu\of{B}
        \end{align*}
        Das heißt $A$ erfüllt die Zerlegungsbedingung und damit $A = \bigsqcup_{j\in\N} A_j \in\mA_{\ast}$. Das heißt $A_{\ast}$ ist ein Dynkinsystem. Wir wollen Satz~\ref{lemma:dynkin-sigma-equiv} anwenden und müssen daher noch zeigen, dass $A_{\ast}$ $\cap$-stabil ist. Es gilt
        \begin{align*}
            A_1 \cap A_2 &= \comp{\pair{\comp{A_1} \cup \comp{A_2}}} \in\mA_{\ast}
        \end{align*}
        für $A_1, A_2\in\mA_{\ast}$. Damit ist $A_{\ast}$ nach Lemma~\ref{lemma:dynkin-sigma-equiv} eine $\sigma$-Algebra.\\[\baselineskip]
        \textsc{Schritt 4}: Wir zeigen, dass $\mu^{\ast}$ ein Maß auf $\mA_{\ast}$ ist. Sei
        \begin{align*}
            B\coloneqq A = \bigsqcup_{j\in\N} A_j \in\mA
            \intertext{Wir zerlegen $B$ mittels $A$}
            \impl \mu^{\ast}\of{B} &\geq \sum_{j=1}^{n} \mu^{\ast}\of{B \cap A_j} + \mu^{\ast}\of{B\setminus A}\\
            \impl \mu^{\ast}\of{A} &\geq \sum_{j=1}^{n} \mu^{\ast}\of{A_j}
            \intertext{Durch die $\sigma$-Subadditivität haben wir auch die Ungleichung in die andere Richtung. Also folgt}
            \mu^{\ast}\of{A} &= \mu^{\ast}\of{\bigsqcup_{j\in\N} A_j} = \sum_{j\in\N}^{} \mu^{\ast}\of{A_j}\qedhere
        \end{align*}
    \end{proof}
    \noindent Um den Satz von Carathéodory auf $\lambda^d$ anwenden zu können, brauchen wir noch
    \begin{enumerate}[label=-]
        \item $\mJ^d \coloneqq$ \textit{Mengensystem der (rechts-)halboffenen Intervalle im $\R^d$} ist ein Halbring
        \item $\lambda^d: \mJ^d \to \interv{0,\infty}$ ist ein Prämaß
    \end{enumerate}

    \begin{lemma}
        \label{lemma:halbring-dotprod}
        Seien $X_1, X_2$ Mengen und $\mS_1$ ein Halbring in $X_1$ sowie $\mS_2$ Halbring in $X_2$. Dann gilt $\mS_1 \times \mS_2$ ist ein Halbring in $X_1 \times X_2$.
        \begin{proof}
            \theoremescape
            \begin{enumerate}[label=($\text{S}_{\arabic*}$)]
                \item $\emptyset\in \mS_1 \times \mS_2$ folgt direkt aus $\emptyset \in \mS_1$ und $\emptyset\in\mS_2$
                \item Seien $J_1^{1}, J_1^2\in\mS_1$ und $J_2^{1}, J_2^2\in\mS_2$. Dann gilt
                \begin{align*}
                    \pair{J_1^1 \times J_2^1} \cap \pair{J_1^2 \times J_2^2} &= \pair{J_1^1 \cap J_1^2}\times\pair{J_2^1 \cap J_2^2} \in \mS_1 \times \mS_2
                \end{align*}
                \item (Übung)
            \end{enumerate}
        \end{proof}
    \end{lemma}

    \begin{satz}
        \begin{align*}
            \mJ^{d} &\coloneqq \set{\linterv{a,b}: a,b\in\R^d, a <b}
        \end{align*}
        ist ein Halbring.
        \begin{proof}
            Für eine Dimension (das heißt $d=1$) zeigen sich die Halbring-Eigenschaften von $\mJ^1$ direkt durch Fallunterscheidungen. Wir wollen also für höhere Dimensionen einfach Lemma~\ref{lemma:halbring-dotprod} induktiv anwenden. Wir definieren also
            \begin{align*}
                \mJ^{2}&\coloneqq \mJ^1 \times \mJ^1
                \intertext{Das ist nach dem Lemma ein Halbring}\\
                \impl \mJ^{3} &\coloneqq \mJ^2 \times \mJ^1\text{ ist ein Halbring}\\
                &\vdots\\
                \impl\mJ^{3} &\coloneqq \mJ^{d-1}\times \mJ^1\text{ ist ein Halbring}\qedhere
            \end{align*}
        \end{proof}
    \end{satz}

    \begin{lemma}
        Sei $\mu$ endlich-additiv auf einem Halbring $\mS$. Wir betrachten die folgenden Aussagen
        \begin{enumerate}[label=(\alph*)]
            \item $\mu$ ein Prämaß
            \item $A_n \in \mS \land A_n \nearrow A \in \mS \impl \lim_{n\toinf} \mu\of{A_n} = \mu\of{A}$
            \item $A_n\in\mS \land A_n \searrow A \in \mS \land \mu\of{A_1} < \infty \impl \lim_{n\toinf} \mu\of{A_n} = \mu\of{A}$
            \item $A_n \in\mS\land A_n \searrow \emptyset \land \mu\of{A_1} < \infty \impl \lim_{n\toinf} A_n = 0$
        \end{enumerate}
        Dann gilt (a) $\equivalent$ (b) $\impl$ (c) $\equivalent$ (d). Gilt ferner $\mu\of{A} < \infty~\forall A\in\mS$, dann ist auch (c) $\impl (b)$. In diesem Fall sind also alle Aussagen äquivalent und damit verwendbar, um zu prüfen, ob $\mu$ ein Prämaß ist.
        \begin{proof}
        (Später)
        \end{proof}
    \end{lemma}

    \begin{satz}
        Sei $\lambda^d$ für $A\in\mJ^d$ mit
        \begin{align*}
            A &= \times_{j=1}^d \linterv{a_j, b_j} = \linterv{a,b}
            \intertext{definiert als}
            \lambda^d\of{\linterv{a,b}} &\coloneqq \prod_{j=1}^{d} \pair{b_j - a_j}
        \end{align*}
        Dann ist $\lambda^d$ ein Prämaß auf $\mJ^d$.
        \begin{proof}
            Es gilt $\lambda^d\of{\emptyset} = \lambda^d\of{\linterv{a,a}} = 0$. ??
        \end{proof}
    \end{satz}

    \marginnote{[06. Dez]}

    Wir wollen Maße nun einfacher darstellen, als nur als Fortsetzung eines Prämaßes, was auf einem Halbring definiert ist. Dafür sei nun $\mu$ ein Borelmaß auf $\R^d$ mit $\mu\of{K} < \infty$ für alle kompakten Mengen $K\subseteq \R^d$.

    \begin{konstruktion}[Maße mit Funktionen identifizieren]
        Wir betrachten zunächst nur den Fall $d=1$. Wir wollen $\mu$ auf den halboffenen Intervallen durch eine Funktion $F: \R \to \R$ darstellen. Das heißt es soll gelten
        \begin{align*}
            \mu\of{\linterv{a,b}} &= F\of{b}- F\of{a}\quad\forall a,b\in\R
        \end{align*}
        Welche Eigenschaften hat dann die Funktion $F$?
        \begin{enumerate}
            \item $F$ ist monoton wachsend
            \begin{align*}
                F\of{b} - F\of{a} &= \mu\of{\linterv{a,b}}\geq 0 \quad\forall b\geq a
            \end{align*}
            \item $F$ ist ?seitig stetig. Wir betrachten eine Folge $x_n \to b$ ($a < b$) mit $x_n \leq x_{n+1}$. Dann gilt
            \begin{align*}
                \bigcup_{n\in\N} \linterv{a, x_n} &= \linterv{a,b}
                \intertext{Da $\mu$ ein Maß ist, ist es von unten stetig. Das heißt es gilt}
                F\of{x_n} - F\of{a} &= \mu\of{\linterv{a, x_n}} \nearrow \mu\of{\linterv{a, b}} = F\of{b} - F\of{a}\\
                \impl \lim_{x\nearrow b} F\of{x} &= F\of{b}
            \end{align*}
        \end{enumerate}
    \end{konstruktion}

    \begin{beispiel}[Dirac-Maß]

    \end{beispiel}

    \begin{satz}
        Jede wachsende Funktion $F: \R\to\R$ die von rechts stetig ist und (von links Grenzwerte hat) erzeugt ein Borelmaß $\mu_F$, sodass
        \begin{align*}
            \mu_F\of{\linterv{a,b}} &= F\of{b} - F\of{a}\quad\forall a,b\in\R: a\leq b
        \end{align*}
        Ferner: Sei $G: \R\to\R$ wachsend und von rechts stetig mit $\mu_G = \mu_F$. Dann folgt
        \begin{align*}
            \exists c\in\R\colon G\of{x}&= F\of{x}+ C\quad\forall x\in\R
        \end{align*}

        \begin{proof}
            Der letzte Teil zeigt sich durch nachrechnen direkt. Erster Teil: Siehe Literatur.
        \end{proof}
    \end{satz}

    \newpage


    \section{[*] Messbare Abbildungen und Bildmaße}

    \subsection{Messbare Abbildungen}
    \thispagestyle{pagenumberonly}

    \begin{definition}
        Seien $\pair{X, \mA}$ und $\pair{X', \mA'}$ Messräume und $T: X\to X'$ eine Funktion. Dann heißt $T$ $\mA-\mA'$-messbar, falls
        \begin{align*}
            T^{-1}\of{A'} \in \mA\quad\forall A'\in\mA'
        \end{align*}
    \end{definition}

    \begin{beispiel}
        \theoremescape
        \begin{enumerate}
            \item Konstante Funktionen sind messbar
            \item Wir betrachten $\pair{X, \mO}$, $\pair{X', \mO'}$ mit $\mO$ Mengensystem der offenen Mengen. Dann ist eine stetige Funktion $T$ $\mB\of{\mO} - \mB\of{\mO'}$-messbar, da die Urbilder offener Mengen unter stetigen Funktionen offen sind
        \end{enumerate}
    \end{beispiel}

    \begin{satz}
        Seien $\pair{X, \mA}$, $\pair{X', \mA'}$ Messräume und $\mE' \subseteq \mP\of{X'}$ Erzeuger von $\mA' = \sigma\of{\mE'}$. Dann ist $T$ genau dann $\mA$-$\mA'$-messbar, wenn
        \begin{align*}
            T^{-1}\of{E'} \in \mA\quad\forall E'\in\mE'
        \end{align*}
        \begin{proof}
            $\Sigma' \coloneqq \set{A' \subseteq X': T^{-1}\of{A'} \in \mA}$ ist eine $\sigma$-Algebra in $X'$ (Übung) mit $\mE' \subseteq \Sigma'$
            \begin{align*}
                \impl \mA' = \sigma\of{\mE'} &\subseteq \sigma\of{\Sigma'} = \Sigma'\\
                \impl \mA' &= \Sigma'\qedhere
            \end{align*}
        \end{proof}
    \end{satz}

    \begin{satz}
        \label{satz:verknuepfung-messbar}
        Seien $\pair{X_1, \mA_1}, \pair{X_2, \mA_2}, \pair{X_3, \mA_3}$ Messräume und $T_1: X_1 \to X_2$ $\mA_1$-$\mA_2$-messbar sowie $T_2: X_2 \to X_3$ $\mA_2 $- $\mA_3$-messbar. Dann ist $T_2 \circ T_1: X_1 \to X_3$ $\mA_1 $-$ \mA_3$-messbar.

        \begin{proof}
            Sei $A_3\in\mA_3$. Dann ist
            \begin{align*}
                \pair{T_2 \circ T_1}^{-1}\of{A_3} &= T_1^{-1}\underbrace{\of{T_2^{-1}\of{A_3}}}_{\in\mA_2}\in\mA_3\qedhere
            \end{align*}
        \end{proof}
    \end{satz}

    \begin{bemerkung}
        Sei $I$ eine Index-Menge und $\pair{X_j, \mA_j}$ ein Messraum für $j\in I$. Abbildung $T_j: X \to X_j$. Dann ist $\sigma\of{\bigcup_{j\in I} T_j^{-1}\of{\mA_j}}$ die kleinste $\sigma$-Algebra, auf $X$, für die alle Abbildungen $T_j$ messbar (also $\sigma\of{\bigcup_{j\in I} T_j^{-1}\of{\mA_j}}$-$\mA_j$-messbar) sind.
    \end{bemerkung}

    \begin{satz}
        $S: X_0 \to X$ ist $\mA_0$-$\mA$-messbar. Dann ist $\mA = \sigma\of{\bigcup_{j\in I} T^{-1}\of{\mA_j}}$ genau dann, wenn
        \begin{align*}
            T_j \circ S: X_0 \to X_j\text{ ist } \mA_0 \text{-} \mA_j\text{-messbar}
        \end{align*}
        \begin{proof}
            \anf{$\impl$} Folgt direkt aus Satz~\ref{satz:verknuepfung-messbar}.\\[.5\baselineskip]
            \anf{$\Leftarrow$} Nehmen $E \in \bigcup_{j\in I} T_j^{-1}\of{\mA_j}$. Dann ist $E \in T^{-1}_j\of{\mA_j}$ für ein $j\in I$. Dann ist $E = T_j^{-1}\of{\mA_j}$. Es folgt $S^{-1}\of{E} = S^{-1}\of{T_j^{-1}\of{\mA_j}} = \pair{T_j \circ S}^{-1}\of{\mA_j} \subseteq \mA_0$.\\
            $\mE \coloneqq \bigcup_{j\in I} T_j^{-1}\of{\mA_j}$ ist Erzeuger von $\mA \coloneqq \sigma\of{\mE}$. Damit folgt die Behauptung.
        \end{proof}
    \end{satz}

    \subsection{Bildmaße}

    \begin{notation}
        \marginnote{[09. Dez]}
        Seien $\pair{X, \mA}, \pair{X', \mA'}$ Messräume. Wir schreiben $T: \pair{X, \mA} \to \pair{X', \mA'}$ für eine Funktion $T: X \to X'$, die $\mA$-$\mA'$-messbar ist.
    \end{notation}

    \begin{konstruktion}
        \label{konstruktion:bildmass}
        Sei $\mu$ ein Maß auf $\pair{X, \mA}$ und $T: \pair{X, \mA} \to \pair{X', \mA'}$ eine Funktion. Wir nehmen $A' \in \mA'$ und definieren
        \begin{align*}
            \mu'\of{A'} &\coloneqq \mu\of{T^{-1}\of{A'}}
        \end{align*}
        Dann ist $\mu'$ ein Maß auf $\pair{X', \mA'}$.
    \end{konstruktion}

    \begin{definition}[Bildmaß]
        Im Sinne von Konstruktion~\ref{konstruktion:bildmass} nennen wir $\mu'$ das Bild von $\mu$ unter $T$. Das heißt wir schreiben
        \begin{align*}
            \mu' &\coloneqq T\of{\mu}
            \intertext{Es gilt}
            T\of{\mu}\of{A'} &\coloneqq \mu\of{T^{-1}\of{A'}}
        \end{align*}
    \end{definition}

    \begin{bemerkung}[Transitivität der Bildmaße]
        Ist $T_1: \pair{X_1, \mA_1} \to \pair{X_2, \mA_2},~T_2: \pair{X_2, \mA_2} \to \pair{X_3, \mA_3}$ und $\mu$ ein Maß auf $\pair{X_1, \mA_1}$. Dann gilt
        \begin{align*}
            \pair{T_2 \circ T_1}\of{\mu} &= T_2\of{T_1\of{\mu}}\\
            \pair{T_2 \circ T_1}^{-1} &= T_1^{-1}\circ T_2^{-1}\\
            \impl \pair{T_2 \circ T_1}^{-1}\of{A_3} &=T_1^{-1}\of{T_2^{-1}\of{A_3}}
        \end{align*}
    \end{bemerkung}

    \newpage


    \section{[*] Messbare numerische Funktionen}
    \imaginarysubsection{Messbare numerische Funktionen}
    \thispagestyle{pagenumberonly}

    \begin{definition}[$\overline{\R}$]
        Wir schreiben $\overline{\R} \coloneqq \R \cup \set{-\infty, \infty}$ für die Menge der reellen Zahlen einschließlich $\pm\infty$. Wir definieren die algebraischen Operationen wie folgt:
        \begin{align*}
            x + \infty &\coloneqq \infty\quad\forall x\in\R\\
            x - \infty &\coloneqq -\infty\quad\forall x\in\R\\
            \infty + \infty &\coloneqq \infty\\
            -\infty + \pair{-\infty} = -\infty -\infty &\coloneqq -\infty\\
            -\infty &< x < \infty\quad\forall x\in\R\\
            x\cdot\infty &\coloneqq\infty\quad\forall x > 0\\
            x\cdot\pair{-\infty} &\coloneqq -\infty\quad\forall x > 0\\
            x\cdot\infty &\coloneqq\pair{-\infty}\quad\forall x < 0\\
            x\cdot\pair{-\infty} &\coloneqq \infty\quad\forall x < 0\\
            0\cdot\pair{\pm \infty} &\coloneqq 0\\
            \frac{x}{\pm\infty} &\coloneqq 0\quad\forall x\in\R\\
            \infty\cdot\infty &= \infty\\
            \infty\cdot\pair{-\infty} &= -\infty
        \end{align*}
        Dabei bleiben Ausdrücke wie $\infty - \infty$ nicht definiert. Wir definieren offene Mengen im $\overline{\R}$ wie folgt: $U \subseteq \overline{\R}$ ist offen, wenn
        \begin{align*}
            \forall x\in U\cap\R\ex R>0\colon \pair{x-R, x+R} &\subseteq U
            \intertext{Außerdem muss für den Fall $\infty\in U$ zu sätzlich gelten}
            \exists R > 0\colon \rinterv{R, \infty} &\subseteq U\\
            \intertext{Analog muss für $-\infty\in U$ gelten}
            \exists L > 0\colon \linterv{-\infty, L} &\subseteq U
        \end{align*}
        Mit dieser Definition können wir die Borel-$\sigma$-Algebra auf $\overline{\R}$ definieren mit $\overline{\mB}^{1} = \mB\of{\overline{\R}} = \overline{\mB}$.\\
        Dann gilt für die Spur-$\sigma$-Algebra $\overline{\mB}^{1} \cap \R = \mB^1$.
    \end{definition}

    \begin{bemerkung}[Identitätsfunktion]
        Sei $\pair{X, \mA}$ ein Messraum und $A \in\mA$. Dann definieren wir die Identitätsfunktion (von $A$) mit
        \begin{align*}
            \mathbf{1}_A\of{x} &\coloneqq \begin{cases}
                                              1 &x\in A\\
                                              0 &x\not\in A
            \end{cases}
        \end{align*}
        Diese ist Borel-messbar. Außerdem gilt
        \begin{align*}
            A\subseteq B \impl \mathbf{1}_A &\leq \mathbf{1}_B\\
            \mathbf{1}_{\comp{A}} &= 1 - \mathbf{1}_A
        \end{align*}
    \end{bemerkung}

    \begin{definition}[Numerische Funktion]
        Eine numerische Funktion ist eine Funktion $f: X\to\overline{\R}$. Ist $\mA$ eine $\sigma$-Algebra in $X$, so heißt $f$ $\mA$-messbar, falls es $\mA$-$\overline{\mB}$-messbar ist. Das heißt
        \begin{align*}
            \forall B\in\overline{\mB}\colon f^{-1}\of{B} \in\mA
        \end{align*}
    \end{definition}

    \begin{satz}
        \label{satz:messbarkeit}
        Sei $\mA$ eine $\sigma$-Algebra auf $X$. Dann ist eine numerische Funktion $f: X\to\overline{\R}$ genau dann $\mA$-messbar, wenn
        \begin{align*}
            \set{x\in X: f\of{x} \geq \alpha} \in\mA\quad\forall \alpha\in\R
        \end{align*}
        \begin{proof}
            Wir setzen $\overline{\mE} \coloneqq \set{\interv{\alpha, \infty}: \alpha\in\R}$. Zu zeigen ist $\sigma\of{\overline{\mE}} = \overline{\mB}^1$
            \begin{enumerate}[label=(\arabic*)]
                \item Es gilt $\interv{\alpha, \infty} \in \overline{\mB}^1~\forall\alpha\in\R$. Das heißt
                \begin{align*}
                    \impl\Sigma &\coloneqq \sigma\of{\overline{\mE}} \subseteq \overline{\mB}^1\\
                    \linterv{\alpha, \beta} &= \interv{\alpha, \infty} \setminus \interv{\beta, \infty} \in\overline{\mB}^1\tag{$\alpha \leq \beta$}\\
                    \impl \linterv{\alpha, \beta} &\in \R\cap\overline{\mB}^1\\
                    \impl \mB^1 &\subseteq \R \cap \overline{B}^1 = \sigma\of{\R \cap \overline{\mB}^1}
                \end{align*}
                \item Es gilt
                \begin{align*}
                    \set{+\infty} &= \bigcap_{n\in\N} \interv{n, \infty} \in \sigma\of{\mE}\\
                    \set{-\infty} &= \bigcap_{n\in\N} \comp{\interv{-n, \infty}} \in \sigma\of{\mE}\\
                    \impl \forall \overline{G}\in\Sigma\colon \overline{G} \cap \R &= \overline{G} \cap \pair{\comp{\set{-\infty,\infty}}} = \Sigma\\
                    \impl \R \cap \Sigma &\coloneqq \Sigma\\
                    \impl \mB^1 &\subseteq \Sigma\\
                    \impl \overline{\mB}^1 &= \mB^1 \cup \set{0, \set{-\infty}, \set{\infty}, \linterv{-\infty, \infty}} \subseteq \Sigma
                \end{align*}
                Also ist $\overline{\mE}$ ein Erzeuger von $\overline{\mB}^1$ und es genügt die Maßeigenschaften auf dem Erzeuger zu haben.
            \end{enumerate}
        \end{proof}
    \end{satz}

    \begin{satz}
        Sei $\mA$ eine $\sigma$-Algebra auf $X$. Dann ist eine numerische Funktion $f: X\to\overline{\R}$ genau dann $\mA$-messbar, wenn eine der folgenden äquivalenten Bedingungen erfüllt ist:
        \begin{align*}
            &\set{x\in X: f\of{x} \geq \alpha} \in\mA\quad\forall \alpha\in\R\tag{1}\\
            \equivalent &\set{x\in X: f\of{x} > \alpha} \in\mA\quad\forall \alpha\in\R\tag{2}\\
            \equivalent &\set{x\in X: f\of{x} \leq \alpha} \in\mA\quad\forall \alpha\in\R\tag{3}\\
            \equivalent &\set{x\in X: f\of{x} < \alpha} \in\mA\quad\forall \alpha\in\R\tag{4}
        \end{align*}

        \begin{proof}
            Die Äquivalenz zu (1) folgt direkt aus Satz~\ref{satz:messbarkeit}. Warum sind die anderen Aussagen dazu äquivalent? (Übung).
        \end{proof}
    \end{satz}

    \begin{satz}
        \label{satz:messbarkeit-1}
        Für messbare Funktionen $f, g: X \to \overline{\R}$ und $\pair{X, \mA}$ ein Messraum folgt
        \begin{align*}
            \set{f < g}, \set{f \leq g}, \set{f = g}, \set{f \neq g} \in\mA
        \end{align*}
        \begin{proof}
            $\Q$ ist abzählbar. Dann ist
            \begin{align*}
                \set{f < g} &= \set{x\in X: f\of{x} < g\of{x}}\\
                &= \bigcup_{r\in\Q} \set{f < r} \cap \set{r < g} \in\mA\\
                \set{f \leq g} &= \comp{\set{f > g}}\\
                \set{f = g} &= \set{f\leq g} \cap \set{f\geq g} = \mA\\
                \set{f\neq g} &= \comp{\set{f = g}} \in \mA\qedhere
            \end{align*}
        \end{proof}
    \end{satz}

    \begin{satz}
        \marginnote{[13. Dez]}
        Seien $f, g: X \to \R$ $\mA$-messbar. Dann folgt $f\pm g$ (falls definiert) sowie $f\cdot g$ sind messbar.
        \begin{proof}
            \begin{align*}
                \set{f - g\geq \alpha} &= \set{f\geq \underbrace{g + \alpha}_{\eqqcolon h}}\\
                &= \set{f\geq h}
                \intertext{Das heißt es reicht aus zu zeigen, dass $\alpha + g$ (bzw. $h$) messbare Funktionen für ein festes $\alpha\in\R$ sind. $g$ ist messbar, das heißt}
                \set{g \leq \beta} &\in\mA\\
                h\geq\beta &\equivalent g + \alpha \geq \beta\equivalent g \leq \beta - \alpha\\
                \set{h\geq \beta} &= \set{g \leq \beta - \alpha}
                \intertext{Das ist messbar, da $g$ messbar ist und wir Satz~\ref{satz:messbarkeit-1} anwenden können. Damit ist $h$ und damit $f - g$ messbar.\endgraf\noindent Um zu zeigen, dass auch $f+g$ messbar ist, können wir einfach zeigen, dass $-g$ messbar ist. Es gilt}
                \set{-g \geq \gamma} &= \set{g\geq -\gamma}
                \intertext{Damit ist $-g$, also auch $f - (-g) = f + g$ messbar.\endgraf\noindent Wir zeigen noch die Messbarkeit von $f\cdot g$. Annahme: $f$ und $g$ sind reellwertig}
                \pair{f+g}^2 - \pair{f-g}^2 &= 4fg\\
                \impl fg &= \frac{1}{4}\pair{\pair{f+g}^2 - \pair{f-g}^2}
                \intertext{Das heißt es reicht aus, zu zeigen, dass die Messbarkeit einer Funktion $h$ auch die Messbarkeit von $h^2$ impliziert}
                \set{h^2 \geq \beta} &= X\quad\text{ falls } \beta \leq 0\\
                \set{h^2\geq \beta} &= \set{h\geq \sqrt{\beta}} \cup \set{h\leq - \sqrt{\beta}} \in \mA\quad\text{ falls } \beta > 0
                \intertext{Damit haben wir die Messbarkeit von $h^2$ gezeigt. Wir wollen nun noch die Annahme loswerden, dass $f$ und $g$ reellwertig sind. Wir definieren}
                X_1 &\coloneqq \set{fg = +\infty} \cup \pair{\set{f > 0} \cap \set{g = \infty} \sqcup \pair{\set{f < 0} \cap \set{g = -\infty}}}\\
                &~~\cup \pair{\set{f = \infty} \cap \set{g\geq 0} \cup \set{f = \infty} \cap \set{g < 0}} (?)\\
                X_2 &\coloneqq \set{fg = -\infty}\\
                X_3 &\coloneqq \set{fg = 0} = \set{f = 0} \cup \set{g = 0}\\
                X_4 &\coloneqq \comp{\pair{X_1 \cup X_2 \cup X_3}}\\
                \impl f, g: X_4 &\to \R\text{ sind reellwertig}\\
                fg &= fg\mathbf{1}_{X_4} + \infty\mathbf{1}_{X_1} - \infty\mathbf{1}_{X_2} + 0\mathbf{1}_{X_3}
            \end{align*}
            Damit ist $fg$ nach Voraussetzung messbar, da $X_4\in\mA$.\qedhere
        \end{proof}
    \end{satz}

    \begin{satz}[WICHTIG]
        \label{satz:funktfolge-sup}
        Sei $(f_n)_n$ eine Folge messbarer Funktionen $f_n: X\to \overline{\R}$. Dann sind $\sup_n f_n$, $\inf_n f_n$, $\limsup_{n\toinf} f_n$ sowie $\liminf_{n\toinf} f_n$ messbar.
    \end{satz}

    \begin{bemerkung}
        Sei $s\coloneqq \sup_n f_n = \sup_{n\in \N} f_n$ ist punktweise definiert. Das heißt
        \begin{align*}
            s\of{x} &= \pair{\sup_n f_n}\of{x} \coloneqq \sup_n f_n\of{x} = \sup\set{f_n\of{x}: n\in \N}
        \end{align*}
    \end{bemerkung}

    \begin{proof}[Beweis von Satz~\ref{satz:funktfolge-sup}]
        Sei $s\of{x} \coloneqq \sup_{n} f_n\of{x}$. Dann ist
        \begin{align*}
            \set{s \leq \alpha} &= \set{x: s\of{x} \leq \alpha} \eqqcolon A_1\\
            \bigcup_{n\in\N}\set{f_n \leq \alpha} &= \bigcap_{n\in\N} \set{x\in X: f_n\of{x} < \alpha} = A_2 \in \mA
            \intertext{Wir zeigen $A_1 = A_2$. Sei $x\in A_1$. Dann ist}
            \alpha &\geq s\of{x} = \sup f_n\of{x} \geq f_n\of{x}\\
            \impl f_m\of{x} &\leq \alpha\quad\forall m\in\N\\
            \impl A_1 &\subseteq A_2
            \intertext{Sei $x\in A_2$. Dann folgt}
            f_n\of{x} &\leq\alpha\quad\forall n\in\N\\
            \impl \sup f_n\of{x} &\leq \alpha\impl x \in A_1
            \intertext{Damit ist $A_1 = A_2$. Es gilt}
            \inf_n f_n &= -\sup_{n} \pair{-f_n}
            \intertext{ist messbar}
            \pair{\limsup_{n} f_n}\of{x} &= \inf_n \sup_{\alpha \geq n} f_{\alpha}\of{x}\\
            \pair{\liminf_{n} f_n}\of{x} &= \sup_n \inf_{\alpha \geq n} f_{\alpha}\of{x}\qedhere
        \end{align*}
    \end{proof}

    \begin{korollar}
        Sei $(f_n)_n$ eine Folge messbarer Funktionen auf $X$ und $\pair{X, \mA}$ ein Messraum. Angenommen $\forall x\in X$ existiert der Grenzwert $f\of{x} \coloneqq \lim_{n\toinf} f_n\of{x}$ (wir schreiben auch $f = \lim_{n\toinf} f_n$). Dann ist $f$ messbar.

        \begin{proof}
            \begin{align*}
                f = \lim_{n\toinf} f_n &= \limsup_{n\toinf} f_n = \liminf_{n\toinf} f_n
            \end{align*}
            Damit ist $f$ nach dem vorherigen Satz messbar.
        \end{proof}
    \end{korollar}

    \begin{korollar}
        Seien $f_1, \ldots, f_n: X\to\overline{\R}$ messbare Funktionen. Dann sind
        \begin{align*}
            f_1 \lor f_2 \lor \dots \lor f_n &\coloneqq \max\of{f_1, f_2, \ldots, f_n}\tag{punktweise}
            \intertext{sowie}
            f_1 \land f_2 \land \dots \land f_n &\coloneqq \min\of{f_1, f_2, \ldots, f_n}
        \end{align*}
        messbar.
        \begin{proof}
            Nehme die Folge $(\overline{f}_n)_n$ mit $\overline{f}_m = f_m$ für $1\leq m \leq n$ und $\overline{f}_m = f_n$ für $m > n$. Damit ist
            \begin{align*}
                f_1 \lor \dots f_n &= \sup_n \overline{f}_n
            \end{align*}
            messbar nach Satz~\ref{satz:funktfolge-sup}. Die zweite Behauptung zeigt sich analog.
        \end{proof}
    \end{korollar}

    \begin{notation}
        Sei $f: X \to \overline{\R}$ eine Funktion. Wir definieren
        \begin{align*}
            f_+ &\coloneqq f \lor 0 = \max\of{f, 0} \geq 0\tag{Positivteil}\\
            f_- &\coloneqq \pair{-f} \lor 0 \geq 0\tag{Negativteil}
            \intertext{Damit gilt außerdem}
            f &= f_+ - f_-\\
            \abs{f} &= f_+ + f_-
        \end{align*}
    \end{notation}

    \begin{korollar}
        Sei $f: X \to \overline{\R}$ eine Funktion. Dann ist $f$ genau dann $\mA$-messbar, wenn $f_+$ und $f_-$ messbar sind.
    \end{korollar}

    \begin{korollar}
        Sei $f: X \to \overline{\R}$ eine $\mA$-messbare Funktion. Dann ist $\abs{f}$ messbar.
    \end{korollar}

    \newpage


    \section{[*] Elementarfunktionen und ihr Integral}
    \imaginarysubsection{Elementarfunktionen}
    \thispagestyle{pagenumberonly}

    \begin{definition}
        Es sei $\pair{X, \mA}$ ein Messraum. Wir definieren $E_+ \coloneqq E_+\of{X, \mA}$ als die Menge aller nicht-negativen Elementarfunktionen. Das heißt $u\in E_+$, falls $u: X \to \R_+ = \linterv{0, \infty}$ $\mA$-messbar ist und $\bild\of{u}$ endlich viele Werte annimmt. Das heißt
        \begin{align*}
            u\of{X} &= \set{\alpha_1, \alpha_2, \ldots, \alpha_n}\\
            \intertext{mit}
            0&\leq \alpha_j < \infty\quad\forall j \leq n\\
            \alpha_i &\neq \alpha_n\quad\forall i\neq j
        \end{align*}
        Wir setzen $A_j \coloneqq u^{-1}\of{\set{\alpha_j}} \in \mA$. Damit gilt $u\of{x} = \sum_{j=1}^{n} \alpha_j \charfunc_{A_j}\of{x}$. Wir haben den Raum $X$ nun folgendermaßen disjunkt zerlegt: $X= A_1 \sqcup A_2 \sqcup \dots \sqcup A_n$.\\
        Sind umgekehrt nicht-notwendigerweise disjunkte $\tilde{A}_1, \ldots, \tilde{A}_n \in \mA$. Dann ist $ \sum_{j=1}^{n} \tilde{\alpha_j} \charfunc_{\tilde{A}_j}$ eine Elementarfunktion.
    \end{definition}

    \begin{bemerkung}[Eigenschaften von $E_+$]
        Sei $\alpha \in \R_+$ und $u, v\in E_+$. Dann gilt
        \begin{enumerate}[label=-]
            \item $\alpha u\in E_+$
            \item $u + v \in E_+$
            \item $u - v\in E_+$
            \item $u \land v \in E_+$
            \item $u\lor v\in E_+$
        \end{enumerate}

        \begin{proof}
            Folgt direkt aus der Definition und den Messbarkeitseigenschaften aus dem letzten Kapitel.
        \end{proof}
    \end{bemerkung}

    \begin{notation}[Normaldarstellung]
        Sei $u \in E_+$. Dann schreiben wir $u\of{x} = \sum_{j=1}^{n} \alpha_j \charfunc_{A_j}$ als Normaldarstellung mit $\bigsqcup_{j=1}^n = X$.
    \end{notation}

    \begin{lemma}
        \label{lemma:elementar-eind}
        Sei $u\in E_+ = E_+\of{X, \mA}$ mit Normaldarstellungen
        \begin{align*}
            u = \sum_{j=1}^{n} \alpha_j \charfunc_{A_j} &= \sum_{k=1}^{n} \beta_k \charfunc_{B_k}
            \intertext{Sei $\mu$ ein Maß auf $\pair{X, \mA}$. Dann gilt}
            \sum_{j=1}^{n} \alpha_j \mu\of{A_j} &= \sum_{k=1}^{n} \beta_k \mu\of{B_k}
        \end{align*}
        Das heißt die Uneindeutigkeit der Normaldarstellung verschwindet bei der Verwendung eines Maßes.
        \begin{proof}
            \begin{align*}
                X &= A_1 \sqcup \dots \sqcup A_m\\
                &= B_1 \sqcup \dots \sqcup B_n\\
                \impl A_j = \bigsqcup_{k=1}^n \pair{A_j \cap B_k} &\quad B_k = \bigsqcup_{j=1}^m \pair{A_j \cap B_k}\\
                \impl \charfunc_{A_j} = \sum_{k=1}^{n} \charfunc_{A_j \cap B_k} &\quad\charfunc_{B_k} = \sum_{j=1}^{m} \charfunc_{A_j \cap B_k}\\
                \impl u &= \sum_{j=1}^{m} \alpha_j \charfunc_{A_j} &= \sum_{j=1}^{n} \sum_{k=1}^{m} \alpha_j \charfunc_{A_j \cap B_k} = \sum_{j=1}^{n} \sum_{k=1}^{m} \beta_k \charfunc_{A_j \cap B_k}
            \end{align*}
            Jeder Punkt $x\in X$ liegt in genau einer Menge $A_j \cap B_k$
            \begin{align*}
                u\of{x} &= \alpha_{j_0} \charfunc_{A_{j_0} \cap B_k}\of{x} = \beta_{k_0} \charfunc_{A_{j_0} \cap B_{k_0}}\\
                \sum_{j=1}^{m} \alpha_j \mu\of{A_j} &= \sum_{j=1}^{m} \alpha_j \mu\of{\bigcup_{k=1}^n A_j \cap B_k}\\
                &= \sum_{k=1}^{n} \mu\of{A_j \cap B_k} = \sum_{j=1}^{m} \sum_{k=1}^{n} \alpha_j \mu\of{A_j \cap B_k}\\
                &= \sum_{j=1}^{m} \sum_{k=1}^{n} \beta_k \mu\of{A_j \cap B_k} = \sum_{k=1}^{n} \beta_k \mu\of{B_k}\qedhere
            \end{align*}
        \end{proof}
    \end{lemma}

    \begin{definition}
        \marginnote{[16. Dez]}
        Sei $\pair{X, \mA, \mu}$ ein Maßraum und $u$ eine Elementarfunktion. Dann heißt die von der spezifischen Normaldarstellung $\mu = \sum_{j=1}^{n} \alpha_j \charfunc_{A_j}$ unabhängige Zahl (Lemma~\ref{lemma:elementar-eind})
        \begin{align*}
            \int_{}^{} u \dif \mu \coloneqq \sum_{j=1}^{n} \alpha_j \mu\of{A_j}
        \end{align*}
        das ($\mu$-)Integral von $u$ (über $X$). Wir schreiben auch
        \begin{align*}
            \int_{}^{} u \dif \mu = \int_{X}^{} u \dif \mu &= \int_{X}^{} u\of{x} \dif \mu\of{x} = \int_{X}^{} u\of{x} \mu\of{\dif x}
        \end{align*}
        und definieren in diesem Sinne eine Abbildung
        \begin{align*}
            E_+ &\to \overline{\R}\\
            u &\mapsto \int_{}^{} u \dif \mu
        \end{align*}
    \end{definition}

    \begin{lemma}[Eigenschaften des Integrals]
        Für $A\subseteq X$, $\alpha\in\R$ und $u,v\in E_+$ gilt
        \begin{enumerate}[label=(\roman*)]
            \item $\dsty \int_{}^{} \charfunc_{A} \dif \mu = \mu\of{A}$
            \item $\dsty \int_{}^{} \alpha u \dif \mu = \alpha \int_{}^{} u \dif \mu$
            \item $\dsty \int_{}^{} \pair{u + v} \dif \mu = \int_{}^{} u \dif \mu + \int_{}^{} v \dif \mu$
            \item $u\leq v \impl \dsty \int_{}^{} u \dif \mu \leq \int_{}^{} v \dif \mu$
        \end{enumerate}
        \begin{proof}
        (i)
            und (ii) folgen direkt aus der Definition. Für (iii) sei
            \begin{align*}
                u = \sum_{j=1}^{m} \alpha_j \charfunc_{A_j} &\quad v = \sum_{k=1}^{n} \beta_k \charfunc_{B_k}\\
                \impl A_j = \bigsqcup_{k=1}^n A_j \cap B_k &\quad B_k = \bigsqcup_{j=1}^m A_j \cap B_k\\
                u = \sum_{j=1}^{m} \sum_{k=1}^{n} \alpha_j \charfunc_{A_j \cap B_k} &\quad v = \sum_{j=1}^{m} \sum_{k=1}^{n} \beta_k \charfunc_{A_j \cap B_k}\\
                \impl u +v = \sum_{j=1}^{m} \sum_{k=1}^{n} &\pair{\alpha_j + \beta_k} \charfunc_{A_j \cap B_k}\\
                \int_{}^{} u \dif \mu = \sum_{j=1}^{m} \sum_{k=1}^{n} \alpha_j \mu\of{A_j \cap B_k} &\quad \int_{}^{} v \dif \mu = \sum_{j=1}^{m} \sum_{k=1}^{n} \beta_k \mu\of{A_j \cap B_k}
            \end{align*}
            \begin{align*}
                \int_{}^{} \pair{u+v} \dif \mu &= \sum_{j=1}^{m} \sum_{k=1}^{n} \pair{\alpha_j + \beta_k} \mu\of{A_j \cap B_k}\\
                &= \sum_{j=1}^{m} \sum_{k=1}^{n} \alpha_j \mu\of{A_j \cap B_k} + \sum_{j=1}^{m} \sum_{k=1}^{n} \beta_k \mu\of{A_j \cap B_k} = \int_{}^{} u \dif \mu + \int_{}^{} v \dif \mu
            \end{align*}
            Für (iv) gilt mit der Definition, die wir auch schon bei (iii) verwendet haben, dass $\alpha_j \leq \beta_k$ auf $A_j \cap B_k$
            \begin{align*}
                \impl \int_{}^{} u \dif \mu = \sum_{j}^{k} \alpha_j \mu\of{A_j \cap B_k} \leq \sum_{j}^{} \sum_{k}^{} \beta_k \mu\of{A_j \cap B_k} = \int_{}^{} v \dif \mu\qedhere
            \end{align*}
        \end{proof}
    \end{lemma}

    \newpage


    \section{[*] Das Integral von nicht-negativen meßbaren Funktionen}
    \imaginarysubsection{Integral nicht-negativer Funktionen}
    \thispagestyle{pagenumberonly}

    \begin{satz}
        Für jede wachsende Folge $(u_n)_n \subseteq E_+$ und jedes $u\in E_+$ gilt
        \begin{align*}
            u \leq \sup_{n\in\N} u_n \impl \int_{}^{} u \dif \mu \leq \sup_{n\in \N} \int_{}^{} u_n \dif \mu
        \end{align*}

        \begin{proof}
            Sei $u = \sum_{j=1}^{m} \alpha_j \charfunc_{A_j}$ und $0 < \delta < 1$. Wir definieren $B_n \coloneqq \set{x\in X: u_n\of{x} \geq \delta u\of{x}}$. Wir wissen nach Voraussetzung, dass $u_n \leq u_{n+1}$, das heißt $\sup_{n\in\N} u_n\of{x} \geq u\of{x}~\forall x\in X$. Sei $x\in X$ fest. Ist $u\of{x} > 0$, dann gilt $\exists N\of{x}\in\N: u_n\of{x}\geq \delta u\of{x}$ für $n\geq N\of{x}$.\\
            Aus $u_{n+1} \geq u_n$ folgt $B_n \subseteq B_{n+1}$. Außerdem ist $\forall x\in X: x\in B_n$ für ein ausreichend großes $n$. Also haben wir $B_n \nearrow X$. Auch ist $u_n \geq \delta u\charfunc_{B_n}$
            \begin{align*}
                \impl \int_{}^{} u_n \dif \mu &\geq \int_{}^{} \delta u\charfunc_{B_n} \dif \mu\\
                u &= \sum_{j=1}^{n} \alpha_j \charfunc_{A_j}\\
                \impl u \charfunc_{B_n} &= \sum_{j=1}^{n} \alpha_j \charfunc_{A_j \cap B_n}\\
                \delta \int_{}^{} u\charfunc_{B_n} \dif\mu &= \delta \int_{}^{} \sum_{j=1}^{n} \alpha_j \charfunc_{A_j \cap B_n} \dif \mu\\
                &= \delta \sum_{j=1}^{n} \alpha_j \mu\of{A_j \cap B_n} \to \delta \sum_{j=1}^{n} \alpha_j \mu\of{A_j}\\
                \impl \sup_{n\in\N} \int_{}^{} u_n \dif \mu = \lim_{n\toinf} \int_{}^{} u \dif \mu &\geq \delta\sum_{j=1}^{n} \alpha_j \lim_{n\toinf} \mu\of{A_j \cap B_n} = \delta\sum_{j=1}^{n} \alpha_j \charfunc_{A_j} = \delta\int_{}^{} u \dif \mu
            \end{align*}
            Das gilt für alle $\delta < 1$. Mit dem Limes für $\delta \to 1$ folgt dann die Behauptung.
        \end{proof}
    \end{satz}

    \begin{korollar}
        Seien $(u_n)_n, (v_n)_n \subseteq E_+$ wachsende Folgen von Elementarfunktionen. Dann gilt
        \begin{align*}
            \sup_{n\in\N} u_n = \sup_{n\in\N} v_n \impl \sup_{n\in\N} \int_{}^{} u_n \dif \mu = \sup_{n\in\N} \int_{}^{} v_n \dif \mu
        \end{align*}

        \begin{proof}
            Nach dem vorherigen Satz und $u_n \leq \sup_{n\in\N} v_n$ gilt
            \begin{align*}
                \int_{}^{} u_n \dif \mu &\leq \sup_{n\in\N} \int_{}^{} v_n \dif \mu\\
                \impl \sup_{n\in\N} \int_{}^{} u_n \dif x &\leq \sup_{n\in\N} \int_{}^{} v_n \dif \mu\\
                \intertext{Aus Symmetriegründen gilt dann}
                \impl \sup_{n\in\N} \int_{}^{} u_n \dif x &= \sup_{n\in\N} \int_{}^{} v_n \dif \mu
            \end{align*}
        \end{proof}
    \end{korollar}

    \begin{notation}
        Wir schreiben $E^{\ast} = E^{\ast}_+ = E^{\ast}_+\of{X, \mA}$ für die Menge aller numerischen meßbaren Funktionen $f\geq 0$ auf $X$, für die es eine wachsende Folge $(u_n)_n \subseteq E_+$ gibt, sodass $\sup_{n\in\N} u_n = f$. Für $f\in E_+^n$ ist dann
        \begin{align*}
            \int_{}^{} f \dif \mu \coloneqq \sup\set{\int_{}^{} u_n \dif \mu: u_n \subseteq E_+,~\sup_{n\in\N} u_n = f}
        \end{align*}
        Das nennen wir das ($\mu$-)Integral von $f$.
    \end{notation}

    \begin{bemerkung}[Eigenschaften des Integrals]
        \marginnote{[20. Dez]}
        Seien $f, g\in E^{\ast}$ und $\alpha\in\R_+$. Dann gilt
        \begin{enumerate}[label=(\roman*)]
            \item $\alpha f\in E^{\ast}$
            \item $f\pm g \in E^{\ast}$
            \item $fg \in E^{\ast}$
            \item $f \lor g \coloneqq \max\of{f, g}\in E^{\ast}$
            \item $f \land g \coloneqq \min\of{f, g}\in E^{\ast}$
            \item $\dsty \int_{}^{} \pair{\alpha f} \dif \mu = \alpha\int_{}^{} f \dif \mu$
            \item $\dsty \int_{}^{} \pair{f+g} \dif \mu = \int_{}^{} f \dif \mu + \int_{}^{} g \dif \mu$
            \item $f\leq g \impl \dsty \int_{}^{} f \dif \mu \leq \int_{}^{} g \dif \mu$
        \end{enumerate}

        \begin{proof}
            \theoremescape
            \begin{enumerate}
                \item[(i)-(v)] Beachte $f = \sup_{n\in\N} u_n$ und $g = \sup_{n\in\N} v_n$. Damit lassen sich entsprechende Folgen finden, um auch Kombinationen von $f$ und $g$ im obigen Sinne zu finden.
                \item[(vi)] (Übung)
                \item[(vii)] Es ist $f = \lim_{n\toinf} u_n$, $g= \lim_{n\toinf} v_n$. Das heißt
                \begin{align*}
                    \int_{}^{} f \dif \mu &= \sup_{n\in\N} \int_{}^{} u_n \dif \mu = \lim_{n\toinf} \int_{}^{} u_n \dif \mu\\
                    \int_{}^{} g \dif \mu &= \sup_{n\in\N} \int_{}^{} v_n \dif \mu = \lim_{n\toinf} \int_{}^{} v_n \dif \mu\\
                    \int_{}^{} f+g \dif \mu &= \sup_{n\in\N} \int_{}^{} (u_n + v_n) \dif \mu = \lim_{n\toinf} \int_{}^{} (u_n + v_n) \dif \mu\\
                    &= \lim_{n\toinf} \pair{\int_{}^{} u_n \dif \mu + \int_{}^{} v_n \dif \mu} = \lim_{n\toinf} \int_{}^{} u_n \dif \mu + \lim_{n\toinf} \int_{}^{} v_n \dif \mu\\
                    &= \int_{}^{} f \dif \mu + \int_{}^{} g \dif \mu
                \end{align*}
                \item[(viii)] $f\leq g \impl u_k \leq \sup_{n\in\N} v_n$. Wir definieren $(\tilde{u}_n)_n$ mit $\tilde{u}_n \coloneqq u_k$ für ein festes $k$
                \begin{align*}
                    \impl \sup_{n\in\N} \tilde{u}_k = u_k &\leq \sup_{n\in\N} v_n\\
                    \impl \int_{}^{} u_k \dif \mu = \sup_{n\in\N} \int_{}^{} \tilde{u}_k \dif \mu &\leq \sup_{n\in\N} \int_{}^{} v_n \dif \mu = \int_{}^{} g \dif \mu\\
                    \impl \int_{}^{} u_k \dif \mu &\leq \int_{}^{} g \dif \mu\\
                    \impl \int_{}^{} f \dif \mu &= \sup_{k\in\N} \int_{}^{} u_k \dif \mu = \int_{}^{} g \dif \mu
                \end{align*}
            \end{enumerate}
        \end{proof}
    \end{bemerkung}

    \begin{bemerkung}[Interpretation des Integrals]
        Das Integral bezüglich einem Maß $\mu$ ist eine monotone Linearform von $E^{\ast}$ nach $\interv{0, \infty}$. Aber was ist $E^{\ast}$?
    \end{bemerkung}

    \begin{notation}
        Es sei $\mM_+ \coloneqq \mM_+\of{X, \mA} \coloneqq \set{f: x\to\interv{0, \infty},~f\text{ ist }\mA\text{-messbar}}$.
    \end{notation}

    \begin{satz}
        $E^{\ast}\of{X, \mA} = \mM_+\of{X, \mA}$.

        \begin{proof}
            $E^{\ast} \subseteq \mM_+$ ist klar. Wir zeigen die Inklusion in die andere Richtung. Wir definieren für ein $n\in\N$, $j\in\set{0, \ldots, n2^{n}}$
            \begin{align*}
                A_{j,n} &\coloneqq \begin{cases}
                                       \set{f \geq j2^{-n}} \cap \set{f \leq (j+1)2^{-n}} & j\in\set{0, \ldots, n2^{n}-1}\\
                                       \set{f\geq n} &j = n2^{n}-1
                \end{cases}\\
                &= \begin{cases}
                       \set{j2^{-n} \leq f \leq (j+1)2^{-n}}& j\in\set{0, \ldots, n2^{n}-1}\\
                       \set{f\geq n} &j = n2^{n}-1
                \end{cases}
            \end{align*}
            Für ein festes $n$ ist $A_{j,n} \cap A_{k, n} = \emptyset$ für $j\neq k$. Wir setzen
            \begin{align*}
                u_n &\coloneqq \sum_{j=0}^{n2^n} j2^{-n} \charfunc_{A_{j,n}}
            \end{align*}
            Behauptung 1: $u_n \leq u_{n+1}~\forall n\in\N$. Bew.:
            \begin{align*}
                A_{2k, n+1} &= \set{2k2^{-(n+1)} \leq f \leq (2k+1)2^{-(n+1)}} = \set{k2^{-n} \leq f \leq (2k+1)2^{-(n+1)}}\\
                A_{2k+1, n+1} &= \set{(2k+1)2^{-(n+1)} \leq f \leq (2k+1)2^{-(n+1)}}\\
                \impl A_{k,n} &= A_{2k, n+1} \sqcup A_{2k+1, n+1}\\
                \impl u_{n} &\leq u_{n+1}
            \end{align*}
            Damit ist auch $u_n \leq f~\forall n\in\N$. Umgekehrt auf $A_{j,n}$ ist $f\of{x} \leq (j+1)2^{-n} = u_n\of{x} + 2^{-n}$. Das heißt $\lim_{n\toinf} u_n = f$ für $x\in A_{j,n}$. Insgesamt gilt damit $\lim_{n\toinf}u_n = f$.
        \end{proof}
    \end{satz}

    \begin{satz}[Monotone Konvergenz] % Satz 5
        \label{satz:monoton-konv}
        Sei $(f_n)_n$ eine wachsende Folge in $E^{\ast}\of{X, \mA}$. Dann folgt $\dsty\sup_{n\in\N} f_n \in E^{\ast}$ und $\dsty\sup_{n\in\N} \int_{}^{} f_n \dif \mu = \int_{}^{} \sup_{n\in\N} f \dif \mu$.

        \begin{proof}
            Sei $f\coloneqq \sup_{n\in\N} f_n$. Dann reicht es, zu zeigen, dass $\exists (v_n)_n\subseteq E_+$ wachsend mit $\sup_{n\in\N} v_n = f$ und $v_n \leq f_n$. Denn in diesem Fall ist
            \begin{align*}
                \int_{}^{} v_n \dif \mu &\leq \int_{}^{} f_n \dif \mu \leq \sup_{n\in\N} \int_{}^{} f_n \dif \mu\\
                \impl \sup_{n\in\N} \int_{}^{} v_n \dif \mu &\leq \sup_{n\in\N} \int_{}^{} f_n \dif \mu
                \intertext{Mit Bemerkung~\ref{bemerkung:temp-lemma} folgt dann}
                \sup_{n\in\N}\int_{}^{} f_n \dif \mu &= \int_{}^{} \sup_{n\in\N} f_n \dif \mu
                \intertext{Wir brauchen also nur noch die Existenz von $(v_n)_n$. Zu $(f_n)$ existiert ein $(u_{m,n})_n \subseteq E_+$ mit $u_{m, n} \leq u_{m-1, n}$ sowie $f_n = \sup_{m} u_{m,n}$. Das heißt}
                v_m &\leq v_{m+1}\\
                \impl \sup_{m} v_m &\in E_+
                \intertext{Behauptung: $\sup_{m} v_m = f = \sup_n f_n$. Da $u_{m,n} \leq f_n \leq f_{n+1}$}
                \impl \sup_{m} v_m &\leq \sup_{m} f_m = f\\
                \intertext{Auch ist $v_m = u_{m-1} \lor u_{m-2} \lor \ldots \geq u_{m,n} \to f_n$}
                f_n &= \sup_{m} u_{m,n} = \sup_{m} v_m\\
                \impl f &= \sup_{n} f_n \leq \sup_{m} v_m \leq f\\
                \impl f &= \sup_{m\in\N} v_m\qedhere
            \end{align*}
        \end{proof}
    \end{satz}

    \begin{bemerkung}
        \label{bemerkung:temp-lemma}
        Im Sinne von Satz~\ref{satz:monoton-konv} gilt die Ungleichung $\dsty\sup_{n\in\N} \int_{}^{} f_n \dif \mu \leq \int_{}^{} \sup_{n\in\N} f \dif \mu$ allgemein. Die Voraussetzung, dass die Folge wächst ist nur für die Ungleichung in die andere Richtung relevant.
    \end{bemerkung}

    \begin{korollar}
        Sei $(f_n)_n \subseteq E^{\ast}$. Dann folgt $\dsty \sum_{n=1}^{\infty} f_n \in E^{\ast}$ und $\dsty \int_{}^{} \pair{ \sum_{n=1}^{\infty} f_n} \dif \mu = \sum_{n=1}^{\infty} \int_{}^{} f_n \dif \mu$.
        \begin{proof}
            Wir haben $f_n \geq 0$. Das heißt die Partialsummen von $(f_n)_n$ sind monoton wachsend. Damit ergibt sich die Behauptung direkt aus Satz~\ref{satz:monoton-konv}.
        \end{proof}
    \end{korollar}

    \newpage


    \section{[*] Integrierbarkeit}

    \subsection{Integration von allgemeinen numerischen Funktionen}
    \thispagestyle{pagenumberonly}

    Wir haben $f\geq 0$ messbar auf $X$. Dann ist $\dsty \int_{X}^{} f \dif \mu$ wohldefiniert. Wir wollen das auch auf Funktionen mit negativen Werten erweitern.

    \begin{definition}
        Eine numerische Funktion $f: X\to\overline{\R}$ heißt ($\mu$-)integrierbar, wenn sie $\mA$-messbar ist sowie $ \int_{}^{} f_+ \dif \mu, \int_{}^{} f_- \dif \mu\in\R$. In diesem Fall definieren wir
        \begin{align*}
            \int_{}^{} f \dif \mu &= \int_{}^{} f_+ \dif \mu - \int_{}^{} f_- \dif \mu
        \end{align*}
    \end{definition}

    \begin{satz}
        \label{satz:integrierbarkeit-equi}
        Sei $f: X \to\overline{\R}$ messbar und $\mu$ ein Maß auf $\mA$. Dann sind folgende Aussagen äquivalent
        \begin{enumerate}[label=(\roman*)]
            \item $f_+$ und $f_-$ sind integrierbar
            \item Es existieren integrierbare Funktionen $u, v\geq 0$ mit $f = u-v$
            \item Es existiert eine integrierbare Funktion $g\geq 0$ mit $\abs{f} \leq g$
            \item $\abs{f}$ ist integrierbar
        \end{enumerate}
        Im Fall (ii) ist $\dsty \int_{}^{} f \dif \mu = \int_{}^{} u \dif \mu - \int_{}^{} v \dif \mu$.
        \marginnote{[10. Jan]}
        \begin{proof}
        (i)
            $\impl$ (ii): Wir wählen $u= f_+$ und $v=f_-$. Damit existieren die geforderten Funktionen.\\[.2\baselineskip]
            (ii) $\impl$ (iii): $u$ und $v$ sind integrierbare, nicht-negative Funktionen. Damit ist auch $u+v$ integrierbar und nicht-negativ und es gilt
            \begin{align*}
                f &= u-v \leq u \leq u+v\\
                -f &= v -u \leq v \leq v+u\\
                \impl \abs{f} &\leq u + v \eqqcolon g
            \end{align*}
            (iii) $\impl$ (iv): Aus der Monotonie des Integrals (für nicht-negative, integrierbare Funktionen) folgt
            \begin{align*}
                \int_{}^{} \abs{f\of{x}} \dif \mu\of{x} &\leq \int_{}^{} g\of{x} \dif \mu\of{x} < \infty
            \end{align*}
            (iv) $\impl$ (i): Es gilt $f_+ \leq \abs{f}$ und $f_- \leq \abs{f}$. Das heißt nach der Argumentation aus der vorherigen Implikation sind $f_+$ und $f_-$ integrierbar.\\[.5\baselineskip]
            Im Fall (ii) ist $f = u-v = f_+ - f_-$. Daraus folgt
            \begin{align*}
                \impl u + f_- &= f_+ + v\\
                \impl \int_{}^{} u \dif \mu + \int_{}^{} f_- \dif \mu &= \int_{}^{} f_+ \dif \mu + \int_{}^{} v \dif \mu\\
                \impl \int_{}^{} u \dif x - \int_{}^{} v \dif \mu &= \int_{}^{} f_+ \dif \mu - \int_{}^{} f_- \dif \mu = \int_{}^{} f \dif \mu
            \end{align*}
        \end{proof}
    \end{satz}

    \begin{bemerkung}[Komplexwertige Integrale]
        Eine messbare Funktion $f: X \to\C$ ist integrierbar, falls $Re\of{f}$ und $\Im\of{f}$ integrierbar sind. Wir definieren dann
        \begin{align*}
            \int_{}^{} f \dif \mu &= \int_{}^{} \Re\of{f} \dif \mu + i\int_{}^{} \Im\of{f} \dif \mu
        \end{align*}
        Eine entsprechende Version von Satz~\ref{satz:integrierbarkeit-equi} gilt dann auch für komplexe Funktionen.
    \end{bemerkung}

    \subsection{Linearität und Monotonie des Integrals}

    \begin{satz} % Satz 3
        \label{satz:int-basic-prop}
        Seien $f, g$ integrierbar und $\alpha\in\R$. Dann sind $\alpha f$ und $f+g$ ebenfalls integrierbar und es gilt
        \begin{align*}
            \int_{}^{} \alpha f \dif \mu &= \alpha \int_{}^{} f \dif \mu\tag{i}\\
            \int_{}^{} f+g \dif \mu &= \int_{}^{} f \dif \mu + \int_{}^{} g \dif \mu\tag{ii}
        \end{align*}
        Außerdem sind auch $\max\of{f, g}$ und $\min\of{f, g}$ integrierbar.\hfill(iii)
        \begin{proof}
            \theoremescape
            \begin{enumerate}[label=(\roman*)]
                \item
                Es gilt $\pair{\alpha f}_+ = \alpha f_+$ und $\pair{\alpha f}_- = \alpha f_-$, falls $\alpha \geq 0$ sowie $\pair{\alpha f}_- = \alpha f_+$ und $\pair{\alpha f}_+ = \alpha f_-$, falls $\alpha < 0$. Für $\alpha \geq 0$ gilt
                \begin{align*}
                    \int_{}^{} \alpha f \dif \mu &= \int_{}^{} \pair{\alpha f}_+ \dif \mu - \int_{}^{} \pair{\alpha f}_- \dif \mu = \int_{}^{} \alpha f_+ \dif \mu - \int_{}^{} \alpha f_- \dif \mu\\
                    &= \alpha \int_{}^{} f_+ \dif \mu - \alpha \int_{}^{} f_- \dif \mu = \alpha \int_{}^{} f \dif \mu
                \end{align*}
                Der Fall $\alpha < 0$ zeigt sich analog.
                \item $f$ und $g$ lassen sich in positiv- und negativ-Teil zerlegen. Das heißt
                \begin{align*}
                    f+g &= f_+ - f_- + g_+ - g_- = \underbrace{f_+ + g_+}_{\eqqcolon u} - \underbrace{\pair{f_- + g_-}}_{\eqqcolon v}
                    \intertext{Mit Satz~\ref{satz:integrierbarkeit-equi} folgt die Integrierbarkeit von $f+g$ und es gilt}
                    \int_{}^{} f+g \dif \mu &= \int_{}^{} u \dif \mu - \int_{}^{} v \dif \mu = \int_{}^{} f_+ + g_+ \dif \mu - \int_{}^{} f_- + g_- \dif \mu\\
                    &= \int_{}^{} f_+ \dif \mu - \int_{}^{} f_- \dif \mu + \int_{}^{} g_+ \dif \mu - \int_{}^{} g_- \dif \mu = \int_{}^{} f \dif \mu + \int_{}^{} g \dif \mu
                \end{align*}
                \item Es gilt $\abs{\max\of{f, g}}\leq \abs{f} + \abs{g}$ sowie $\abs{\min\of{f, g}}\leq \abs{f} + \abs{g}$. Damit folgt die Integrierbarkeit direkt aus Satz~\ref{satz:integrierbarkeit-equi}, da $f$ und $g$ integrierbar sind.
            \end{enumerate}
        \end{proof}
    \end{satz}

    \begin{bemerkung}
        Für komplexe Funktionen $f, g: X\to \C$ gilt Satz~\ref{satz:int-basic-prop} analog (mit $\alpha\in\C$).
        \begin{proof}
            Der Beweis ist eine Standardrechnung und funktioniert genauso wie der Beweis des Satzes für reelle Funktionen mit einer zusätzlichen Zerlegung in Real- und Imaginärteil.
        \end{proof}
    \end{bemerkung}

    \begin{satz}
        \label{satz:int-monoton-dreieck}
        Seien $f, g$ integrierbar auf $X$. Dann gilt
        \begin{align*}
            f \leq g \impl \int_{}^{} f \dif \mu &\leq \int_{}^{} g \dif \mu\tag{Monotonie}
            \intertext{sowie}
            \abs{ \int_{}^{} f \dif \mu} &\leq \int_{}^{} \abs{f} \dif \mu\tag{Dreiecksungleichung}
        \end{align*}

        \begin{proof}
            Monotonie: Aus $f\leq g$ folgen $f_+ \leq g_+$ und $f_- \geq g_-$
            \begin{align*}
                \impl \int_{}^{} f \dif \mu &= \int_{}^{} f_+ \dif \mu - \int_{}^{} f_- \dif \mu\\
                &\leq \int_{}^{} g_+ \dif \mu - \int_{}^{} g_- \dif \mu = \int_{}^{} g \dif \mu
            \end{align*}
            Dreiecksungleichung: Da $f\leq \abs{f}$ und $-f \leq \abs{f}$ folgt aus der Monotonie
            \begin{align*}
                \int_{}^{} f \dif \mu &\leq \int_{}^{} \abs{f} \dif \mu\\
                -\int_{}^{} f \dif \mu = \int_{}^{} -f \dif \mu &\leq \int_{}^{} \abs{f} \dif \mu\\
                \impl \abs{ \int_{}^{} f \dif \mu} &\leq \int_{}^{} \abs{f} \dif \mu
            \end{align*}
        \end{proof}
    \end{satz}

    \begin{bemerkung}
        Sei $f: X \to \C$ integrierbar. Bisher haben sich alle Sätze und Argumente für reelle Funktionen ohne Probleme auch auf komplexe Funktionen übertragen. Bei Satz~\ref{satz:int-monoton-dreieck} ist hier jedoch etwas Vorsicht geboten. Wir versuchen, die Dreiecksungleichung analog zum vorherigen Beweis auch auf komplexe Funktionen zu übertragen:
        \begin{align*}
            \abs{ \int_{}^{} f \dif \mu} &= \abs{\int_{}^{} \Re\of{f} \dif \mu + i \int_{}^{} \Im\of{f} \dif \mu}\\
            &\leq \abs{ \int_{}^{} \Re\of{f} \dif \mu} + \abs{\int_{}^{} \Im\of{f} \dif \mu}\\
            &\leq \int_{}^{} \abs{\Re\of{f}} \dif \mu + \int_{}^{} \abs{\Im\of{f}} \dif \mu\\
            &= \int_{}^{} \abs{\Re\of{f}} + \abs{\Im\of{f}} \dif \mu
            \intertext{Die einzige Abschätzung, die wir jetzt noch sinnvoll machen können, um wieder $\abs{f}$ im Integral stehen zu haben, ist, sowohl Real- als auch Imaginärteil gegen den Betrag abzuschätzen}
            &\leq 2 \int_{}^{} \abs{f} \dif \mu
        \end{align*}
        Der Faktor, den wir hier zusätzlich erhalten, ergibt sich aber nicht aus den Eigenschaften des komplexen Integrals, sondern entsteht nur durch unsere zu großzügigen Abschätzungen. Wir führen stattdessen ein besseres Argument mit Polarkoordinaten: \textit{(fehlt)}

        %%
    \end{bemerkung}

    \subsection{Der Raum $\mL^1\of{\mu}$}
    \begin{definition}
        Wir definieren $\mL^{1}\of{\mu} \coloneqq \mL^{1}\of{\mu, \R}$ als die Menge aller $\mu$-integrierbaren reellen Funktionen auf $X$. Analog dazu ist $\mL^{1}\of{\mu, \C}$ die Menge aller $\mu$-integrierbaren komplexen Funktionen. Aus Satz~\ref{satz:int-basic-prop} und der zugehörigen Bemerkung folgt bereits, dass $\mL^{1}\of{\mu, \R}$ bzw. $\mL^1\of{\mu, \C}$ Vektorräume bezüglich Addition und Multiplikation mit $\alpha\in\R$ bzw. $\alpha\in\C$ sind.\\
        Es gilt
        \begin{align*}
            \abs{ \int_{}^{} f \dif x - \int_{}^{} g \dif \mu} = \abs{\int_{}^{} f-g \dif \mu} \leq \int_{}^{} \abs{f-g} \dif \mu
        \end{align*}
        Damit definiert $\norm{f}_1 \coloneqq \int_{}^{} \abs{f} \dif \mu$ eine Halbnorm auf unserem Vektorraum. Dabei ist $\norm{f}_1$ keine Norm, denn aus $\norm{f}_1 = 0$ folgt nicht-notwendigerweise, dass $f=0$. Betrachten wir dafür das Lebesgue-Maß und die Funktion $f=\charfunc_{0}$. Dann gilt $ \int_{}^{} \abs{f} \dif \lambda^{1} = \lambda^{1}\of{\set{0}} = 0$, aber $f$ ist nicht die Nullfunktion.
    \end{definition}

    \subsection{Integration über (messbaren) Teilmengen}
    \begin{definition}
        Sei $f$ eine messbare numerische Funktion, die entweder nicht-negativ oder integrierbar bezüglich $\mu$ ist. Sei weiter $A\in\mA$. Dann setzt man
        \begin{align*}
            \int_{A}^{} f \dif \mu = \int_{}^{} f\cdot \charfunc_{A} \dif \mu
        \end{align*}
        und nennt $ \int_{A}^{} f \dif \mu$ das ($\mu$-)Integral von $f$ über $A$.
    \end{definition}

    \begin{bemerkung}
        Seien $A, B\in \mA$. Dann gilt $\charfunc_{A\cup B} + \charfunc_{A \cap B} = \charfunc_{A} + \charfunc_{B}$. Das heißt
        \begin{align*}
            \int_{A \cup B}^{} f \dif x + \int_{A\cap B}^{} f \dif x &= \int_{A}^{} f \dif x + \int_{B}^{} f \dif \mu
        \end{align*}
        Insbesondere gilt $ \int_{A\cup B}^{} f \dif \mu = \int_{A}^{} f \dif \mu + \int_{B}^{} f \dif \mu$, falls $A\cap B = \emptyset$.
        \begin{proof}
        (Übung)
        \end{proof}
        Zusätzlich übertragen sich auch Monotonie, Linearität und Dreiecksungleichung (mit entsprechenden Einschränkungen auf die Menge $A$).
    \end{bemerkung}

    \begin{lemma}

    \end{lemma}

    \newpage


    \section{[*] Fast überall bestehende Eigenschaften}
    \imaginarysubsection{Fast überall bestehende Eigenschaften}
    \thispagestyle{pagenumberonly}

    \begin{definition}
        \marginnote{[13. Jan]}
        Sei $\pair{X, \mA, \mu}$ ein Maßraum und \textit{E} eine Eigenschaft, die für $x\in X$ sinnvoll definiert ist. Dann sagen wir \textit{E} ist ($\mu$-)fast-sicher wahr oder \textit{E} gilt ($\mu$-)fast-sicher (abgekürzt ($\mu$-)fs. und im Englischen ($\mu$-)almost-everywhere bzw. ($\mu$-)ae.), falls es eine Nullmenge $N\in\mA$ (d.h. $\mu\of{N} = 0$) gibt, sodass \textit{E} gilt für alle $x\not\in N$.
    \end{definition}

    \begin{beispiel}
        Sei $f: X \to \overline{\R}$ eine messbare, numerische Funktion. Dann ist $f$ $\mu$-fast-überall endlich, falls $\exists N\in\mA\colon \mu\of{N} = 0 \land \pair{\forall x\in\comp{N}\colon f\of{x}\in\R}$.
    \end{beispiel}

    \begin{beispiel}
        Sei $f_n$ eine Folge von messbaren Funktionen. Dann gilt $f_n \to f$ $\mu$-fs. genau dann, wenn eine Nullmenge $N\in\mA$ existiert, sodass $f_n\of{x} \to f\of{x}$ für alle $x\in\comp{N}$ und $n\toinf$.
    \end{beispiel}

    \begin{satz} % Satz 2
        \label{satz:nullmenge-integral-null}
        Sei $f \in \mM_+\of{X, \mA}$ eine nicht-negative, messbare Funktion. Dann gilt $ \int_{}^{} f \dif \mu = 0$ genau dann, wenn $f=0$ $\mu$-fast-überall.

        \begin{proof}
            \anf{$\impl$}: Wir definieren $N\coloneqq \set{f > 0} \in\mA$. Dann gilt
            \begin{align*}
                N &= \bigcup_{n\in\N} \underbrace{\set{f > \frac{1}{n}}}_{\eqqcolon N_n \in \mA}\\
                \mu\of{N_n} &= \mu\of{\set{f > \frac{1}{n}}} = \int_{}^{} \charfunc_{\set{f > \frac{1}{n}}} \dif \mu\\
                &= \int_{\set{f > \frac{1}{n}}}^{} 1 \dif \mu \leq \int_{\set{f > \frac{1}{n}}}^{} nf \dif \mu\\
                &\leq n \int_{}^{} f \dif \mu = 0\\
                \impl \mu\of{N} &= 0
            \end{align*}
            \anf{$\Leftarrow$}: Angenommen $f = 0$ $\mu$-fs. Das heißt es existiert eine Nullmenge $N$, sodass $f\of{x} = 0~\forall x\in\comp{N}$. Sei $u_m \coloneqq m \cdot \charfunc_{N}$, dann gilt $\forall x\in X$, dass $f\of{x} \leq \sup_{m} u_m\of{x} \eqqcolon u\of{x}$.
            \begin{align*}
                0\leq \int_{}^{} f \dif \mu &\leq \int_{}^{} u \dif \mu = \lim_{m\toinf} \int_{}^{} u_m \dif \mu = \lim_{m\toinf} \pair{m \mu\of{N}} = 0\qedhere
            \end{align*}
        \end{proof}
    \end{satz}

    \begin{korollar}
        Sei $f: X \to\overline{R}$ eine beliebige meßbare, numerische Funktion und $N\in\mA$ eine $\mu$-Nullmenge. Dann ist $f$ über $N$ integrierbar und $ \int_{N}^{} f \dif \mu = 0$.

        \begin{proof}
            \textsc{Fall 1}: Ist $f\geq 0$, dann folgt die Behauptung direkt aus Satz~\ref{satz:nullmenge-integral-null}.\\
            \textsc{Fall 2}: Ist $f = f_+ - f_-$, dann haben wir nach \textsc{Fall 1}, dass
            \begin{align*}
                \int_{N}^{} f_+ \dif \mu &= 0 = \int_{N}^{} f_- \dif \mu\\
                \impl \int_{N}^{} f \dif \mu &= \int_{N}^{} f_+ \dif \mu - \int_{N}^{} f_- \dif \mu = 0
            \end{align*}
        \end{proof}
    \end{korollar}

    \begin{satz} % Satz 4
        \label{satz:funkt-vergleich}
        Seien $f, g$ messbare, numerische Funktionen auf $X$, $\pair{X, \mA, \mu}$ ein Maßraum sowie $f = g$ $\mu$-fs. Dann gilt
        \begin{enumerate}[label=(\roman*)]
            \item $f, g \geq 0 \impl \int_{}^{} f \dif \mu = \int_{}^{} g \dif \mu$.
            \item $f$ integrierbar $\impl$ $g$ integrierbar mit $ \int_{}^{} g \dif \mu = \int_{}^{} f \dif \mu$.
        \end{enumerate}

        \begin{proof}
            Sei $N = \set{f \neq g}$. Dann ist $N$ eine Nullmenge.
            \begin{enumerate}[label=(\roman*)]
                \item Es gilt
                \begin{align*}
                    0 = \int_{N}^{} f \dif \mu &= \int_{N}^{} g \dif \mu = 0
                    \intertext{Sei $M \coloneqq \comp{N}$}
                    \impl \int_{M}^{} f \dif \mu = \int_{}^{} \charfunc_{M} f \dif \mu &= \int_{}^{} \charfunc_{M} g \dif \mu = \dif x \int_{M}^{} g \dif \mu\\
                    \impl \int_{}^{} f \dif \mu = \int_{N}^{} f \dif \mu + \int_{M}^{} f \dif \mu &= \int_{N}^{} g \dif \mu + \int_{M}^{} g \dif \mu = \int_{}^{} g \dif \mu
                \end{align*}
                \item Es ist $f_+ = g_+$ $\mu$-fs. sowie $f_- = g_-$ $\mu$-fs. Damit gilt nach (i), dass
                \begin{align*}
                    \underbrace{\int_{}^{} f_+ \dif \mu}_{< \infty} &= \int_{}^{} g_+ \dif \mu\\
                    \impl \int_{}^{} f \dif \mu &= \int_{}^{} f_+ \dif \mu - \int_{}^{} f_- \dif \mu = \int_{}^{} g \dif \mu
                \end{align*}
            \end{enumerate}
        \end{proof}
    \end{satz}

    \begin{korollar} % Korollar 5
        Seien $f, g$ meßbare, numerische Funktionen auf $X$ und $\abs{f} \leq g$ $\mu$-fast-überall und $g$ integrierbar. Dann ist $f$ integrierbar (bezüglich $\mu$).
        \begin{proof}
            Sei $\tilde{g} \coloneqq g \lor \abs{f} = \max\of{g, \abs{f}}$. Damit ist $\tilde{g} = g$ fast-überall nach Voraussetzung und $\abs{f} \leq \tilde{g}$. Damit gilt nach Satz~\ref{satz:funkt-vergleich}, dass
            \begin{align*}
                \int_{}^{} \tilde{g} \dif \mu &= \int_{}^{} g \dif \mu < \infty\\
                \impl \int_{}^{} \abs{f} \dif \mu &\leq \int_{}^{} \tilde{g} \dif \mu = \int_{}^{} g \dif \mu < \infty\\
                \impl &f\text{ ist integrierbar}\qedhere
            \end{align*}
        \end{proof}
    \end{korollar}

    \begin{satz} % Satz 6
        Sei $f$ eine integrierbare, numerische Funktion auf $\pair{X, \mA, \mu}$. Dann ist $f$ $\mu$-fs. reellwertig (d.h. $f$ ist $\mu$-fs. endlich). Ferner hat die Menge $\set{f \neq 0}$ ein $\sigma$-endliches Maß.

        \begin{proof}
            Sei $N\coloneqq \set{\abs{f} = \infty}\in\mA$. Dann gilt für beliebiges $\alpha \geq 0$, dass $\alpha \charfunc_{N} \leq \abs{f}$. Damit folgt außerdem
            \begin{align*}
                \alpha \mu\of{N} = \int_{N}^{} \alpha \charfunc_{N} \dif \mu &\leq \int_{N}^{} \abs{f} \dif \mu \leq \int_{}^{} \abs{f} \dif x < \infty\\
                \impl \mu\of{N} &\leq \frac{1}{\alpha} \int_{}^{} \abs{f} \dif \mu \to 0 \text{ für }\alpha \to \infty\\
                \impl \mu\of{N} &= 0
                \intertext{Also ist $\abs{f} < \infty$ $\mu$-fs. Zu zeigen bleibt die zweite Behauptung}
                \set{f\neq 0} = \set{\abs{f} > 0} &= \bigcup_{n\in\N} \underbrace{\set{\abs{f}\geq \frac{1}{n}}}_{\eqqcolon A_n}\\
                \mu\of{A_n} &= \int_{A_n}^{} 1 \dif \mu = \int_{A_n}^{} n\abs{f} \dif \mu\\
                &= n \int_{A_n}^{} \abs{f} \dif \mu \leq n \int_{}^{} \abs{f} \dif \mu < \infty\\
                \impl \mu\of{A_n} &< \infty\quad\forall n\in\N
            \end{align*}
            Also hat $\set{f \neq 0}$ ein $\sigma$-endliches Maß.
        \end{proof}
    \end{satz}

    \begin{bemerkung}
        Sei $A\in\mA$ und $f: A \to\overline{\R}$ $\pair{A \cap \mA}$-messbar. Wir definieren
        \begin{align*}
            \tilde{f}\of{x} &\coloneqq \begin{cases}
                                           f\of{x} &x\in A\\
                                           0 &x\not\in A
            \end{cases}
        \end{align*}
        Dann ist $\tilde{f}: X\to\overline{\R}$ $\mA$-messbar und genau dann integrierbar, wenn $f$ über $A$ integrierbar ist. Wir sagen $f$ ist $\mu$-fast-überall definiert, falls $\mu\of{\comp{A}} = 0$ und es gilt
        \begin{align*}
            \int_{}^{} \tilde{f} \dif \mu &= \int_{X}^{} \tilde{f} \dif \mu = \int_{A}^{} f \dif \mu
        \end{align*}
    \end{bemerkung}

    \newpage


    \section{[*] Konvergenzsätze}

    \subsection{Monotone Konvergenz und das Lemma von Fatou}
    \thispagestyle{pagenumberonly}

    \begin{satz}[Erweiterung von der monotonen Konvergenz]
        \label{satz:monton-konv}
        Sei $g\geq 0$ integrierbar und $f_n \geq -g~\forall n\in\N$ und $f_n \leq f_{n+1}~\forall n\in\N$. Dann gilt
        \begin{align*}
            \int_{}^{} \lim_{n\toinf} f_n \dif \mu &= \lim_{n\toinf} \int_{}^{} f_n \dif \mu = \sup_{n} \int_{}^{} f_n \dif \mu
        \end{align*}

        \begin{proof}
            Sei $h_n \coloneqq f_n - g \geq 0$ mit $h_n \leq h_{n+1}$. Dann ist

            \begin{align*}
                \int_{}^{} \lim_{n\toinf}  h_n\dif \mu &= \lim_{n\toinf} \int_{}^{} h_n \dif \mu = \int_{}^{} (f+g) \dif \mu\\
                &= \lim_{n\toinf} \int_{}^{} f_n \dif \mu + \int_{}^{} g \dif \mu \leq \int_{}^{} f_n \dif \mu + \int_{}^{} g \dif \mu\quad(???)
            \end{align*}
        \end{proof}
    \end{satz}

    \begin{lemma}[Lemma von Fatou]
        \marginnote{[16. Jan]}
        \label{lemma:fatou}
        Sei $(f_n)_{n\in\N} \subset\mM_+\of{X, \mA}$ eine Folge nicht-negativer\footnote{Es reicht sogar, nur zu fordern, dass $0\leq f_n$ $\mu$-fs.}, messbarer numerischer Funktionen. Dann gilt
        \begin{align*}
            \int_{}^{} \liminf_{n\toinf} f_n \dif \mu &\leq \liminf_{n\toinf} \int_{}^{} f_n \dif \mu
        \end{align*}

        \begin{proof}
            Nach Def. ist
            \begin{align*}
                f\coloneqq \liminf_{n\toinf} f_n &= \lim_{n\toinf} \underbrace{\inf_{k\geq n} f_k}_{\eqqcolon g_n}\\
                \impl g_n &\leq g_{n+1}\\
                \int_{}^{} f \dif \mu &= \int_{}^{} \lim_{n\toinf} g_n \dif \mu \annot{=}{Satz~\ref{satz:monoton-konv}} \lim_{n\toinf} \int_{}^{} g_n \dif \mu\\
                &= \liminf_{n\toinf} \int_{}^{} g_n \dif \mu \leq \liminf_{n\toinf} \int_{}^{} f_n \dif \mu\qedhere
            \end{align*}
        \end{proof}
    \end{lemma}

    \begin{lemma}[Leicht verallgemeinertes Fatou]
        \label{lemma:fatou2}
        Sei $g\geq 0$ eine $\mu$-integrierbare Funktion und $(f_n)_n$ eine Folge von meßbaren numerischen Funktionen mit $f_{n} \geq -g$ $\mu$-fs. für $n\in\N$ und $f_n \leq f_{n+1}$. Dann sind $f_n$ und $f \coloneqq \liminf_{n\toinf} f_n$ quasi-integrierbar (d.h. $ \int_{}^{} (f_n)_- \dif \mu < \infty$ und $ \int_{}^{} f_- \dif \mu< \infty$) und es ist
        \begin{align*}
            -\infty < \int_{}^{} \liminf_{n\toinf} f_n \dif \mu &\leq \liminf_{n\toinf} \int_{}^{} f_n \dif \mu \leq \infty
        \end{align*}

        \begin{proof}
            Da $(f_n)_- \leq g$ und $f_- \leq g$ $\mu$-fs. folgt $f_n$ und $f$ sind quasi-integrierbar. Das heißt es existiert $N_n$ mit $\mu\of{N_n} = 0$ und $f_n\of{x}\leq f_{n+1}\of{x}~x\in\comp{N_n}$. Es sei $h_n \coloneqq f_n + g \geq 0 \impl h_n \leq h_{n+1}$ fast überall. Das heißt
            \begin{align*}
                \liminf_{n\toinf} h_n &= \liminf_{n\toinf} \pair{f_n + g} = \pair{\liminf_{n\toinf} h_n} + g = f+ g \geq 0\\
                N &= \bigcup_{n} N_n \impl \mu\of{N} \leq \sum_{j=1}^{\infty} \mu\of{N_n} = C
                \intertext{und}
                f_n\of{x} &\leq f_{n+1}\of{x}\quad\forall x\in\comp{N},~n\in\N
                \intertext{und}
                \mu\of{M} &= 0\quad f \geq -g \text{ auf }\comp{M}\\
                \mu\of{M_n} &= 0\quad f_n \geq -g \text{ auf }\comp{M}\\
                \tilde{M} &= M \cup \bigcup_{n\in\N} M_n \cup N\impl \mu\of{\tilde{M}} = 0
                \intertext{und}
                f_n &\leq f_{n+1}\quad f_n \geq -g~f \geq -g \text{ auf }\comp{M}
                \intertext{Redefiniere $f_n,f$ auf $f_n\of{x} = f\of{x} = 0$ für $x\in M$. Nach Lemma~\ref{lemma:fatou} gilt dann}
                \int_{}^{} \liminf_{n\toinf} \pair{f_n+g} \dif \mu &\leq \liminf_{n\toinf} \int_{}^{} \pair{f_n+g} \dif \mu\\
                \impl \int_{}^{} \liminf_{n\toinf} f_n \dif \mu + \int_{}^{} g \dif \mu &\leq \liminf_{n\toinf} \int_{}^{} f_n \dif \mu + \int_{}^{} g \dif \mu\\
                \impl \int_{}^{} \liminf_{n\toinf} f_n \dif \mu &\leq \liminf_{n\toinf} \int_{}^{} f_n \dif \mu\qedhere
            \end{align*}
        \end{proof}
    \end{lemma}

    \begin{korollar}[$\limsup$-Version von Fatou] % Korollar 3
        \label{korollar:fatou3}
        Sei $g\geq 0$ integrierbar und $(f_n)_n$ eine Folge messbarer numerischer Funktionen mit $f_n \leq g$ $\mu$-fast-überall. Dann gilt $f \coloneqq \limsup_{n\toinf} f_n$ ist quasi-integrierbar und
        \begin{align*}
            \int_{}^{} f \dif \mu &= \int_{}^{} \limsup_{n\toinf} f \dif \mu \geq \limsup_{n\toinf} \int_{}^{} f_n \dif \mu
        \end{align*}

        \begin{proof}
            Wende Lemma~\ref{lemma:fatou2} auf $h_n \coloneqq g - f_n$ an und $\liminf_{n\toinf} h_n = g - \limsup_{n\toinf} f_n$
            \begin{align*}
                \impl \int_{}^{} \liminf_{n\toinf} h_n \dif \mu &\leq \liminf \int_{}^{} h_n \dif \mu = \int_{}^{} g \dif \mu - \limsup_{n\toinf} \int_{}^{} f_n \dif \mu\\
                \impl \int_{}^{} g \dif \mu - \int_{}^{} f \dif \mu &\leq \int_{}^{} g \dif \mu - \limsup_{n\toinf} \int_{}^{} f_n \dif \mu\\
                \impl \limsup_{n\toinf} \int_{}^{} f_n \dif \mu &\leq \int_{}^{} \limsup_{n\toinf} f_n \dif \mu\qedhere
            \end{align*}
        \end{proof}
    \end{korollar}

    \subsection{Majorante Konvergenz (Lebesgue)}

    \begin{satz} % Satz 4
        \label{satz:majorante-konv}
        Seien $f, f_n: X \to \overline{\R}$ meßbare Funktionen und $\lim_{n\toinf} f_n = f$ $\mu$-fast-überall. und es gebe eine $\mu$-integrierbare Funktion $g\geq 0$ mit $\abs{f_n} \leq g$ $\mu$-fast-überall. Dann sind $f_n, f$ integrierbar und
        \begin{align*}
            \lim_{n\toinf} \int_{}^{} f_n \dif \mu &= \int_{}^{} f \dif \mu\tag{1}\\
            \lim_{n\toinf} \int_{}^{} \abs{f_n} \dif \mu &= \int_{}^{} \abs{f} \dif \mu\tag{1'}\\
            \lim_{n\toinf} \int_{}^{} \abs{f-f_n} \dif \mu &= 0\tag{2}
        \end{align*}

        \begin{proof}
            \begin{align*}
                \abs{\int_{}^{} f \dif \mu - \int_{}^{} f_n \dif \mu} &= \abs{\int_{}^{} \pair{f-f_n} \dif \mu}\\
                &\leq \int_{}^{} \abs{f-f_n} \dif \mu \to 0\text{ für }n\toinf\\
                \abs{ \int_{}^{} \abs{f} \dif \mu - \int_{}^{} \abs{f_n} \dif \mu} &= \abs{ \int_{}^{} \pair{\abs{f} - \abs{f_n}} \dif \mu}\\
                &\leq \int_{}^{} \abs{f} - \abs{f_n} \dif \mu \leq \int_{}^{} \abs{f-f_n} \dif \mu
            \end{align*}
            Das heißt es gelten (1) und (1'), sofern (2) gilt. Wir beweisen zusätzlich noch (2)
            \begin{align*}
                0 \leq \int_{}^{} \abs{f-f_n} \dif \mu\\
                0 \leq h_n &= \abs{f-f_n} \leq \abs{f} + \abs{f_n} \leq \abs{f} + \abs{g} \leq g + g = 2 g\text{ $\mu$-fs.}
                \intertext{Nach Korollar~\ref{korollar:fatou3} gilt dann}
                \limsup_{n\toinf} \int_{}^{} h_n \dif \mu &\leq \int_{}^{} \limsup_{n\toinf} h_n \dif \mu\\
                &= \int_{}^{} \limsup_{n\toinf} \abs{f-f_n} \dif \mu = 0\text{ $\mu$-fs.}\\
                \impl \limsup_{n\toinf}\int_{}^{} \abs{f-f_n} \dif \mu &= 0\qedhere
            \end{align*}
        \end{proof}
    \end{satz}

    \begin{bemerkung}
        Der vorherigen Satz gilt auch für komplexwertige Funktionen.
    \end{bemerkung}

    \begin{satz} % Satz 7
        \marginnote{[17. Jan]}
        Seien $f_n, f: X \to\overline{\R}$ (oder $\C$) integrierbar mit $f_n \to f$ $\mu$-fs. Dann gilt
        \begin{align*}
            \lim_{n\toinf}\int_{}^{} \abs{f_n} \dif \mu &= \int_{}^{} \abs{f} \dif \mu
            \intertext{genau dann, wenn}
            \lim_{n\toinf} \int_{}^{} \abs{f-f_n} \dif \mu &= 0
        \end{align*}

        \begin{proof}
            \anf{$\Leftarrow$}:
            \begin{align*}
                \abs{ \int_{}^{} \abs{f} \dif \mu - \int_{}^{} \abs{f_n} \dif \mu} &= \abs{ \int_{}^{} \abs{f} - \abs{f_n} \dif \mu}\\
                &\leq \int_{}^{} \abs{\abs{f} - \abs{f_n}} \dif \mu \leq \int_{}^{} \abs{f-f_n} \dif \mu \to 0
            \end{align*}
            \anf{$\impl$}:
            Sei $g_n \coloneqq \abs{f_n} + \abs{f} - \abs{f-f_n} \geq 0$  $\mu$-fs. Dann folgt mit Lemma~\ref{lemma:fatou}
            \begin{align*}
                \int_{}^{} \liminf_{n\toinf} g_n \dif \mu &\leq \liminf_{n\toinf} \int_{}^{} g_n \dif \mu\\
                \liminf_{n\toinf} g_n &= 2\abs{f} - 0 = 2\abs{f}\\
                \int_{}^{} g_n \dif \mu &= \int_{}^{} \abs{f_n} \dif \mu + \int_{}^{} \abs{f} \dif \mu - \int_{}^{} \abs{f-f_n} \dif \mu\\
                \impl \liminf_{n\toinf} \int_{}^{} g_n \dif \mu &= 2 \int_{}^{} \abs{f} \dif \mu - \limsup_{n\toinf} \int_{}^{} \abs{f-f_n} \dif \mu\\
                \impl 0 &\leq \limsup_{n\toinf} \int_{}^{} \abs{f-f_n} \dif \mu \leq 0\qedhere
            \end{align*}
        \end{proof}
    \end{satz}

    \begin{satz} % Satz 8
        Seien $f, f_n: X\to \overline{\R}$ (oder $C$) meßbar und es gebe $g\geq 0$ $\mu$-integrierbar mit $\displaystyle \abs{ \sum_{k=1}^{n} f_k} \leq g$ $\mu$-fs. Ferner existiere $f \coloneqq \sum_{n=1}^{\infty} f_n = \lim_{L\toinf} \sum_{n=1}^{L} f_n$ $\mu$-fs. Dann sind $f_n$ und $f$ integrierbar und
        \begin{align*}
            \int_{}^{} f \dif\mu &= \sum_{n=1}^{\infty} \int_{}^{} f_n \dif \mu
        \end{align*}
        \begin{proof}
        (Übung)
        \end{proof}
    \end{satz}

    \newpage


    \section{[*] Anwendung der Konvergenzsätze}

    \subsection{Parameter-abhängige Integrale}
    \thispagestyle{pagenumberonly}

    \begin{satz} % Satz 1
        Sei $T$ ein metrischer Raum, $\pair{X, \mA, \mu}$ ein Maßraum und $f: T \times X \to\overline{\R}$ (oder $\C$) eine Funktion mit
        \begin{enumerate}[label=(\roman*)]
            \item $x\mapsto f\of{t, x}$ ist $\mu$-integrierbar $\forall t\in T$
            \item $t \mapsto f\of{t, x}$ ist stetig in $t_0$ für $\mu$-fast-alle $x\in X$
            \item Es existiert eine $\mu$-integrierbare Funktion $g\geq 0$, sodass $\abs{f\of{t, \cdot}}\leq g$ $\mu$-fs. $\forall t\in T$
        \end{enumerate}
        Dann gilt
        \begin{align*}
            \varphi: T &\to\R\\
            t &\mapsto \int_{}^{} f\of{t, x} \mu\of{\dif x}
        \end{align*}
        ist stetig in $t_0$.

        \begin{proof}
            Zu zeigen: Sei $(t_n)_n \subseteq T$ mit $t_n \to t_0$, dann muss folgen $\varphi\of{t_n} \to \varphi\of{t_0}$.\\
            Wir haben Nullmenge $N_1$, sodass $f\of{t_n, x}\to f\of{t_0, x}$ für $x\in\comp{N_1}$ und wir haben Nullmenge $M_n$, sodass $\abs{f\of{t_n, x}}\leq g\of{x}~\forall x\in\comp{M_n}$. Das heißt für $M \coloneqq = \bigcup_{n} M_n$ gilt $\abs{f\of{t_n, x}} \leq g\of{x}~\forall x \in\comp{M}$. Wir definieren $\tilde{N} \coloneqq M \cup N$ auch Nullmenge und beide obigen Folgerungen gelten für alle $n\in\N, x\in\comp{\tilde{N}}$. Das heißt nach Satz~\ref{satz:majorante-konv} folgt mit $f_n \coloneqq f\of{t_n, \cdot}$
            \begin{align*}
                \lim_{n\toinf} \int_{}^{} f_n \dif \mu &= \int_{}^{} f \dif \mu = \int_{}^{} f\of{t_0, x} \mu\of{\dif x} = \varphi\of{t_0} = \lim_{n\toinf} \int_{}^{} f\of{t_n, x} \mu\of{\dif x}\qedhere
            \end{align*}
        \end{proof}
    \end{satz}

    \begin{lemma}[Differentiationslemma] % Lemma 2
        \label{lemma.differentiationslemma}
        Sei $I \subseteq\R$ eine nicht-ausgeartetes Intervall (das heißt weder leer noch einpunktig) und $f: I \times X \to\R$ (oder $\C$) eine Funktion mit
        \begin{enumerate}[label=(\roman*)]
            \item $f\of{t, \cdot}$ ist $\mu$-integrierbar $\forall t\in I$
            \item $t\mapsto f\of{t, x}$ ist differenzierbar auf $I$ für alle $x\in X$
            \item Es existiert eine $\mu$-integrierbare Funktion $g\geq 0$ mit $\abs{f'\of{t, \cdot}} \leq g$ $\mu$-fs. $\forall t\in I$ (das heißt es existiert \textit{eine} Nullmenge $N\in\mA$ mit $\abs{f'\of{t, x}} \leq g\of{x}~\forall x\in\comp{N}, t\in I$)
        \end{enumerate}
        Dann ist $\varphi\of{t} \coloneqq \int_{}^{} f\of{t, x} \mu\of{\dif x}$ differenzierbar auf $I$ und $\forall t\in I$ ist $f'\of{t, \cdot}$ $\mu$-integrierbar mit
        \begin{align*}
            \varphi'\of{t} &= \int_{}^{} f'\of{t, x} \mu\of{\dif x}
        \end{align*}

        \begin{proof}
            Sei $t_0\in I$ beliebig und $(t_n)_n \subseteq I$ eine Folge mit $t_n \to t_0$, wobei $t_n \neq t_0~\forall n\in\N$. Dann soll gelten
            \begin{align*}
                h_n\of{x} \coloneqq \frac{f\of{t_n, x} - f\of{t_0, x}}{t_n - t_0}\to f'\of{t_0, x}
            \end{align*}
            Nach Voraussetzung ist $h_n$ $\mu$-integrierbar.\\
            Aus dem Mittelwertsatz der Differentialrechnung folgt
            \begin{align*}
                \frac{f\of{t, x} - f\of{t_0, x}}{t-t_0} &= f'\of{\xi_t, x}
                \intertext{für ein $\xi_t$ zwischen $t_0$ und $t$}
                \impl \abs{ \frac{f\of{t_n, x} - f\of{t_0, x}}{t_n - t_0}} &= \abs{f'\of{\xi_{t_n}, x}} \leq g\of{x}
                \intertext{Nach Satz~\ref{satz:majorante-konv} folgt}
                \lim_{n\toinf} \int_{}^{} h_n \dif \mu &= \int_{}^{} \lim_{n\toinf} h_n \dif \mu = \int_{}^{} f'\of{t_0, x} \dif \mu\\
                \impl \varphi\text{ ist differenzierbar in }t_0\text{ und } \varphi'\of{t_0} &= \lim_{n\toinf} \frac{\varphi\of{t_n} - \varphi\of{t_0}}{t_n-t_0} = \int_{}^{} f'\of{t_0, x} \mu\of{\dif x}\qedhere
            \end{align*}
        \end{proof}
    \end{lemma}

    \begin{korollar}[Mehrdimensionaler Fall] % Korollar 3
        Sei $U\subseteq\R^d$ offen und $f: U \times X \to \overline{\R}$ (oder $\C$) mit
        \begin{enumerate}[label=(\roman*)]
            \item $f\of{t, \cdot}$ ist $\mu$-integrierbar $\forall t\in U$
            \item $t\mapsto f\of{t, x}$ ist auf $U$ partiell differenzierbar $\mu$-fs. in der Variablen $t_j~\forall x\in X$
            \item Es existiert eine $\mu$-integrierbare Funktion $g\geq 0$ mit
            \begin{align*}
                \abs{ \frac{\partial f}{\partial t_j}\of{t, x}} &\leq g\of{x}\text{ $\mu$-fs. für alle $x\in X$ und $t\in U$}
            \end{align*}
        \end{enumerate}
        Dann ist $\varphi\of{t}\coloneqq \int_{}^{} f\of{t, x} \mu\of{\dif x}$ partiell differenzierbar bezüglich $t_j$ und
        \begin{align*}
            \frac{\partial \varphi}{\partial t_j}\of{x} &= \int_{}^{} \frac{\partial f}{\partial t_j}\of{t, x} \mu\of{\dif x}
        \end{align*}
        \begin{proof}
            Halte alle Variablen bis auf $t_j$ und $x$ fest und wende Lemma~\ref{lemma.differentiationslemma} an.
        \end{proof}
    \end{korollar}

    \subsection{Vergleich des Riemann- mit dem Lebesgue-Integral}

    \begin{satz} % Satz 4
        Sei $f: \interv{a,b} \to \K\in\set{\R, \C}$ differenzierbar und $f'$ beschränkt. Dann ist $f'$ Lebesgue-integrierbar auf $\interv{a,b}$ und
        \begin{align*}
            \int_{a}^{b} f' \dif \lambda &= f\of{b} - f\of{a}
        \end{align*}
        \textbf{Warnung:} $f'$ braucht hierfür nicht Riemann-integrierbar zu sein.

        \begin{proof}
            \marginnote{[20. Jan]}
            (Übung)
        \end{proof}
    \end{satz}

    \begin{satz} % Satz 5
        Sei $f: \interv{a,b}\to \R$ Borel-meßbar. Ist $f$ Riemann-integrierbar (also auch beschränkt), so ist $f$ Lebesgue-integrierbar mit
        \begin{align*}
            \int_{a}^{b} f \dif x &= \int_{a}^{b} f \dif \lambda
        \end{align*}

        \begin{proof}
        (Übung)
        \end{proof}
    \end{satz}

    \begin{korollar} % Korollar 6
        Sei $f\geq 0$ Borel-meßbar auf $\pair{a,b}$ und Riemann-integrierbar über den kompakten Teilintervallen von $\linterv{a,b}$. Dann ist $f$ genau dann Lebesgue-integrierbar über $\pair{a,b}$, wenn das uneigentliche Riemann-integral
        \begin{align*}
            \lim_{\tilde{a}\searrow a,~\tilde{b}\nearrow b} \int_{\tilde{a}}^{\tilde{b}} f \dif x
        \end{align*}
        existiert und das Riemann-Integral und Lebesgue-Integral gleich sind.\\
        \textbf{Warnung}:
        \begin{align*}
            \int_{0}^{\infty} \frac{\sin\of{x}}{x} \dif x &= \lim_{b\toinf} \int_{0}^{b} \frac{\sin\of{x}}{x} \dif x
            \intertext{existiert, aber ist nicht Lebesgue-integrierbar über $\linterv{0, \infty}$, denn}
            \int_{1}^{\infty} \frac{\pair{\sin\of{x}}_+}{x} \dif \lambda\of{\dif x} &= \infty\\
            \int_{1}^{\infty} \frac{\pair{\sin\of{x}}_-}{x} \dif \lambda\of{\dif x} &= \infty\\
        \end{align*}

        \begin{proof}
        (Übung)
        \end{proof}
    \end{korollar}

    \subsection{Berechnung von $ \int_{\R}^{} e^{-x^2} \lambda\of{\dif x}$}

    \begin{anwendung}
        Wir setzen
        \begin{align*}
            f\of{t, x} &\coloneqq \frac{e^{-t\of{1+x^2}}}{1+x^2}\tag{$x\in\R$, $t \geq 0$}
            \intertext{und wählen als Ansatz}
            a^2 + b^2 &\geq 2ab\tag{$a,b\geq 0$}\\
            \impl 1+x^2 &\geq 2\abs{x}\geq \abs{x}\\
            \impl e^{-t\pair{1+x^2}} &\leq e^{-t\abs{x}}\\
            \impl 0 \leq f\of{t, x} &\leq e^{-t\abs{x}}\\
            0\leq \int_{a}^{b} f\of{t, x} \lambda\of{\dif x} &\leq \int_{0}^{n} e^{-t\abs{x}} \lambda\of{\dif x}\\
            &= \int_{0}^{n} e^{-t\abs{x}} \dif x = \frac{1}{t}\interv{-e^{-tx}}_{0}^{n}\\
            &= \frac{1}{t}\interv{1-e^{-tn}} \to \frac{1}{t}\\
            \impl \int_{0}^{\infty} f\of{t, x} \lambda\of{\dif x}&\text{ existiert für } t > 0\\
            \intertext{Genauso zeigt sich}
            \impl \int_{-\infty}^{0} f\of{t, x} \lambda\of{\dif x}&\text{ existiert für } t > 0\\
            \intertext{Wir betrachten noch den Fall $t=0$. Dann ist}
            f\of{0, x} &= \frac{1}{1+x^2} \leq \frac{1}{x^2}\\
            \impl f&\text{ ist Lebesgue-integrierbar auf $\R$ für $t=0$}
            \intertext{Sei $x\in\R$ fest, dann ist $0\leq t \mapsto f\of{t, x}$ stetig und $0\leq f\of{t, x} \leq \frac{1}{1+x^2}$ ist eine integrierbare Majorante}
            \varphi\of{t} &= \int_{\R}^{} f\of{t,x} \lambda\of{\dif x}\text{ ist stetig über }t\geq 0
            \intertext{und $f\of{t, x}$ ist differenzierbar für $t > 0$. Für $t_0 > 0$, $t \geq t_0$}
            \partial_t f\of{t, x} &= \frac{-1}{1+x^2} \pair{1+x^2} e^{-t\pair{1+x^2}}\\
            &= e^{-t\pair{1+x^2}}\\
            \abs{\partial_t f\of{t, x}} &= e^{-t\pair{1+x^2}} \leq e^{-t\abs{x}} \leq e^{-t_0\abs{x}}\text{ ist Majorante}
            \intertext{Diffbarkeit: Für $t > t_0$ ist $\varphi$ diff.-bar mit}
            \varphi'\of{t} &= \int_{\R}^{} e^{-t\pair{1+x^2}} \lambda\of{\dif x}
            \intertext{Da $t_0 > 0$ beliebig, ist $\varphi$ diffbar $\forall t > 0$. $t> 0$}
            \varphi'\of{t} &= -\int_{-\infty}^{\infty} e^{-t\pair{1+x^2}} \lambda\of{\dif x}\\
            &= - \int_{-\infty}^{\infty} e^{-t\pair{1+x^2}} \dif x\\
            &= -\lim_{n\toinf} \int_{-n}^{\infty} e^{-t\pair{1+x^2}} \dif x
            \intertext{Substitution $x = t^{-\frac{1}{2}} y$}
            \int_{-n}^{n} e^{-t\pair{1+x^2}} \dif x &= t^{-1\frac{1}{2}} \int_{-n\sqrt{t}}^{n\sqrt{t}} e^{-t\pair{1+t^{-1}y}} \dif y\\
            &= t^{-\frac{1}{2}} e^{-t} \int_{-n\sqrt{t}}^{n\sqrt{t}} e^{-y^2} \dif y\to t^{-\frac{1}{2}} e^{-t}\int_{-\infty}^{\infty} e^{-y^2} \dif y\\
            &\eqqcolon -Gt^{-\frac{1}{2}}e^{-t}\\
            \varphi'\of{t} &= -Gt^{-\frac{1}{2}}e^{-t}
            \intertext{Für $0 < \alpha < s$}
            \varphi\of{x} - \varphi\of{\alpha} &= \int_{\alpha}^{s} \varphi'\of{t} \dif t\\
            &= -G \int_{\alpha}^{s} e^{-t}t^{-\frac{1}{2}} \dif t
            \intertext{Substitution $t = r^2$, $\dif t = 2r\dif r$}
            \int_{\alpha}^{s} e^{t}t^{-\frac{1}{2}} \dif t &= 2 \int_{\sqrt{\alpha}}^{\sqrt{s}} e^{-r^2}t^{-1}2r \dif r\\
            \varphi\of{s} - \varphi\of{\alpha} &= -2G \int_{\sqrt{\alpha}}^{\sqrt{s}} e^{-r^2} \dif r\\
            \impl \varphi\of{s} - \varphi\of{0} &= \lim_{\alpha \searrow 0} \pair{\varphi\of{s} - \varphi\of{\alpha}}\\
            &= -2G \lim_{\alpha \searrow 0} \int_{\sqrt{\alpha}}^{\sqrt{s}} e^{-t^2} \lambda\of{\dif t}\\
            &= -2G \int_{0}^{\sqrt{s}} e^{-t^2} \lambda\of{\dif t}\\
            \impl \lim_{s\toinf} \pair{\varphi\of{s} - \varphi\of{0}} &= -2 G \lim_{s\toinf} \int_{0}^{\sqrt{s}} e^{-t^2} \lambda\of{\dif t}\\
            &= -2 G \int_{0}^{\infty} e^{-t^2} \lambda\of{\dif x} = -G \int_{-\infty}^{\infty} e^{-r^2} \lambda\of{\dif r} = -G^2\\
            \impl \lim_{s\toinf} \varphi\of{s} &= 0\\
            \impl t\varphi\of{0} &= t G^2\\
            0 \leq \varphi\of{s} &= \int_{\R}^{} \frac{e^{-s\pair{1+x^2}}}{1+x^2} \lambda\of{\dif x}\\
            \int_{-n}^{n} \frac{1}{1+x^2} \dif x &= \arctan\of{n} - \arctan\of{-n} = 2\arctan\of{n}\\
            \frac{\dif}{\dif x} \arctan\of{x} &= \frac{1}{1+x^2}\\
            \impl \int_{\R}^{} e^{-x^2} \lambda\of{\dif x} &= \sqrt{\pi}
        \end{align*}
    \end{anwendung}

    \newpage


    \section{[*] Die $\mL^{p}$-Räume}

    \subsection{Definitino der $\mL^{p}$-Räume}
    \thispagestyle{pagenumberonly}
    Sei $p\geq 1$ und $f: X \to\overline{\R}$ (oder $\C$) meßbar sowie $\pair{X, \mA}$ ein Messraum. Dann ist auch $\abs{f}$ meßbar. Das heißt $\abs{f}^p$ ist meßbar und $\set{\abs{f}^p > \alpha} = \set{\abs{f} > \alpha^{\frac{1}{p}}}$ für $\alpha \geq 0$.

    \begin{notation}
        Sei $\pair{X, \mA, \mu}$ ein Maßraum. Wir schreiben
        \begin{align*}
            N_p\of{f} &\coloneqq \pair{ \int_{X}^{} \abs{f}^p \dif \mu}^{\frac{1}{p}}
        \end{align*}
        Es gilt $0\leq N_p\of{f} \leq \infty$. Für $\alpha\in\R$ gilt $N_p\of{\alpha f} = \abs{\alpha} N_p\of{f}$
    \end{notation}

    \begin{satz}[Hölder-Ungleichung] % Satz 1
        Sei $p > 1$ und $1 < q < \infty$ dual zu $p$ (das heißt $\frac{1}{p}+\frac{1}{q} = 1$). Dann gilt für zwei meßbare numerische Funktionen $f,g$ auf $X$
        \begin{align*}
            N_1\of{fg} &\leq N_p\of{f}N_q\of{g}
        \end{align*}
        \begin{proof}
            Benutzt die Ungleichung von Young. Für $a, b \geq 0$ folgt $ab \leq \frac{1}{p}a^p + \frac{1}{q}b^q$.

            \begin{proof}[Beweis von der Ungleichung von Young]
                Sei $G\of{a} \coloneqq \frac{1}{p}a^p$. Wir wollen eine Funktion $F$, sodass $ab \leq G\of{a} + F\of{b}~\forall a,b\geq 0$. Das ist äquivalent zu
                \begin{align*}
                    F\of{b} &\geq ab - G\of{a}\quad\forall a\geq 0, b\geq 0\text{ fest}
                    \intertext{Die beste Wahl von $F$ ist die kleinste obere Schranke. Das heißt}
                    F\of{b} &= \sup_{a\geq 0} \pair{ab- G\of{a}}\\
                    \partial_a\of{ab - G\of{a}} &= b- G'\of{a} = b-a^{p-1}\\
                    \impl F\of{b} &= b\cdot b^{\frac{1}{p-1}} - \frac{1}{p}\pair{b^{\frac{1}{q-1}}}^p\\
                    &= b^{\frac{1}{p-1}+1} - \frac{1}{p}b^{\frac{p}{p-1}} = \frac{1}{q}b^{q}\qedhere
                \end{align*}
            \end{proof}
            \marginnote{[23. Jan]}
            \noindent Wir setzen $\sigma = N_p\of{f}$ und $\tau = N_q\of{g}$. Im Fall $\tau = 0$ folgt direkt, dass $N_1\of{fg} = 0$. In diesem Fall ist die Ungleichung erfüllt. Es sei also O.B.d.A. $\sigma > 0$ und $\tau < \infty$ (da die Ungleichung sonst trivialerweise erfüllt ist). Sei $\tilde{f} = \frac{f}{\tau}$, $\tilde{g} = \frac{g}{\sigma}$. Dann ist $N_p\of{\tilde{f}} = N_q\of{\tilde{g}} = 1$. Wir müsen also noch zeigen, dass $N_1\of{\tilde{f}\tilde{g}} \leq 1$ (Skalierungstrick). Es ist
            \begin{align*}
                \abs{\tilde{f}\tilde{g}} &= \abs{\tilde{f}}\abs{\tilde{g}} \leq \frac{1}{p}\abs{\tilde{f}}^p + \frac{1}{q}\abs{\tilde{g}}^q\\
                \impl N_1\of{\tilde{f}\tilde{g}} &= \int_{X}^{} \abs{\tilde{f}\tilde{g}} \dif \mu \leq \int_{X}^{} \pair{\frac{1}{p}\abs{\tilde{f}}^p + \frac{1}{q}\abs{\tilde{g}}^q} \dif \mu\\
                &= \frac{1}{p} \int_{X}^{} \abs{\tilde{f}}^p \dif \mu + \frac{1}{q} \int_{X}^{} \abs{\tilde{g}}^q \dif \mu\\
                &= \frac{1}{p}N_p\of{\tilde{f}}^p + \frac{1}{q}N_q\of{\tilde{g}}^q = \frac{1}{p}1^p + \frac{1}{q}1^q = 1\qedhere
            \end{align*}
        \end{proof}
    \end{satz}

    \begin{satz} % Satz 2
        Es seien $f, g: X \to\overline{\R}$ (oder $\C$) zwei numerische Funktionen, für die $f+g$ definiert ist und $1\leq p < \infty$. Dann gilt
        \begin{align*}
            N_p\of{f+g} &\leq N_p\of{f} + N_p\of{g}\tag{Dreiecksungleichung}
        \end{align*}

        \begin{proof}
            \begin{align*}
                \abs{f+g}^p &\leq \pair{\abs{f} + \abs{g}}^p \leq \pair{2\pair{\abs{f} \lor \abs{g}}}^p\\
                &= 2^p\pair{\abs{f} \lor \abs{g}} = 2^p\pair{\abs{f}^p \lor \abs{g}^p}  \leq 2^p\pair{\abs{f}^p + \abs{g}^p}\\
                \impl N_p\of{f+g}^p &= \int_{X}^{} \abs{f+g}^p \dif \mu \leq 2^p\pair{N_p\of{f}^p + N_p\of{g}^p}
                \intertext{Ist $N_p\of{f}, N_p\of{g} < \infty$, folgt $N_p\of{f+g} < \infty$}
                \abs{f+g}^p &= \abs{f+g}\abs{f+g}^{p-1} \leq \abs{f}\abs{f+g}^{p-1} + \abs{g}\abs{f+g}^{p-1}\\
                \impl N_p\of{f+g}^p &= \int_{}^{} \abs{f+g}^p \dif \mu \leq \int_{}^{} \abs{f}\abs{f+g}^{p-1} \dif \mu + \int_{}^{} \abs{g}\abs{f+g}^{p-1} \dif \mu
                \intertext{Es sei $\frac{1}{p}+ \frac{1}{q} = 1$. Dann folgt nach Hölder}
                &\leq \pair{ \int_{}^{} \abs{f}^p \dif \mu}^{\frac{1}{p}}\pair{ \int_{}^{} \abs{f+g}^{p-1} \dif \mu}^{\frac{1}{q}} + \pair{ \int_{}^{} \abs{g}^p \dif \mu}^{\frac{1}{p}}\pair{ \int_{}^{} \abs{f+g}^{p-1} \dif \mu}^{\frac{1}{q}}\\
                &= N_p\of{f}N_p\of{f+g}^{\frac{p}{q}} + N_p\of{g}N_p\of{f+g}^{\frac{p}{q}} = \pair{N_p\of{f} + N_p\of{g}}N_p\of{f+g}^{p-1}\\
                \impl N_p\of{f+g} &= N_p\of{f+g}^p N_p\of{f+g}^{1-p} \leq N_p\of{f} + N_p\of{g}\qedhere
            \end{align*}
        \end{proof}
    \end{satz}

    \begin{bemerkung}
        Ist $N_p\of{f} < \infty$ und $N_{p}\of{g} < \infty$, so ist $f, g$ fast-überall reellwertig und $f+g$ fast-überall definiert.
    \end{bemerkung}

    \begin{definition}
        Sei $f: X \to\overline{\R}$ (oder $\C$) meßbar. Dann heißt $f$ $p$-fach integrierbar für $1\leq p < \infty$, wenn $N_p\of{f} < \infty$. Wir definieren
        \begin{align*}
            \mL^p\of{\mu} = \mL^p\of{X, \mA, \mu} = \set{\text{messbare Funktion }f: X \to\overline{\R}: N_p\of{f}<\infty}
        \end{align*}
        Analog schreiben wir $\mL_{\C}^p$ für den Raum solcher komplexwertiger Funktionen. Die definierten Räume sind Vektorräume. Zusätzlich gilt Homogenität und Dreiecksungleichung für $N_p$. Das heißt $N_p$ ist bereits eine Halbnorm. Aber $N_p\of{f} \impl f = 0$ gilt nur $\mu$-fast-überall. Das kann man beheben, indem man Äquivalenzklassen von $\mL^p$-Funktionen definiert, die $\mu$-fast-überall gleich sind.
    \end{definition}

    \begin{satz} % Satz 4
        Seien $f, g\in \mL^p\of{\mu}$. Dann gilt $\alpha f, f+g\in\mL^p\of{\mu}$ sowie $f\land g, f\lor g \in\mL_{\R}^p\of{\mu}$.

        \begin{proof}
        (Übung)
        \end{proof}
    \end{satz}

    \begin{korollar} % Korollar 5
        Sei $f: X \to\overline{\R}$ eine messbare, numerische Funktion. Dann ist $f\in\mL^p$ genau dann, wenn $f_+, f_- \in\mL^p$.

        \begin{proof}
            \anf{$\impl$}: $f_+ = f \lor 0$, $f_- = (-f) \lor 0$.\\
            \anf{$\Leftarrow$}: Sind $f_{\pm}\in\mL^p$. Dann gilt
            \begin{align*}
                f &= f_+ + \pair{-f_-}\\
                N_p\of{f} &\leq N_p\of{f_+} + N_p\of{f_-} < \infty\qedhere
            \end{align*}
        \end{proof}
    \end{korollar}

    \begin{satz}
        Sei $f\in\mL^p$, $g\in\mL^q$ sowie $\frac{1}{q}+\frac{1}{p} = 1$ mit $1 < p < \infty$. Dann ist $fg \in\mL^1$.
        \begin{proof}
            Hölder!
        \end{proof}
    \end{satz}

    \begin{bemerkung}
        Man schreibt auch häufig $\norm{f}_p = N_p\of{f}$.
    \end{bemerkung}

    \begin{korollar} % Korollar 6
        Sei $\mu\of{X} < \infty$. Dann folgt aus $f\in\mL^p\of{\mu} \impl \mL^1\of{\mu}$ (für $1\leq p < \infty$).

        \begin{proof}
            \begin{align*}
                \int_{X}^{} \abs{f} \dif \mu &= \int_{X}^{} 1\cdot\abs{f} \dif \mu\\
                \intertext{Nach Hölder}
                &\leq N_p\of{f}N_q\of{1} = \pair{\mu\of{X}}^{\frac{1}{q}}N_p\of{f}
                \intertext{$\frac{1}{p} + \frac{1}{q} = 1$}
                \impl N_1\of{f} &\leq \pair{\mu\of{x}}^{\frac{p-1}{p}}N_p\of{f}\qedhere
            \end{align*}
        \end{proof}
    \end{korollar}

    \begin{satz} % Satz 8
        Sei $f: X \to\overline{\R}$ (oder $\C$) eine $p$-fach integrierbare Funktion ($f\in\mL^p\of{\mu}$). $g$ sei fast-überall beschränkt, das heißt es existiert ein $\alpha > 0$ und eine Nullmenge $N$, sodass $\abs{g\of{x}} \leq \alpha~\forall x\in\comp{N}$. Dann folgt $gf$ ist $p$-fach integrierbar und $N_p\of{fg} \leq \alpha N_p\of{f}$.
        \begin{proof}
            Es gilt $\abs{fg} = \abs{f}\abs{g} \leq \abs{f}\alpha$ $\mu$-fast-überall
            \begin{align*}
                \impl N_p\of{fg}^p = \int_{}^{} \abs{fg}^p \dif \mu &\leq \int_{}^{} \alpha^p\abs{f}^p \dif \mu = \alpha^p \int_{}^{} \abs{f}^p \dif \mu = \alpha^p N_p\of{f}^p\qedhere
            \end{align*}
        \end{proof}
    \end{satz}

    \begin{notation}
        \begin{align*}
            \mL^{\infty}\of{\mu} \coloneqq \set{g: X \to\overline{\R} \text{ messbar}: \exists M_g \geq 0, N_g\in\mA: \mu\of{N_g} = 0\text{ sodass } \abs{g\of{x}} \leq M_g~\forall x\in\comp{N_g}}
        \end{align*}
    \end{notation}

    \begin{definition}
        Wir definieren $N_\infty\of{g}\coloneqq \inf\set{M\geq 0: \abs{g}\leq M ~\mu\text{-fast-überall}}$. Es gilt
        \begin{align*}
            N_1\of{fg} &\leq N_1\of{f}N_{\infty}\of{g}
            \intertext{Auch}
            N_{\infty}\of{f+g} &\leq N_{\infty}\of{f}+ N_{\infty}\of{g}
        \end{align*}
    \end{definition}

    \begin{konvention}
        Wenn wir die $\mL^p$-Räume betrachten, beschränken wir uns auf den Fall $1\leq p < \infty$, da für $p < 1$ Eigenschaften wie die Dreiecksungleichung verloren gehen und die resultierenden Räume sich daher von denen mit $p\geq 1$ unterscheiden.
    \end{konvention}

    \subsection{Konvergenzsätze}

    Sei $1\leq p < \infty$ und $\pair{X, \mA, \mu}$ ein Maßraum. $N_p$ ist auf $\mL^p\of{\mu}$ eine Halbnorm. Wir können also auch einen Abstandsbegriff definieren:
    \begin{align*}
        d_p\of{f,g} \coloneqq N_p\of{f-g} = \norm{f-g}_p
    \end{align*}
    Es gilt $d_p\of{f,g} = 0 \impl f=g$ $\mu$-fast-überall. Wir definieren außerdem Folgen $(f_n)_n \subset \mL^p\of{\mu}$. Wir sagen $f_n$ konvergiert im $p$-ten Mittel gegen $f$, wenn
    \begin{align*}
        \lim_{n\toinf} d_p\of{f_n - f} = 0
    \end{align*}
    Man sagt auch $f_n$ konvergiert in $\mL^p$ gegen $f$. Wir sagen außerdem, dass $(f_n)_n$ eine Cauchy-Folge in $\mL^p\of{\mu}$ ist, falls
    \begin{align*}
        \forall\varepsilon > 0\ex N\in\N\colon N_p\of{f_n-f_m} < \varepsilon\quad\forall m,n\geq N
    \end{align*}

    \begin{satz}
        \marginnote{[24. Jan]}
        Sei $(f_n)_n \subseteq \mL^p$ eine Folge für $1\leq p < \infty$ mit $f_n \to f$ und $\pair{X, \mA, \mu}$ ein Maßraum. Dann gilt für alle $A \in\mA$
        \begin{align*}
            \int_{A}^{} \abs{f_n}^p \dif \mu \to \int_{A}^{} \abs{f}^p \dif \mu
        \end{align*}
        \begin{proof}
            \begin{align*}
                \int_{A}^{} \abs{f_n}^p \dif \mu &= \int_{}^{} \charfunc_{A} \abs{f_n}^p \dif \mu\\
                &= \int_{}^{} \abs{\charfunc_A f_n}^p \dif \mu = N_p\of{\charfunc_{A} f_n}\\
                \impl \abs{N_p\of{\charfunc_A f_n} - N_p\of{\charfunc_A f}} &\leq N_p\of{\charfunc_{A}\pair{f_n - f}}\\
                &\leq N_p\of{f_n - f} \to 0\qedhere
            \end{align*}
        \end{proof}
    \end{satz}

    \begin{satz}[Riesz] % Satz 2
        Konvergiere $(f_n)_n \subseteq \mL^p$ $\mu$-fast-überall gegen $f\in\mL^p$. Dann gilt
        \begin{align*}
            \lim_{n\toinf} \int_{}^{} \abs{f_n}^p \dif \mu &= \int_{}^{} \abs{f}^p \dif \mu
            \intertext{genau dann, wenn}
            \lim_{n\toinf} N_p\of{f_n-f} &= 0
        \end{align*}

        \begin{proof}
            \anf{$\impl$}: siehe oben.\\
            \anf{$\Leftarrow$}: Schritt 1
            \begin{align*}
                \pair{\alpha+\beta}^p &\leq \pair{2\pair{\alpha \lor \beta}}^p = 2^p\pair{\alpha^p \lor \beta^p} \leq 2^p\pair{\alpha^p+\beta^p}\tag{$\alpha, \beta \geq 0$}
            \end{align*}
            Schritt 2: Es gilt
            \begin{align*}
                \abs{\alpha - \beta}^p &\leq \abs{\abs{\alpha} + \abs{\beta}}^p \leq 2^p\pair{\abs{\alpha}^p + \abs{\beta}^p}\\
                g_n &\coloneqq 2^p\pair{\abs{f_n}^p + \abs{f}^p} - \abs{f_n - f}^p\\
                \lim_{n\toinf} g_n &= 2^{p+1}\abs{f}^p
                \intertext{Nach Fatou gilt}
                2^{p+1} \int_{}^{} \abs{f}^p \dif \mu &= \int_{}^{} \liminf_{n\toinf} fg_n \dif \mu \leq \liminf_{n\toinf} \int_{}^{} g_n \dif \mu\\
                &= 2^p\pair{\lim_{n\toinf} \int_{}^{} \abs{f_n}^p \dif \mu + \int_{}^{} \abs{f}^p \dif \mu} - \limsup_{n\toinf} \pair{ \int_{}^{} \abs{f_n-f}^p \dif \mu}\\
                &~~~- 2^{p+1} \int_{}^{} \abs{f}^p \dif x - \limsup_{n\toinf} \pair{\ldots}\\
                \impl \limsup_{n\toinf} \int_{}^{} \abs{f_n-f}^p \dif \mu &= 0\qedhere
            \end{align*}
        \end{proof}
    \end{satz}

    \begin{lemma} % Lemma 3
        Sei $(f_n)_n \subseteq \mM_+\of{X, \mA}$. Dann gilt
        \begin{align*}
            N_p\of{ \sum_{n=1}^{\infty} f_n} \leq \sum_{n=1}^{\infty} N_p\of{f_n}
        \end{align*}

        \begin{proof}
            Sei $k\in\N$. Dann ist
            \begin{align*}
                N_p\of{ \sum_{n=1}^{k} f_n} &\leq \sum_{n=1}^{k} N_p\of{f_n} \leq \sum_{n=1}^{\infty} N_p\of{f_n}\\
                S_k &\coloneqq \sum_{n=1}^{k} f_n \text{ monoton wachsend}\\
                \lim_{k\toinf} N_p\of{S_k}^p &= \lim_{k\toinf}  \int_{}^{} \abs{S_k}^p \dif \mu = \int_{}^{} \lim_{k\toinf} \abs{S_k}^p \dif \mu\\
                &= \int_{}^{} \abs{ \sum_{n=1}^{\infty} f_n}^p \dif x = N_p\of{ \sum_{n=1}^{\infty} f_n}^p \qedhere
            \end{align*}
        \end{proof}
    \end{lemma}

    \begin{satz}[Maj. Konvergenz in $\mL^p$] % Satz 4
        Sei $(f_n)_n$ eine Folge $\mu$-fast-überall definierter Funktionen $f_n \in\mL^p\of{\mu}$. Es existiert ein $\mL^p\ni g\geq 0$ mit $\abs{f_n}\leq g$ $\mu$-fast-überall $\forall n\in\N$ und $\lim_{n\toinf} \abs{f_n}$ existiert $\mu$-fast-überall.\\
        Dann existiert ein $f\in\mL^p\of{\mu}$, sodass $f_n \to f$ $\mu$-fast-überall und $f_n$ konvergiert gegen $f$ in $\mL^p\of{\mu}$.

        \begin{proof}
            Entsprechend der Voraussetzungen existiert eine Nullmenge $N_1$, sodass $f\of{x} \coloneqq \lim_{n\toinf} f_n\of{x}$ existiert $\forall x\in\comp{N_1}$. Wir definieren außerdem $f\of{x} = 0~\forall x\in N_1$. Das heißt $\abs{f} \leq g$ $\mu$-fast-überall. Damit können wir folgern
            \begin{align*}
                N_p\of{f} &\leq N_p\of{g} < \infty\\
                \impl f&\in\mL^p\\
                h_n &\coloneqq \abs{f_n - f}^p \leq \pair{\abs{f_n} + \abs{f}}^p \leq \pair{g + \abs{f}}^p \leq 2^p\pair{g^p +\abs{f}^p}
                \intertext{$\mu$-fast-sicher. Das heißt es ist zu zeigen $ \int_{}^{} h_n \dif \mu \to 0$}
                g_n &\coloneqq 2^p\pair{g^p + \abs{f}^p} - h_n = 2^p\pair{g^p + \abs{f}^p} - \abs{f_n - f}^p\\
                \impl \liminf_{n\toinf} &= \lim_{n\toinf} g_n = 2^p\pair{g^p + \abs{f}^p} \text{ $\mu$-fas-überall}
                \intertext{Ferner}
                \int_{}^{} 2^p\pair{g^p+\abs{f}^p} \dif \mu &= \int_{}^{} \liminf_{n\toinf} g_n \dif \mu \leq \liminf_{n\toinf} \int_{}^{} g_n \dif \mu\\
                &= \int_{}^{} 2^p\pair{g^p+\abs{f}^p} \dif \mu - \limsup_{n\toinf} \int_{}^{} \abs{f_n-f} \dif \mu\\
                \impl 0 &\leq \limsup_{n\toinf} \int_{}^{} \abs{f_n-f}^p \dif \mu\leq 0\\
                \impl \lim_{n\toinf} N_p\of{f_n - f}^p &= 0\\
                \impl \lim_{n\toinf} N_p\of{f_n - f} &= 0\qedhere
            \end{align*}
        \end{proof}
    \end{satz}

    \begin{satz}[Vollständigkeit von $\mL^p$]
        Sei $1\leq p < \infty$. Jede Cauchy-Folge $(f_n)_n \subseteq \mL^p\of{\mu}$ konvergiert gegen ein $f\in\mL^p\of{\mu}$ im $p$-ten Mittel und eine geordnete Teilfolge $(f_{n_k})_k$ konvergiert $\mu$-fast-überall gegen $f$. Das gilt auch für den Fall $p=\infty$.

        \begin{proof}
            Sei $(f_n)_n$ eine Cauchy-Folge im $\mL^p$. Dann gilt
            \begin{align*}
                \lim_{n\toinf} \sup_{m\geq n} N_p\of{f_m - f_n} &= 0\tag{1}
            \end{align*}
            Schritt 1: Wir zeigen, dass eine Teilfolge $(f_{n_k})$ existiert, sodass $ \sum_{k=1}^{\infty} N_p\of{f_{n_{k-1}} - f_{n_k}} < \infty$. Nach (1) gilt
            \begin{align*}
                \exists n_1\in\N\colon \sup_{m\geq n_1} N_p\of{f_m - f_{n_1}} &\leq \frac{1}{2}\\
                \impl \exists n_2\in\N\colon \sup_{m\geq n_2} N_p\of{f_m - f_{n_2}} &\leq \frac{1}{2^2}\\
                &\vdots\\
                \impl \exists n_k\in\N\colon \sup_{m\geq n_k} N_p\of{f_m - f_{n_k}} &\leq \frac{1}{2^k}\\
                \\
                \impl N_p\of{f_{n_{k+1}} - f_{n_k}} &\leq \sup_{m\geq n_k} N_p\of{f_m - f_{n_k}}\\
                \impl \sum_{k=1}^{\infty} N_p\of{f_{n_k} - f_{n_k}} &\leq \sum_{k=1}^{\infty} \frac{1}{2^k} = 1
                \intertext{Schritt 2: Definiere}
                g_k &\coloneqq f_{n_{k+1}} - f_{n_k}\\
                g &\coloneqq \sum_{k=1}^{\infty} \abs{g_k}\\
                f_{n_{k+1}} &= g_k + f_{n_k} = g_k + g_{k-1} + f_{n_{k-1}}\\
                &= g_k + g_{k-1} + \ldots + g_1 + f_{n_1}\\
                &\leq N_p\of{ \sum_{k=1}^{\infty} \abs{g_k}} \leq \sum_{k=1}^{\infty} N_p\of{\abs{g_k}} \leq \sum_{k=1}^{\infty} \frac{1}{2^k} = 1\\
                \impl N_p\of{g} &\leq 1\\
                \impl g &\in \mL^p\\
                \impl g &\text{ ist $\mu$-fast-überall endlich}\\
                \impl \exists \text{Nullmenge }M\colon \sum_{n=1}^{\infty} \abs{g_k\of{x}} &= g\of{x} < \infty \quad\forall x\in\comp{M}\\
                \impl \sum_{k=1}^{\infty} g_k\of{x}&\text{ konvergiert}\quad\forall x\in\comp{M}\\
                \impl \lim_{L\toinf} \sum_{k=1}^{L} g_k\of{x}&\text{ existiert}\\
                \impl \lim_{L \toinf} f_{n_{L+1}}\of{x} &= \sum_{k=1}^{\infty} g_k\of{x} + f_{n_1}\of{x}\text{ existiert}\quad\forall x\in\comp{M}\\
                \impl (f_{n_k})_k&\text{ konvergiert }\mu\text{-fast-überall}
                \intertext{Schritt 3: Definiere $f\of{x} \coloneqq \lim_{k\toinf} f_{n_k}\of{x}$ existiert für $\mu$-fast-alle $x\in X$. Frage: Ist $f\in\mL^p$?}
            \end{align*}
        \end{proof}
    \end{satz}


    \marginnote{[27. Jan]}

    \newpage


    \section{[*] Produkte von $\sigma$-Algebren}

    \imaginarysubsection{Produkte von $\sigma$-Algebren}
    \thispagestyle{pagenumberonly}

    \marginnote{[30. Jan]}
    Situation: Wir haben endlich viele Messräume $\pair{X_j, \mA_j}$, $j\in\set{1, \ldots, n}$. Wir definieren ein Produkt
    \begin{align*}
        X &\coloneqq \prod_{j=1}^{n} X_j = X_1 \times X_2 \times \ldots \times X_n
        \intertext{und für ein $x=\pair{x_1, \ldots, x_n}\in X$ ist $x_j\in X_j$. Wir definieren außerdem \textit{Projektionsabbildungen}}
        p_j&: X \to X_j\\
        x &= \pair{x_1, \ldots, x_n} \mapsto x_j
        \intertext{Wir haben $\sigma$-Algebren $\mA_j$ und wollen nun auch eine $\sigma$-Algebra auf $X$ definieren. Versuch für $n=2$: Wir definieren die Elemente der $\sigma$-Algebra als $A_1 \times A_2$ mit $A_1 \in\mA_1$, $A_2\in\mA_2$. Allerdings ist}
        &\set{A_1 \times A_2: A_1\in\mA_1 \land A_2 \in\mA_2}
        \intertext{nur ein Halbring, aber im Allgemeinen keine $\sigma$-Algebra. Wir setzen also stattdessen}
        \bigotimes_{j=1}^n \mA_j &\coloneqq \mA_1 \otimes \mA_2 \otimes \ldots \otimes \mA_n \coloneqq \sigma\of{p_1, p_2, \ldots, p_n}\\
        &\coloneqq \text{ kleinste $\sigma$-Algebra auf $X=X_1 \times \ldots \times X_n$, sodass}\\
        &\quad\quad\text{alle Funktionen $p_j: X \to X_j$ $\mA-\mA_j$-messbar ist.}\\
    \end{align*}

    \begin{satz} % Satz 1
        \label{satz:prod-sigma}
        Sei für $j=1, \ldots, n$ $\xi_j$ ein Erzeuger von $\mA_j$ ($\mA_j = \sigma\of{\xi_j}$), welcher eine Folge von Mengen $(E_{jk})_{k\in\N}$ mit $E_{jk}\nearrow X_j$ enthält. Dann gilt $\mA_1 \otimes \ldots \otimes \mA_n$ wird von den Mengen $E_1 \times E_2 \times \ldots \times E_n$ für $E_j\in\xi_j$ erzeugt.

        \begin{proof}
            Es gilt $\mA = \mA_1 \otimes \ldots \otimes \mA_n = \sigma\of{p_1, \ldots, p_n}$ genau dann, wenn $p_j$ $\mA-\mA_j$-messbar ist und $\mA$ die kleinste solcher $\sigma$-Algebren ist. Das heißt nach $\mA_j = \sigma\of{\xi_j}$ gilt $p_j$ ist $\mA-\mA_j$-meßbar. Das heißt
            \begin{align*}
                \equivalent p_j^{-1}\of{E_j} &\in\mA\quad\forall E_j \in\xi_j\\
                p_j^{-1}\of{E_j} &= X_1 \times \ldots \times X_{j-1}\times E_j \times X_{j+1}\times X_n\\
                \underbrace{p_1^{-1}\of{E_1} \cap p_2^{-1}\of{E_2} \cap \ldots \cap p_n^{-1}\of{e_n}}_{\text{ist meßbar}} &= E_1 \times E_2 \times \ldots \times E_n\in\mA_1 \otimes \ldots \otimes \mA_n\\
                \impl \sigma\of{\set{E_1 \times \ldots \times E_n: E_j \in\xi_j}} &\subseteq \mA_1 \otimes \mA_2 \otimes \ldots \otimes \mA_n
                \intertext{Umgekehrt ist}
                E_1 \times E_2 \times \ldots \times E_n &\subseteq \mA
            \end{align*}
            für alle $E_j \in\xi_j$. Nehme für $1\leq j\leq j$
            \begin{align*}
                F_k &\coloneqq E_{1,k} \times \ldots \times E_{j-1, k} \times E_j \times E_{j+1, k} \times \ldots \times E_{n,k} \subseteq \mA\\
                F_k &\nearrow X_1 \times \ldots \times X_{j-1}\times E_j \times X_{j+1} \times \ldots \times X_n = p_j^{-1}\of{E_j}\\
                \impl p_j &\text{ ist }\mA-\mA_j \text{-messbar}\\
            \end{align*}
            Das heißt $\mA$ ist die $\sigma$-Algebra, die von $E_1 \times E_2 \times \ldots \times E_n$ ($E_j \in\xi_j$) erzeugt wird.\qedhere
        \end{proof}
    \end{satz}

    \begin{bemerkung}
        Sei $\mA_1 = \set{\emptyset, X_1}$ und $\mA_2 = \set{\emptyset, A, \comp{A}, X_2}$. Außerdem sei $\xi_1 = \set{\emptyset}$, $\xi_2 = \mA_2$. Dann ist
        \begin{align*}
            \xi_1 \times \xi_2 &= \set{E_1 \times E_2: E_1\in\xi_1, E_2\in\xi_2} = \emptyset\\
            \impl \sigma\of{\xi_1 \times \xi_2} &= \sigma\of{\emptyset} = \set{\emptyset, X_1 \times X_2} &\neq \mA_1 \otimes \mA_2
        \end{align*}
        Das heißt wir brauchen die Bedingung, dass es ausschöpfende Folgen in den Erzeugern gibt, im vorherigen Satz.
    \end{bemerkung}

    \begin{beispiel}
        Sei $X_j = \R$ und $\mA_j = \mB^1$, $\xi_j = \mJ^1$. Dann ist $\xi_1\times \ldots \times \xi_n = \mJ^n$. Nach Satz~\ref{satz:prod-sigma} gilt
        \begin{align*}
            \mB^1 \otimes\ldots \otimes \mB^1 &= \sigma\of{\xi_1 \times \ldots \times \xi_n} = \sigma\of{\mJ^n} = \mB^n
            \intertext{Analog gilt auch}
            \mB^{n} \otimes \mB^{m} &= \mB^{n+m}
        \end{align*}
        Das heißt die $d$-dimensionalen Borel-Mengen sind bereits als $d$-fache Produkt-$\sigma$-Algebra von den eindimensionalen Borel-Mengen definiert.
    \end{beispiel}

    \newpage


    \section{[*] Produktmaße und der Satz von Fubini-Tonelli}

    \subsection{Eindeutigkeit von Produktmaßen}
    \thispagestyle{pagenumberonly}

    Situation: Wir haben Maßräume $\pair{X_j, \mA_j, \mu_j}$, $j\in\set{1, \ldots, n}$ und Erzeuger $\xi_j$ von $\mA_j$. Wann gibt es dann ein Maß $\pi$ auf $\mA_1 \otimes \ldots\otimes \mA_n$ (Produktmaß) mit
    \begin{align*}
        \pi\of{E_1 \times \ldots \times E_n} &= \mu_1\of{E_1}\cdot \ldots \cdot \mu_n\of{E_n}\quad\forall E_j\in\xi_j
    \end{align*}
    und wann ist ein solches Maß eindeutig?

    ´\begin{satz} % Satz 2
         Seien $\xi_j$ $\cap$-stabile Erzeuger von $\sigma$-Algebren $\mA_j$, $j\in\set{1, \ldots, m}$ und es gebe Folgen $(E_{j,k})_{k\in\N} \subseteq \xi_j$ mit $E_{j,k}\nearrow \xi_j$ für $k\toinf$ mit $\mu_j\of{E_{j,k}} < \infty~\forall k$. Dann gibt es höchstens ein Produktmaß $\pi$ mit der Eigenschaft
         \begin{align*}
             \pi\of{E_1 \times \ldots \times E_n} &= \mu_1\of{E_1}\cdot \ldots \cdot \mu_n\of{E_n}\quad\forall E_j\in\xi_j
         \end{align*}

         \begin{proof}
             Sei $\xi \coloneqq \xi_1 \times \ldots \times \xi_n$. Dann folgt nach Satz~\ref{satz:prod-sigma}, dass $\mA_1 \otimes \ldots \otimes \mA_n = \sigma\of{\xi}$ und $\xi$ ist $\cap$-stabil. Außerdem ist
             \begin{align*}
                 \pair{\prod_{j=1}^{n} E_j} \cap \pair{ \prod_{j=1}^{n} F_j} &= \prod_{j=1}^{n} \pair{E_j \cap F_j}
                 \intertext{was induktiv aus ${E_1 \times E_2 \cap F_1 \times F_2 = \pair{E_1 \cap F_1 }\times \pair{E_2 \cap F_2}}$ ergibt. Dann gilt}
                 E_k &\coloneqq E_{1, k}\times \ldots \times E_{n,k}\nearrow X = X_1 \times \ldots \times X_n\\
                 \mu_j\of{E_{j,k}} &< \infty
                 \intertext{Wenn wir nun definieren, dass}
                 \pi\of{E_1 \times \ldots \times E_n} &\coloneqq \prod_{j=1}^{n} \mu_j\of{E_j}
             \end{align*}
             dann ist $\pi$ durch die Definition als Prämaß nach dem Eindeutigkeitssatz eindeutig auf ein Maß fortsetzbar. Das heißt es gibt höchstens ein solches Maß.\qedhere
         \end{proof}
    \end{satz}

    \subsection{Existenz von Produktmaßen}

    Idee für $n=2$: Für $\pair{X_1, \mA_1, \mu_1}$ und $\pair{X_2, \mA_2, \mu_2}$ gilt
    \begin{align*}
        \mu_1\of{A_1} &= \int_{}^{} \charfunc_{A_1}\of{x_1} \mu_1{\dif x_1}\\
        \mu_2\of{A_2} &= \int_{}^{} \charfunc_{A_2}\of{x_2} \mu_2\of{\dif x_2}
        \intertext{Wir hätten gerne, dass}
        \pair{\mu_1\otimes\mu_2}\of{A_1\times A_2} \annot[{&}]{=}{!} \mu_1\of{A_1}\mu\of{A_2}\\
        &= \pair{\int_{}^{} \charfunc_{A_1}\of{x_1} \mu\of{\dif x_1}}\pair{ \int_{}^{} \charfunc_{A_2}\of{x_2} \mu\of{\dif x_2}}
        \intertext{Vergleiche mit}
        &\int_{}^{} \pair{ \int_{}^{} \charfunc_{A_1}\of{x_1}\charfunc_{A_2}\of{x_2} \mu_2\of{x_2}} \mu_1\of{x_1}\\
        &= \int_{}^{} \pair{ \int_{}^{} \charfunc_{A_2}\of{x_2}\mu\of{\dif x_2}}\charfunc_{A_1}\of{x_1} \mu\of{\dif x_1}\\
        &= \pair{\int_{}^{} \charfunc_{A_1}\of{x_1} \mu_2\of{\dif x_1}}\pair{ \int_{}^{} \charfunc_{A_2}\of{x_2} \mu_1\of{\dif x_2}}
    \end{align*}

    \begin{konstruktion}[Schnitte im Fall $n=2$]
        \marginnote{[31. Jan]}
        Es sei $Q\subseteq X_1\times X_2$ für Messräume $\pair{X_1, \mA_1},\pair{X_2, \mA_2}$. Dann definieren wir
        \begin{align*}
            Q_{x_1} &\coloneqq \set{x_2 \in X_2: \pair{x_1, x_2}\in Q}\subseteq X_2\\
            Q_{x_2} &\coloneqq \set{x_1 \in X_1: \pair{x_1, x_2}\in Q}\subseteq X_1\\
            \intertext{Wir haben dann}
            \charfunc_{Q_{x_1}}\of{x_2} &= \charfunc_{Q}\of{x_1, x_2} = \charfunc_{Q_{x_2}}\of{x_1}
        \end{align*}
    \end{konstruktion}

    \begin{lemma} % Lemma 1
        Sei $Q\in\mA_1 \times \mA_2$. Dann gilt $Q_{x_1}\in \mA_2$ und $Q_{x_2}\in\mA_1$ $\forall x_1\in X_1, x_2\in X_2$.

        \begin{proof}
            Prinzip der guten Mengen. Für $x_1$-Schnitte ($x_2$-Schnitte funktionieren analog) sei
            \begin{align*}
                \mC &\coloneqq \set{Q\in\mA_1 \otimes\mA_2: Q_{x_1}\in\mA_2}\subseteq \mA_1 \otimes\mA_2
                \intertext{Wir zeigen, dass $\mC$ eine $\sigma$-Algebra ist:}
                \pair{X_1 \times X_2}_{x_1} &= X_2\\
                \pair{\comp{Q}}_{x_1} &= \pair{X_1 \times X_2 \setminus Q}_{x_1} = \pair{X_2\setminus Q_{x_1}} = \comp{\pair{Q_{x_1}}}\\
                \pair{\bigcup_{n=1}^{\infty} Q_n}_{x_1} &= \bigcup_{n=1}^{\infty} \pair{Q_n}_{x_1}
                \intertext{Das heißt $\mC$ ist eine $\sigma$-Algebra in $X_1\times X_2$ und}
                A_1 \times A_2 &\in \mC\quad\forall A_1\in\mA_1, A_2\in\mA_2\\
                \impl \sigma\of{\mA_1 \times \mA_2} &\subseteq \mC \subseteq \mA_1\otimes\mA_2\\
                \impl \mA_1\otimes\mA_2&\subseteq \mC \subseteq \mA_1\otimes\mA_2\\
                \impl \mC &= \mA_1\otimes\mA_2
                \intertext{Noch zu zeigen: $x_1 \mapsto \mu_2\of{Q_{x_1}}$ ist $\mA_1$-messbar. Definieren}
                s_Q\of{x_1} &\coloneqq \mu_2\of{Q_{x_1}}\\
                \mD &\coloneqq \set{Q\in\mA_1\otimes\mA_2: s_Q\text{ ist $\mA_1$-messbar}}\\
                s_{\comp{Q}}\of{x_1} &= \mu_2\of{\comp{Q_{x_1}}} = \mu_2\of{X_2 \setminus Q_{x_1}} = \mu_2\of{X_2} - \mu_2\of{Q_{x_1}}
                \intertext{Für den letzten Schritt brauchen wir allerdings, dass $\mu_2\of{X_2} < \infty$}
                s_{\bigcup_{n=1}^{\infty} Q_{n}}\of{x_1} &= \mu_2\of{\pair{\bigcup_{n=1}^{\infty} Q_n}_{x_1}} = \mu_2\of{\bigcup_{n\in\N} \pair{Q_n}_{x_1}}\\
                &= \sum_{n=1}^{\infty} \mu_2\of{\pair{Q_n}_{x_1}}
            \end{align*}
        \end{proof}
    \end{lemma}

    \marginnote{[*03. Feb]}

    \newpage

    \begin{satz}[Tonelli]
        \label{satz:tonelli}
        Sei $f\in\mM_+\of{X_1\times X_2, \mA_1 \otimes \mA_2}$. Dann ist
        \begin{align*}
            x_1 &\mapsto \int_{x_2}^{} f\of{x_1, x_2} \mu\of{\dif x_2}
            \intertext{$\mA_1$-messbar und}
            x_2 &\mapsto \int_{x_1}^{} f\of{x_1, x_2}\mu\of{\dif x_1}
            \intertext{ist $\mA_2$-messbar. Außerdem gilt insgesamt}
            \int_{X_1 \times X_2}^{} f \dif\of{\mu_1\otimes \mu_2} &= \int_{X_1}^{} \int_{X_2}^{} f\of{x_1, x_2}\mu_2\of{\dif x_2}\mu_1\of{\dif x_1}\\
            &= \int_{X_2}^{} \int_{X_1}^{} f\of{x_1, x_2}\mu_1\of{\dif x_1}\mu_2\of{\dif x_2}\\
        \end{align*}

        \begin{proof}[Beweisskizze]
            Klar, wenn $f$ eine einfache Funktion $\geq 0$ ist. Ist $f$ eine meßbare Funktion auf $X_1\times X_2$, dann verwenden wir eine wachsende Folge einfacher Funktionen $(u_n)_n \nearrow f$. Dann ist
            \begin{align*}
                \int_{X_1 \times X_2}^{} u_n \dif\of{\mu_1\otimes\mu_2} &= \int_{}^{} \int_{X_1}^{} u_n\of{x_1, x_2}\mu_2\of{\dif x_2}\mu_1\of{\dif x_1}
            \end{align*}
            und aus monotoner Konvergenz folgt dann die Behauptung. Dass sich die Integrale taushcen lassen, zeigt sich analog.
        \end{proof}
    \end{satz}

    \begin{satz}[Fubini] % Satz 7
        \label{satz:fubini}
        Seien $\pair{X_j, \mA_j, \mu_j}$ $\sigma$-endliche Maßräume für $j=1, 2$ und $f: X_1 \times X_2 \to \overline{\R}$ $\mA_1\otimes\mA_2$-messbare Funktionen welche bezüglich des Produktmaßes $\mu_1\otimes\mu_2$ integrierbar ist. Dann ist $x_1\mapsto f\of{x_1, x_2}$ für festes $x_2$ $\mA_1$-messbar und für $\mu_2$-fast-alle $x_2$ bezüglich $\mu_1$ integrierbar. Das heißt
        \begin{align*}
            \int_{}^{} &f\of{x_1, x_2} \mu_1\of{\dif x_1}
            \intertext{existiert für $x_2$ fest. Und umgekehrt ist $x_2 \mapsto f\of{x_1, x_2}$ $\mA_2$-messbar und für alle $x_1\in X_1$ $\mu_2$-integrierbar. Ferner gilt}
            \int_{X_1 \times X_2}^{} f \dif\of{\mu_1\otimes \mu_2} &= \int_{X_1}^{} \pair{\int_{X_2}^{} f\of{x_1, x_2} \mu_2\of{\dif x_2}} \mu_1\of{\dif x_1}\\
            &= \int_{X_2}^{} \pair{\int_{X_1}^{} f\of{x_1, x_2} \mu_1\of{\dif x_1}} \mu_2\of{\dif x_2}
        \end{align*}
    \end{satz}

    \marginnote{[06. Feb]}

    \noindent\textbf{Wichtig}: Die Voraussetzung für Satz~\ref{satz:fubini} ist bereits, dass $f$ $\mu_1\otimes \mu_2$-integrierbar ist. Beziehungsweise äquivalent, dass $\abs{f}$ $\mu_1\otimes \mu_2$-integrierbar ist. Zu wissen, dass das der Fall ist, ist eigentlich gerade unser Ursprungsproblem, aber hier können wir Satz~\ref{satz:tonelli} anwenden, da $\abs{f}\geq 0$. Das heißt, genau dann, wenn nach Tonelli eines der einzelnen Integrale von $\abs{f}$ endlich ist, können wir dann Fubini anwenden.

    \begin{beispiel}
        Sei $f\of{x_1, x_2} \coloneqq \charfunc_{\Q}\of{x_1}$. Dann ist
        \begin{align*}
            \int_{}^{} f\of{x_1,x_2}\lambda^1\of{\dif x_1} &= 0\quad\forall x_2\in X_2\\
            \int_{}^{} f\of{x_1, x_2}\lambda^1\of{\dif x_2} &= \int_{}^{} \charfunc_{\Q}\of{x_1}\lambda^1\of{\dif x_2}\\
            &= \charfunc_{\Q}\of{x_1}\lambda^1\of{\R} = \infty \times \charfunc_{\Q}\of{x_1}\\
            &= \begin{cases}
                   0 &x_1\in\R\setminus\Q\\
                   \infty &x_1\in\Q
            \end{cases}
            \intertext{Zerlege $f = f_+-f_-$. Dann ist $\abs{f}$ $\mu_1\otimes\mu_2$-integrierbar, genau dann, wenn $f_+, f_-$ $\mu_1\otimes\mu_2$-integrierbar sind. Nach Tonelli folgt}
            \int_{}^{} \int_{}^{} f_{\pm}\of{x_1, x_2}\mu_2\of{\dif x_2}\mu_1\of{\dif x_1} &< \infty\\
            \impl \int_{}^{} f_{\pm}\of{x_1, x_2}\mu_2\of{\dif x_2} &< \infty
            \intertext{für $\mu_1$-fast-alle $x_1$}
            \int_{}^{} f_{\pm}\of{x_1, x_2}\mu_1\of{\dif x_1} &< \infty
            \intertext{für $\mu_2$-fast-alle $x_2$ nach Tonelli und Symmetrie. Das heißt nach Tonelli}
            \int_{X_2}^{} \int_{X_1}^{} f_{\pm}\of{x_1, x_2}\mu_1\of{\dif x_1}\mu_2\of{\dif x_2} &= \int_{}^{}f_{\pm} \dif\of{\mu_1\otimes\mu_2}\\
            \impl \int_{}^{} f \dif\of{\mu_1\otimes \mu_2} &= \int_{}^{} \int_{}^{} f_+\of{x_1, x_2}\mu_1\of{\dif x_1}\mu_2\of{\dif x_2}\\
            &~~~-\int_{}^{} \int_{}^{} f_-\of{x_1, x_2}\mu_1\of{\dif x_1}\mu_2\of{\dif x_2}\\
            &= \int_{}^{} f\of{x_1, x_2}\mu_1\of{\dif x_1} \mu_2\of{\dif x_2}
            \intertext{Mit den Einzelintegralen oben folgt dann}
            \impl \int_{}^{} \int_{}^{} f\dif\of{\lambda^1\otimes\lambda^2} &= 0
        \end{align*}
    \end{beispiel}

    \begin{satz}[Extrem nützlich] % Satz 8
        Sei $\pair{X, \mA, \mu}$ ein $\sigma$-endlicher Maßraum und $f: x\to\interv{0, \infty}$ messbar sowie $\phi: \interv{0, \infty}\to\interv{0,\infty}$ monoton wachsend, stetig, $\phi\of{0} = 0$ und $\phi$ diff.bar auf $\pair{0, \infty}$. Dann ist
        \begin{align*}
            \int_{}^{} \phi\circ f \dif \mu &\coloneqq \int_{X}^{} \phi\of{f} \dif \mu = \int_{0}^{\infty} \phi'\of{t}\mu\of{\set{f\geq t}}\lambda^1\of{\dif t}
        \end{align*}

        \begin{proof}
            \begin{align*}
                \phi\of{a} &= \phi\of{0} + \int_{0}^{a} \phi'\of{s} \dif s\\
                &= \int_{0}^{a} \phi'\of{s} \dif s = \int_{0}^{\infty} \phi'\of{s}\charfunc_{\interv{0, a}}\of{s} \dif s\\
                \impl \phi\of{f\of{x}} &= \int_{0}^{\infty} \phi'\of{s}\charfunc_{\interv{0, f\of{x}}}\of{s} \dif s\\
                \impl \int_{}^{} \phi\circ f \dif \mu &= \int_{}^{} \pair{ \int_{0}^{\infty} \phi'\of{s}\charfunc_{\interv{0, f\of{x}}}\of{s} \dif s} \mu\of{\dif x}\\
                &= \int_{}^{} \pair{ \int_{0}^{\infty} \phi'\of{s}\charfunc_{\set{f\geq s}}\of{x} \dif s} \mu\of{\dif x}\\
                \intertext{Wir müssten uns jetzt noch überlegen, dass $\pair{s, x}\mapsto \phi'\of{s}\charfunc_{\set{f\geq s}}\of{x} \geq 0$ $\mB^1\otimes \mA$-messbar ist. Dann können wir mit Tonelli folgern, dass}
                &= \int_{0}^{\infty} \pair{ \int_{}^{} \phi'\of{s}\charfunc_{\set{f\geq s}} \dif \mu} \dif s\\
                &= \int_{0}^{\infty} \phi'\of{s}\mu\of{\set{f\geq s}} \dif s\qedhere
            \end{align*}
        \end{proof}

        \begin{proof}[\anf{Richtiger} Beweis]
            Wir definieren $\mB_+\coloneqq\interv{0, \infty}\cap \mB^1$ und $\lambda_+ \coloneqq \lambda^1\vert_{\mB_+}$. $F\of{x, t} = \pair{f\of{x}, t}\in\R^2$. $F: X\times \R_+\to\R^2$ ist $\mA\otimes \mB_+ - \mB^2$-messbar. Dann ist $p_j\of{s_1, s_2}\coloneqq s_j$ die Koordinatenprojektion und $p_1 \circ F = f$ sowie $f_2 \circ F\of{x, t} = t$. Wir betrachten $E\coloneqq \set{\pair{x, t}\in X \times \R_+: f\of{x}\geq t}\in\mA\otimes\mB_+$. Dann ist $E = F^{-1}\of{\set{\pair{s, t}: s\geq t}}\in\mA\times \mB_+$. Nach Tonelli gilt
            \begin{align*}
                \int_{}^{} \pair{ \int_{}^{} \phi'\of{t}\charfunc_{E}\of{x, t}\lambda_+\of{\dif t}} \mu\of{\dif x} &= \int_{}^{} \int_{}^{} \phi'\of{t}\charfunc_{E}\of{x, t} \mu\of{\dif x}\lambda_+\of{\dif t}\\
                &= \int_{}^{} \phi'\of{t}\pair{ \int_{}^{} \charfunc_{E}\of{x, t}\mu\of{\dif x}} \lambda_+\of{\dif t}\\
                &= \int_{}^{} \phi'\of{t}\mu\of{\set{f\geq t}} \lambda_+\of{\dif t}
                \intertext{Ferner}
                \int_{0}^{a} \phi'\of{t}\lambda_+\of{\dif t} &= \int_{\rinterv{0, a}}^{} \phi'\of{t} \dif t\\
                &= \lim_{n\toinf} \int_{\interv{\frac{1}{n}, a}}^{} \phi'\of{t} \dif t = \lim_{n\toinf} \phi\of{a} - \phi\of{\frac{1}{n}} = \phi\of{a}\\
                \int_{}^{} \phi'\of{t}\charfunc_{E}\of{x, t}\lambda_+\of{\dif t}&= \int_{\pair{0, \infty}}^{} \phi'\of{t}\charfunc_{\set{f\of{x}\geq t}} \lambda^1\of{\dif t}\\
                &= \int_{0}^{f\of{x}} \phi'\of{t} \dif t = \phi\of{f\of{x}}\\
                \impl \int_{}^{} \int_{}^{} \phi'\of{t}\charfunc_{E}\of{x, t}\lambda_+\of{\dif t}\mu\of{\dif x} &= \int_{}^{} \phi\of{f\of{x}} \mu\of{\dif x}\\
                &= \int_{}^{} \phi\circ f \dif \mu\qedhere
            \end{align*}
        \end{proof}
    \end{satz}

    \begin{beispiel}
        Sei $\phi\of{a} = a$, $\phi' = 1$. Dann ist $\phi \circ f = f$ und
        \begin{align*}
            \int_{}^{} f \dif \mu &= \int_{0}^{\infty} \phi'\of{t}\mu\of{\set{f\geq t}} \lambda^1\of{\dif t}\\
            &= \int_{0}^{\infty} \mu\of{\set{f\geq t}} \dif t
        \end{align*}
        Das heißt wir messen an jeder Stelle $t$ das Maß des Levels, wo $f\geq t$. Da das in der Vorstellung einer Funktion, die stufenweise steigt und dann stufenweise wieder sinkt (was dem Querschnitt eines mehrstöckigen Hochzeitskuchen ähnelt), nennt sich diese Darstellung auch \textit{Wedding cake representation}.
    \end{beispiel}

    \begin{beispiel}
        Sei $\phi\of{a} = a^p \impl \phi'\of{a} = pa^{p-1}$. Dann gilt
        \begin{align*}
            \int_{}^{} f^p \dif\mu &= p \int_{0}^{\infty} t^{p-1}\mu\of{\set{f\geq t}} \dif t\\
            \int_{}^{} \abs{f}^p \dif\mu &= p \int_{0}^{\infty} t^{p-1}\mu\of{\set{\abs{f}\geq t}} \dif t
        \end{align*}
    \end{beispiel}

\end{document}
